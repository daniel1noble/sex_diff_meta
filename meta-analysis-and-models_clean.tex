\documentclass[]{article}
\usepackage{lmodern}
\usepackage{amssymb,amsmath}
\usepackage{ifxetex,ifluatex}
\usepackage{fixltx2e} % provides \textsubscript
\ifnum 0\ifxetex 1\fi\ifluatex 1\fi=0 % if pdftex
  \usepackage[T1]{fontenc}
  \usepackage[utf8]{inputenc}
\else % if luatex or xelatex
  \ifxetex
    \usepackage{mathspec}
  \else
    \usepackage{fontspec}
  \fi
  \defaultfontfeatures{Ligatures=TeX,Scale=MatchLowercase}
\fi
% use upquote if available, for straight quotes in verbatim environments
\IfFileExists{upquote.sty}{\usepackage{upquote}}{}
% use microtype if available
\IfFileExists{microtype.sty}{%
\usepackage[]{microtype}
\UseMicrotypeSet[protrusion]{basicmath} % disable protrusion for tt fonts
}{}
\PassOptionsToPackage{hyphens}{url} % url is loaded by hyperref
\usepackage[unicode=true]{hyperref}
\hypersetup{
            pdftitle={A meta-analysis of sex differences in animal personality: no evidence for greater male variability},
            pdfauthor={Lauren Harrison, Daniel Noble \& Michael Jennions},
            pdfborder={0 0 0},
            breaklinks=true}
\urlstyle{same}  % don't use monospace font for urls
\usepackage[margin=1in]{geometry}
\usepackage{color}
\usepackage{fancyvrb}
\newcommand{\VerbBar}{|}
\newcommand{\VERB}{\Verb[commandchars=\\\{\}]}
\DefineVerbatimEnvironment{Highlighting}{Verbatim}{commandchars=\\\{\}}
% Add ',fontsize=\small' for more characters per line
\usepackage{framed}
\definecolor{shadecolor}{RGB}{248,248,248}
\newenvironment{Shaded}{\begin{snugshade}}{\end{snugshade}}
\newcommand{\KeywordTok}[1]{\textcolor[rgb]{0.13,0.29,0.53}{\textbf{#1}}}
\newcommand{\DataTypeTok}[1]{\textcolor[rgb]{0.13,0.29,0.53}{#1}}
\newcommand{\DecValTok}[1]{\textcolor[rgb]{0.00,0.00,0.81}{#1}}
\newcommand{\BaseNTok}[1]{\textcolor[rgb]{0.00,0.00,0.81}{#1}}
\newcommand{\FloatTok}[1]{\textcolor[rgb]{0.00,0.00,0.81}{#1}}
\newcommand{\ConstantTok}[1]{\textcolor[rgb]{0.00,0.00,0.00}{#1}}
\newcommand{\CharTok}[1]{\textcolor[rgb]{0.31,0.60,0.02}{#1}}
\newcommand{\SpecialCharTok}[1]{\textcolor[rgb]{0.00,0.00,0.00}{#1}}
\newcommand{\StringTok}[1]{\textcolor[rgb]{0.31,0.60,0.02}{#1}}
\newcommand{\VerbatimStringTok}[1]{\textcolor[rgb]{0.31,0.60,0.02}{#1}}
\newcommand{\SpecialStringTok}[1]{\textcolor[rgb]{0.31,0.60,0.02}{#1}}
\newcommand{\ImportTok}[1]{#1}
\newcommand{\CommentTok}[1]{\textcolor[rgb]{0.56,0.35,0.01}{\textit{#1}}}
\newcommand{\DocumentationTok}[1]{\textcolor[rgb]{0.56,0.35,0.01}{\textbf{\textit{#1}}}}
\newcommand{\AnnotationTok}[1]{\textcolor[rgb]{0.56,0.35,0.01}{\textbf{\textit{#1}}}}
\newcommand{\CommentVarTok}[1]{\textcolor[rgb]{0.56,0.35,0.01}{\textbf{\textit{#1}}}}
\newcommand{\OtherTok}[1]{\textcolor[rgb]{0.56,0.35,0.01}{#1}}
\newcommand{\FunctionTok}[1]{\textcolor[rgb]{0.00,0.00,0.00}{#1}}
\newcommand{\VariableTok}[1]{\textcolor[rgb]{0.00,0.00,0.00}{#1}}
\newcommand{\ControlFlowTok}[1]{\textcolor[rgb]{0.13,0.29,0.53}{\textbf{#1}}}
\newcommand{\OperatorTok}[1]{\textcolor[rgb]{0.81,0.36,0.00}{\textbf{#1}}}
\newcommand{\BuiltInTok}[1]{#1}
\newcommand{\ExtensionTok}[1]{#1}
\newcommand{\PreprocessorTok}[1]{\textcolor[rgb]{0.56,0.35,0.01}{\textit{#1}}}
\newcommand{\AttributeTok}[1]{\textcolor[rgb]{0.77,0.63,0.00}{#1}}
\newcommand{\RegionMarkerTok}[1]{#1}
\newcommand{\InformationTok}[1]{\textcolor[rgb]{0.56,0.35,0.01}{\textbf{\textit{#1}}}}
\newcommand{\WarningTok}[1]{\textcolor[rgb]{0.56,0.35,0.01}{\textbf{\textit{#1}}}}
\newcommand{\AlertTok}[1]{\textcolor[rgb]{0.94,0.16,0.16}{#1}}
\newcommand{\ErrorTok}[1]{\textcolor[rgb]{0.64,0.00,0.00}{\textbf{#1}}}
\newcommand{\NormalTok}[1]{#1}
\usepackage{graphicx,grffile}
\makeatletter
\def\maxwidth{\ifdim\Gin@nat@width>\linewidth\linewidth\else\Gin@nat@width\fi}
\def\maxheight{\ifdim\Gin@nat@height>\textheight\textheight\else\Gin@nat@height\fi}
\makeatother
% Scale images if necessary, so that they will not overflow the page
% margins by default, and it is still possible to overwrite the defaults
% using explicit options in \includegraphics[width, height, ...]{}
\setkeys{Gin}{width=\maxwidth,height=\maxheight,keepaspectratio}
\IfFileExists{parskip.sty}{%
\usepackage{parskip}
}{% else
\setlength{\parindent}{0pt}
\setlength{\parskip}{6pt plus 2pt minus 1pt}
}
\setlength{\emergencystretch}{3em}  % prevent overfull lines
\providecommand{\tightlist}{%
  \setlength{\itemsep}{0pt}\setlength{\parskip}{0pt}}
\setcounter{secnumdepth}{0}
% Redefines (sub)paragraphs to behave more like sections
\ifx\paragraph\undefined\else
\let\oldparagraph\paragraph
\renewcommand{\paragraph}[1]{\oldparagraph{#1}\mbox{}}
\fi
\ifx\subparagraph\undefined\else
\let\oldsubparagraph\subparagraph
\renewcommand{\subparagraph}[1]{\oldsubparagraph{#1}\mbox{}}
\fi

% set default figure placement to htbp
\makeatletter
\def\fps@figure{htbp}
\makeatother

\usepackage{booktabs}
\usepackage{longtable}
\usepackage{array}
\usepackage{multirow}
\usepackage{wrapfig}
\usepackage{float}
\usepackage{colortbl}
\usepackage{pdflscape}
\usepackage{tabu}
\usepackage{threeparttable}
\usepackage{threeparttablex}
\usepackage[normalem]{ulem}
\usepackage{makecell}

\title{A meta-analysis of sex differences in animal personality: no evidence
for greater male variability}
\author{Lauren Harrison, Daniel Noble \& Michael Jennions}
\date{last updated: 16/09/2021}

\begin{document}
\maketitle

{
\setcounter{tocdepth}{2}
\tableofcontents
}
\section{Supplementary Material - Data checks, analysis and meta-a
models}\label{supplementary-material---data-checks-analysis-and-meta-a-models}

\section{Data Set-up}\label{data-set-up}

\subsection{Load packages}\label{load-packages}

\begin{Shaded}
\begin{Highlighting}[]
\CommentTok{# Clear work space}
  \CommentTok{# rm(list=ls())}

\CommentTok{# Install CRAN packages}
  \KeywordTok{library}\NormalTok{(}\StringTok{"pacman"}\NormalTok{)}
\end{Highlighting}
\end{Shaded}

\begin{verbatim}
## Warning: package 'pacman' was built under R version 3.5.2
\end{verbatim}

\begin{Shaded}
\begin{Highlighting}[]
\CommentTok{# Install orchard plot and metaAidR packages from GitHub}
\NormalTok{  devtools}\OperatorTok{::}\KeywordTok{install_github}\NormalTok{(}\StringTok{"itchyshin/orchard_plot"}\NormalTok{, }\DataTypeTok{subdir =} \StringTok{"orchaRd"}\NormalTok{, }\DataTypeTok{force =} \OtherTok{TRUE}\NormalTok{, }\DataTypeTok{build_vignettes =} \OtherTok{TRUE}\NormalTok{); }
\end{Highlighting}
\end{Shaded}

\begin{verbatim}
## Error in get(genname, envir = envir) : object 'testthat_print' not found
## 
##   
   checking for file ‘/private/var/folders/0b/pxghylq157gfhs1vrzdpx2gc0000gq/T/RtmpyX6Pi3/remotescf00568b00af/itchyshin-orchard_plot-27d6281/orchaRd/DESCRIPTION’ ...
  
v  checking for file ‘/private/var/folders/0b/pxghylq157gfhs1vrzdpx2gc0000gq/T/RtmpyX6Pi3/remotescf00568b00af/itchyshin-orchard_plot-27d6281/orchaRd/DESCRIPTION’
## 
  
-  preparing ‘orchaRd’:
##    checking DESCRIPTION meta-information ...
  
v  checking DESCRIPTION meta-information
## 
  
-  installing the package to build vignettes
## 
  
   creating vignettes ...
  
v  creating vignettes (1.6s)
## 
  
-  checking for LF line-endings in source and make files and shell scripts
## 
  
-  checking for empty or unneeded directories
## -  looking to see if a ‘data/datalist’ file should be added
## 
  
-  building ‘orchaRd_0.0.0.9000.tar.gz’
## 
  
   
## 
\end{verbatim}

\begin{Shaded}
\begin{Highlighting}[]
\NormalTok{  devtools}\OperatorTok{::}\KeywordTok{install_github}\NormalTok{(}\StringTok{"daniel1noble/metaAidR"}\NormalTok{); }

\NormalTok{pacman}\OperatorTok{::}\KeywordTok{p_load}\NormalTok{(knitr, metafor, dplyr, kableExtra, tidyverse, rotl, phytools, GGally, R.rsp, patchwork, devtools, robumeta, ape, geiger, phytools, phangorn, rlist, orchaRd, metaAidR, corrplot, stringr)}

\CommentTok{# set working directory}
 \CommentTok{# setwd("~/Documents/GitHub/sex_meta/")}
  
\CommentTok{# Source our own functions}
  \KeywordTok{source}\NormalTok{(}\StringTok{"./R/func.R"}\NormalTok{)}
 
\CommentTok{# Set the rerun object to FALSE so that you don't need to re-run all models again. Some take quite a lot of time to run. If FALSE, it will just re-load saved output. }

\NormalTok{  rerun_models =}\StringTok{ }\OtherTok{FALSE}
\end{Highlighting}
\end{Shaded}

\section{Organising data for
analysis}\label{organising-data-for-analysis}

\begin{Shaded}
\begin{Highlighting}[]
\NormalTok{  ## load our original pers dataset and our dataset with all of our sexual selection info}
\NormalTok{      pers <-}\StringTok{ }\KeywordTok{read.csv}\NormalTok{(}\StringTok{"./data/pers_data.csv"}\NormalTok{, }\DataTypeTok{stringsAsFactors =} \OtherTok{FALSE}\NormalTok{)}
\NormalTok{      bodysize <-}\StringTok{ }\KeywordTok{read.csv}\NormalTok{(}\StringTok{"./data/bodysize_SSD.csv"}\NormalTok{, }\DataTypeTok{stringsAsFactors =} \OtherTok{FALSE}\NormalTok{)}

\NormalTok{  ## Merge the two by spp_names columns}
\NormalTok{      pers <-}\StringTok{ }\KeywordTok{merge}\NormalTok{(}\DataTypeTok{x =}\NormalTok{ pers,}
                   \DataTypeTok{y =}\NormalTok{ bodysize[,}\KeywordTok{c}\NormalTok{(}\StringTok{"species_name"}\NormalTok{, }\StringTok{"SSD_index"}\NormalTok{, }\StringTok{"mating_system"}\NormalTok{)],}
                   \DataTypeTok{by=}\StringTok{"species_name"}\NormalTok{, }\DataTypeTok{all.x=}\OtherTok{TRUE}\NormalTok{,  }\DataTypeTok{no.dups =} \OtherTok{TRUE}\NormalTok{)}

\NormalTok{  ## Select only the relevant columns to make life easier}
\NormalTok{      pers_new <-}\StringTok{ }\NormalTok{pers }\OperatorTok\StringTok{ }
\StringTok{          }\KeywordTok{select}\NormalTok{(study_ID, year, species_name, SSD_index, taxo_group, data_type, personality_trait, male_n, male_mean_conv, }
\NormalTok{          male_sd_conv, female_n, female_mean_conv, female_sd_conv, depend, directionality, spp_name_phylo, mating_system, }
\NormalTok{          age, population, study_environment, study_type, measurement_type)}

\NormalTok{  ## Add in observation level random effect (metafor doesn't do this, need to do it manually)}
\NormalTok{      pers_new <-}\StringTok{ }\NormalTok{pers_new }\OperatorTok\StringTok{ }
\StringTok{          }\KeywordTok{group_by}\NormalTok{(taxo_group) }\OperatorTok\StringTok{ }
\StringTok{          }\KeywordTok{mutate}\NormalTok{(}\DataTypeTok{obs =} \DecValTok{1}\OperatorTok{:}\KeywordTok{length}\NormalTok{(study_ID))  }
\end{Highlighting}
\end{Shaded}

\section{Calculating effect sizes}\label{calculating-effect-sizes}

\subsection{means - SMD using Hedges' g and variability -
lnCVR}\label{means---smd-using-hedges-g-and-variability---lncvr}

\begin{Shaded}
\begin{Highlighting}[]
\NormalTok{ ## SMD (Hedges' g)}
\NormalTok{   pers_new <-}\StringTok{ }\KeywordTok{escalc}\NormalTok{(}\DataTypeTok{measure =} \StringTok{"SMD"}\NormalTok{, }
                     \DataTypeTok{n1i =}\NormalTok{ male_n, }\DataTypeTok{n2i =}\NormalTok{ female_n,}
                     \DataTypeTok{m1i =}\NormalTok{ male_mean_conv, }\DataTypeTok{m2i =}\NormalTok{ female_mean_conv,}
                     \DataTypeTok{sd1i =}\NormalTok{ male_sd_conv, }\DataTypeTok{sd2i =}\NormalTok{ female_sd_conv, }\DataTypeTok{data =}\NormalTok{ pers_new, }\DataTypeTok{var.names=}\KeywordTok{c}\NormalTok{(}\StringTok{"SMD_yi"}\NormalTok{,}\StringTok{"SMD_vi"}\NormalTok{), }\DataTypeTok{append =} \OtherTok{TRUE}\NormalTok{)}
  
\NormalTok{ ## lnCVR}
\NormalTok{   pers_new <-}\StringTok{ }\KeywordTok{escalc}\NormalTok{(}\DataTypeTok{measure =} \StringTok{"CVR"}\NormalTok{,}
                     \DataTypeTok{n2i =}\NormalTok{ female_n, }\DataTypeTok{n1i =}\NormalTok{ male_n,}
                     \DataTypeTok{m2i =}\NormalTok{ female_mean_conv, }\DataTypeTok{m1i =}\NormalTok{ male_mean_conv,}
                     \DataTypeTok{sd2i =}\NormalTok{ female_sd_conv, }\DataTypeTok{sd1i =}\NormalTok{ male_sd_conv, }\DataTypeTok{data =}\NormalTok{ pers_new, }\DataTypeTok{var.names=}\KeywordTok{c}\NormalTok{(}\StringTok{"CVR_yi"}\NormalTok{,}\StringTok{"CVR_vi"}\NormalTok{))}

  \CommentTok{# we have some NAs where one or both sexes have a value of 0 for either mean or SD. Will be easiest to just remove these.}
    
   \CommentTok{# Exclude NAs}
\NormalTok{      pers_new <-}\StringTok{ }\NormalTok{pers_new }\OperatorTok
\StringTok{                    }\KeywordTok{filter}\NormalTok{(}\OperatorTok{!}\KeywordTok{is.na}\NormalTok{(CVR_yi), }\OperatorTok{!}\KeywordTok{is.na}\NormalTok{(SMD_yi))}
    
      \KeywordTok{dim}\NormalTok{(pers_new)    }\CommentTok{# check they've been removed with no issues}
\end{Highlighting}
\end{Shaded}

\section{Data Checks}\label{data-checks}

\subsection{mean-variance relationship in our
dataset}\label{mean-variance-relationship-in-our-dataset}

Looking at the strength of the correlation between the mean and SD to
check that using lnCVR as a measure of variability is valid.\\
If the mean and SD are NOT strongly correlated then using lnCVR is
pointless.

\begin{Shaded}
\begin{Highlighting}[]
\CommentTok{# females and males seperately because they are in different columns}
   
   \CommentTok{# use ggplot to make a scatterplot of females}
\NormalTok{    fem <-}\StringTok{ }\KeywordTok{ggplot}\NormalTok{(pers_new, }\KeywordTok{aes}\NormalTok{(}\DataTypeTok{x =}\NormalTok{ female_mean_conv, }\DataTypeTok{y =}\NormalTok{ female_sd_conv)) }\OperatorTok{+}\StringTok{ }\KeywordTok{geom_point}\NormalTok{()}
    
    \CommentTok{# on log scale}
\NormalTok{    fem }\OperatorTok{+}\StringTok{ }\KeywordTok{scale_x_continuous}\NormalTok{(}\DataTypeTok{trans =} \StringTok{'log10'}\NormalTok{) }\OperatorTok{+}\StringTok{ }\KeywordTok{scale_y_continuous}\NormalTok{(}\DataTypeTok{trans =} \StringTok{'log10'}\NormalTok{)}
\end{Highlighting}
\end{Shaded}

\includegraphics{meta-analysis-and-models_clean_files/figure-latex/mean-variance correlation-1.pdf}

\begin{Shaded}
\begin{Highlighting}[]
    \CommentTok{# mean and SD on log scale to calculate correlation}
\NormalTok{    logfemale_mean <-}\StringTok{ }\KeywordTok{log}\NormalTok{(pers_new}\OperatorTok{$}\NormalTok{female_mean_conv)}
\NormalTok{    logfemale_SD <-}\StringTok{ }\KeywordTok{log}\NormalTok{(pers_new}\OperatorTok{$}\NormalTok{female_sd_conv)}
    
    \CommentTok{# correlation between mean and SD}
    \KeywordTok{cor}\NormalTok{(logfemale_mean, logfemale_SD) }\CommentTok{#0.91}
\end{Highlighting}
\end{Shaded}

\begin{verbatim}
## [1] 0.9182668
\end{verbatim}

\begin{Shaded}
\begin{Highlighting}[]
  \CommentTok{# Males}
    \CommentTok{# use ggplot to make a scatterplot of females}
\NormalTok{    male <-}\StringTok{ }\KeywordTok{ggplot}\NormalTok{(pers_new, }\KeywordTok{aes}\NormalTok{(}\DataTypeTok{x =}\NormalTok{ male_mean_conv, }\DataTypeTok{y =}\NormalTok{ male_sd_conv)) }\OperatorTok{+}\StringTok{ }\KeywordTok{geom_point}\NormalTok{()}
    
    \CommentTok{# on log scale}
\NormalTok{    male }\OperatorTok{+}\StringTok{ }\KeywordTok{scale_x_continuous}\NormalTok{(}\DataTypeTok{trans =} \StringTok{'log10'}\NormalTok{) }\OperatorTok{+}\StringTok{ }\KeywordTok{scale_y_continuous}\NormalTok{(}\DataTypeTok{trans =} \StringTok{'log10'}\NormalTok{)}
\end{Highlighting}
\end{Shaded}

\includegraphics{meta-analysis-and-models_clean_files/figure-latex/mean-variance correlation-2.pdf}

\begin{Shaded}
\begin{Highlighting}[]
    \CommentTok{# mean and SD on log scale to calculate correlation}
\NormalTok{    logmale_mean <-}\StringTok{ }\KeywordTok{log}\NormalTok{(pers_new}\OperatorTok{$}\NormalTok{male_mean_conv)}
\NormalTok{    logmale_SD <-}\StringTok{ }\KeywordTok{log}\NormalTok{(pers_new}\OperatorTok{$}\NormalTok{male_sd_conv)}
    
    \CommentTok{# correlation between mean and SD}
    \KeywordTok{cor}\NormalTok{(logmale_mean, logmale_SD) }\CommentTok{#0.90}
\end{Highlighting}
\end{Shaded}

\begin{verbatim}
## [1] 0.9071225
\end{verbatim}

\subsection{Checking for outliers and removing weird effect
sizes}\label{checking-for-outliers-and-removing-weird-effect-sizes}

This is an important data checking step - here we can identify whether
data has been entered or reported incorrectly (i.e.~outliers)

First, let's look at the funnel plots for lnCVR and SMD. NOTE: these
funnels have had our 2 big outliers already removed.

\begin{Shaded}
\begin{Highlighting}[]
\CommentTok{#funnel plot for lnCVR}

    \KeywordTok{funnel}\NormalTok{(}\DataTypeTok{x =}\NormalTok{ pers_new}\OperatorTok{$}\NormalTok{CVR_yi, }\DataTypeTok{vi =}\NormalTok{ pers_new}\OperatorTok{$}\NormalTok{CVR_vi, }\DataTypeTok{yaxis=}\StringTok{"seinv"}\NormalTok{, }\DataTypeTok{xlim =} \KeywordTok{c}\NormalTok{(}\OperatorTok{-}\DecValTok{10}\NormalTok{, }\DecValTok{10}\NormalTok{))}
\end{Highlighting}
\end{Shaded}

\includegraphics{meta-analysis-and-models_clean_files/figure-latex/lncvr funnel plots-1.pdf}

SMD

\begin{Shaded}
\begin{Highlighting}[]
\CommentTok{#funnel plot for SMD}

    \KeywordTok{funnel}\NormalTok{(}\DataTypeTok{x =}\NormalTok{ pers_new}\OperatorTok{$}\NormalTok{SMD_yi, }\DataTypeTok{vi =}\NormalTok{ pers_new}\OperatorTok{$}\NormalTok{SMD_vi, }\DataTypeTok{yaxis=}\StringTok{"seinv"}\NormalTok{)}
    \KeywordTok{text}\NormalTok{(}\KeywordTok{as.character}\NormalTok{(pers_new}\OperatorTok{$}\NormalTok{study_ID), }\DataTypeTok{x =}\NormalTok{ pers_new}\OperatorTok{$}\NormalTok{SMD_yi, }\DataTypeTok{y =} \DecValTok{1}\OperatorTok{/}\KeywordTok{sqrt}\NormalTok{(pers_new}\OperatorTok{$}\NormalTok{SMD_vi))}
\end{Highlighting}
\end{Shaded}

\includegraphics{meta-analysis-and-models_clean_files/figure-latex/smd funnel plots-1.pdf}

Removing outliers

\begin{Shaded}
\begin{Highlighting}[]
\NormalTok{    pers_new }\OperatorTok\StringTok{ }
\StringTok{    }\KeywordTok{filter}\NormalTok{(study_ID }\OperatorTok{==}\StringTok{ "P015"} \OperatorTok{&}\StringTok{ }\NormalTok{SMD_yi }\OperatorTok{<}\StringTok{ }\OperatorTok{-}\DecValTok{15}\NormalTok{) }\CommentTok{# P015 has 1 large effect size, remove?}
    
      \CommentTok{# filter out that large effect size}
\NormalTok{      pers_new <-}\StringTok{ }\NormalTok{pers_new }\OperatorTok
\StringTok{      }\KeywordTok{filter}\NormalTok{(}\OperatorTok{!}\NormalTok{study_ID }\OperatorTok{==}\StringTok{ "P015"} \OperatorTok{|}\StringTok{ }\OperatorTok{!}\NormalTok{obs }\OperatorTok{==}\StringTok{ "509"}\NormalTok{)}
      
      \KeywordTok{dim}\NormalTok{(pers_new) }
      
    \CommentTok{# checking SMD outliers - inverse SE > 14}
    \KeywordTok{funnel}\NormalTok{(}\DataTypeTok{x =}\NormalTok{ pers_new}\OperatorTok{$}\NormalTok{SMD_yi, }\DataTypeTok{vi =}\NormalTok{ pers_new}\OperatorTok{$}\NormalTok{SMD_vi, }\DataTypeTok{yaxis=}\StringTok{"seinv"}\NormalTok{)}
    \KeywordTok{text}\NormalTok{(}\KeywordTok{as.character}\NormalTok{(pers_new}\OperatorTok{$}\NormalTok{obs), }\DataTypeTok{x =}\NormalTok{ pers_new}\OperatorTok{$}\NormalTok{SMD_yi, }\DataTypeTok{y =} \DecValTok{1}\OperatorTok{/}\KeywordTok{sqrt}\NormalTok{(pers_new}\OperatorTok{$}\NormalTok{SMD_vi), }\DataTypeTok{offset =} \FloatTok{0.8}\NormalTok{)}
    
    \CommentTok{# checking lnCVR outliers}
    \KeywordTok{funnel}\NormalTok{(}\DataTypeTok{x =}\NormalTok{ pers_new}\OperatorTok{$}\NormalTok{CVR_yi, }\DataTypeTok{vi =}\NormalTok{ pers_new}\OperatorTok{$}\NormalTok{CVR_vi, }\DataTypeTok{yaxis=}\StringTok{"seinv"}\NormalTok{, }\DataTypeTok{xlim =} \KeywordTok{c}\NormalTok{(}\OperatorTok{-}\DecValTok{10}\NormalTok{, }\DecValTok{10}\NormalTok{))}
    \KeywordTok{text}\NormalTok{(}\KeywordTok{as.character}\NormalTok{(pers_new}\OperatorTok{$}\NormalTok{obs), }\DataTypeTok{x =}\NormalTok{ pers_new}\OperatorTok{$}\NormalTok{CVR_yi, }\DataTypeTok{y =} \DecValTok{1}\OperatorTok{/}\KeywordTok{sqrt}\NormalTok{(pers_new}\OperatorTok{$}\NormalTok{CVR_vi), }\DataTypeTok{offset =} \FloatTok{0.8}\NormalTok{) }
    
\CommentTok{# Some measures are more physiological/not personality than personality, so probably wise to remove these before we run the models:}
    \CommentTok{# P029 - obs 22, 23, 32 }
    \CommentTok{# P084 - obs 59, 62, 63, 65, 68, 70, 71, 72, 74}
    \CommentTok{# P060 - obs 216, 217}
    \CommentTok{# P211 - obs 230, 245}
    \CommentTok{# P117 - obs 397, 393, 402, 414}
    \CommentTok{# P197 - obs 541, 544, 546, 547}
    \CommentTok{# P069 - obs 669, 672, 673, 682, 683, 684, 686, 694 }
    
    \CommentTok{# remove these by study }
\NormalTok{    pers_new <-}\StringTok{ }\NormalTok{pers_new }\OperatorTok\StringTok{ }\KeywordTok{filter}\NormalTok{(}\OperatorTok{!}\NormalTok{study_ID }\OperatorTok{==}\StringTok{ "P029"} \OperatorTok{|}\StringTok{ }\OperatorTok{!}\NormalTok{obs }\OperatorTok\StringTok{ }\KeywordTok{c}\NormalTok{(}\StringTok{"21"}\NormalTok{, }\StringTok{"25"}\NormalTok{, }\StringTok{"26"}\NormalTok{, }\StringTok{"28"}\NormalTok{, }\StringTok{"31"}\NormalTok{))}
    
\NormalTok{    pers_new <-}\StringTok{ }\NormalTok{pers_new }\OperatorTok\StringTok{ }\KeywordTok{filter}\NormalTok{(}\OperatorTok{!}\NormalTok{study_ID }\OperatorTok{==}\StringTok{ "P084"} \OperatorTok{|}\StringTok{ }\OperatorTok{!}\NormalTok{obs }\OperatorTok\StringTok{ }\KeywordTok{c}\NormalTok{(}\StringTok{"59"}\NormalTok{, }\StringTok{"62"}\NormalTok{, }\StringTok{"63"}\NormalTok{, }\StringTok{"65"}\NormalTok{, }\StringTok{"68"}\NormalTok{, }\StringTok{"70"}\NormalTok{, }\StringTok{"71"}\NormalTok{, }\StringTok{"72"}\NormalTok{, }\StringTok{"74"}\NormalTok{))}
    
\NormalTok{    pers_new <-}\StringTok{  }\NormalTok{pers_new }\OperatorTok\StringTok{ }\KeywordTok{filter}\NormalTok{(}\OperatorTok{!}\NormalTok{study_ID }\OperatorTok{==}\StringTok{ "P060"} \OperatorTok{|}\StringTok{ }\OperatorTok{!}\NormalTok{obs }\OperatorTok\StringTok{ }\KeywordTok{c}\NormalTok{(}\StringTok{"216"}\NormalTok{, }\StringTok{"217"}\NormalTok{))}
    
\NormalTok{    pers_new <-}\StringTok{ }\NormalTok{pers_new }\OperatorTok\StringTok{ }\KeywordTok{filter}\NormalTok{(}\OperatorTok{!}\NormalTok{study_ID }\OperatorTok{==}\StringTok{ "P211"} \OperatorTok{|}\StringTok{ }\OperatorTok{!}\NormalTok{obs }\OperatorTok\StringTok{ }\KeywordTok{c}\NormalTok{(}\StringTok{"230"}\NormalTok{, }\StringTok{"245"}\NormalTok{))}
    
\NormalTok{    pers_new <-}\StringTok{ }\NormalTok{pers_new }\OperatorTok\StringTok{ }\KeywordTok{filter}\NormalTok{(}\OperatorTok{!}\NormalTok{study_ID }\OperatorTok{==}\StringTok{ "P117"} \OperatorTok{|}\StringTok{ }\OperatorTok{!}\NormalTok{obs }\OperatorTok\StringTok{ }\KeywordTok{c}\NormalTok{(}\StringTok{"397"}\NormalTok{, }\StringTok{"393"}\NormalTok{, }\StringTok{"402"}\NormalTok{, }\StringTok{"414"}\NormalTok{))}
    
\NormalTok{    pers_new <-}\StringTok{ }\NormalTok{pers_new }\OperatorTok\StringTok{ }\KeywordTok{filter}\NormalTok{(}\OperatorTok{!}\NormalTok{study_ID }\OperatorTok{==}\StringTok{ "P197"} \OperatorTok{|}\StringTok{ }\OperatorTok{!}\NormalTok{obs }\OperatorTok\StringTok{ }\KeywordTok{c}\NormalTok{(}\StringTok{"541"}\NormalTok{, }\StringTok{"544"}\NormalTok{, }\StringTok{"546"}\NormalTok{, }\StringTok{"547"}\NormalTok{))}
    
\NormalTok{    pers_new <-}\StringTok{ }\NormalTok{pers_new }\OperatorTok\StringTok{ }\KeywordTok{filter}\NormalTok{(}\OperatorTok{!}\NormalTok{study_ID }\OperatorTok{==}\StringTok{ "P197"} \OperatorTok{|}\StringTok{ }\OperatorTok{!}\NormalTok{obs }\OperatorTok\StringTok{ }\KeywordTok{c}\NormalTok{(}\StringTok{"669"}\NormalTok{, }\StringTok{"672"}\NormalTok{, }\StringTok{"673"}\NormalTok{, }\StringTok{"682"}\NormalTok{, }\StringTok{"683"}\NormalTok{, }\StringTok{"684"}\NormalTok{, }\StringTok{"686"}\NormalTok{, }\StringTok{"694"}\NormalTok{)) }
\NormalTok{    pers_new <-}\StringTok{ }\NormalTok{pers_new }\OperatorTok\StringTok{ }\KeywordTok{filter}\NormalTok{(}\OperatorTok{!}\NormalTok{study_ID }\OperatorTok{==}\StringTok{ "P041"} \OperatorTok{|}\StringTok{ }\OperatorTok{!}\NormalTok{obs }\OperatorTok\StringTok{ }\KeywordTok{c}\NormalTok{(}\StringTok{"120"}\NormalTok{, }\StringTok{"124"}\NormalTok{))}
    
    \KeywordTok{dim}\NormalTok{(pers_new) }\CommentTok{# check they've been removed without issue}
    
    \CommentTok{# after checking the data, there are a few effect sizes that might be driving weird results so let's drop them and see}
    
\NormalTok{    pers_new <-}\StringTok{ }\NormalTok{pers_new }\OperatorTok\StringTok{ }\KeywordTok{filter}\NormalTok{(}\OperatorTok{!}\NormalTok{study_ID }\OperatorTok{==}\StringTok{ "P100"} \OperatorTok{|}\StringTok{ }\OperatorTok{!}\NormalTok{obs }\OperatorTok{==}\StringTok{ "519"}\NormalTok{) }\CommentTok{# big outlier}
\end{Highlighting}
\end{Shaded}

\subsection{Flip signs of effects for
SMD}\label{flip-signs-of-effects-for-smd}

The directional meaning of effect sizes vary depending on the specific
units and trait being measured. The data has a directionality column
that tells one if the meaning should be reversed (1) or left the same.

\begin{Shaded}
\begin{Highlighting}[]
\NormalTok{   pers_new}\OperatorTok{$}\NormalTok{directionality <-}\StringTok{ }\KeywordTok{ifelse}\NormalTok{(}\KeywordTok{is.na}\NormalTok{(pers_new}\OperatorTok{$}\NormalTok{directionality), }\DecValTok{0}\NormalTok{, }\DecValTok{1}\NormalTok{)}

\NormalTok{      pers_new}\OperatorTok{$}\NormalTok{SMD_yi_flip <-}\StringTok{ }\KeywordTok{ifelse}\NormalTok{(pers_new}\OperatorTok{$}\NormalTok{directionality }\OperatorTok{==}\StringTok{ }\DecValTok{1}\NormalTok{, pers_new}\OperatorTok{$}\NormalTok{SMD_yi}\OperatorTok{*}\NormalTok{(}\OperatorTok{-}\DecValTok{1}\NormalTok{), pers_new}\OperatorTok{$}\NormalTok{SMD_yi)}
\end{Highlighting}
\end{Shaded}

\begin{center}\rule{0.5\linewidth}{0.5pt}\end{center}

\section{Prepare the phylogenetic
trees}\label{prepare-the-phylogenetic-trees}

We constructed seperate phylogenetic trees for each taxonomic group. The
tree for birds was constructed using BirdTree.org, the rest were
constructed using TimeTree.org. We'll use these trees for multi-level
meta-analytic models throughout the analysis.

\begin{Shaded}
\begin{Highlighting}[]
\CommentTok{# Find all tree file names}
\NormalTok{  tree_files <-}\StringTok{ }\KeywordTok{paste0}\NormalTok{(}\StringTok{"./trees/"}\NormalTok{, }\KeywordTok{list.files}\NormalTok{(}\StringTok{"./trees"}\NormalTok{))[}\OperatorTok{-}\DecValTok{1}\NormalTok{]}
  
    \CommentTok{# Bird tree has been constructed already, just need to get trees for the rest of the taxo groups   }
\NormalTok{      trees <-}\StringTok{ }\KeywordTok{lapply}\NormalTok{(tree_files, }\ControlFlowTok{function}\NormalTok{(x) }\KeywordTok{read.tree}\NormalTok{(x))}
\NormalTok{      names <-}\StringTok{ }\KeywordTok{gsub}\NormalTok{(}\StringTok{"~./trees/"}\NormalTok{, }\StringTok{""}\NormalTok{, tree_files)}
      \KeywordTok{names}\NormalTok{(trees) <-}\StringTok{ }\NormalTok{names}

    \CommentTok{# Plot the trees and see how they look}
      \KeywordTok{par}\NormalTok{(}\DataTypeTok{mfrow =} \KeywordTok{c}\NormalTok{(}\DecValTok{1}\NormalTok{,}\DecValTok{5}\NormalTok{), }\DataTypeTok{mar =} \KeywordTok{c}\NormalTok{(}\DecValTok{1}\NormalTok{,}\DecValTok{1}\NormalTok{,}\DecValTok{1}\NormalTok{,}\DecValTok{1}\NormalTok{))}
      \KeywordTok{lapply}\NormalTok{(trees, }\ControlFlowTok{function}\NormalTok{(x) }\KeywordTok{plot}\NormalTok{(x, }\DataTypeTok{cex =} \DecValTok{1}\NormalTok{))}
\end{Highlighting}
\end{Shaded}

\includegraphics{meta-analysis-and-models_clean_files/figure-latex/phylo-1.pdf}

\begin{verbatim}
## $`./trees/bird_species.nwk`
## $`./trees/bird_species.nwk`$type
## [1] "phylogram"
## 
## $`./trees/bird_species.nwk`$use.edge.length
## [1] TRUE
## 
## $`./trees/bird_species.nwk`$node.pos
## [1] 1
## 
## $`./trees/bird_species.nwk`$node.depth
## [1] 1
## 
## $`./trees/bird_species.nwk`$show.tip.label
## [1] TRUE
## 
## $`./trees/bird_species.nwk`$show.node.label
## [1] FALSE
## 
## $`./trees/bird_species.nwk`$font
## [1] 3
## 
## $`./trees/bird_species.nwk`$cex
## [1] 1
## 
## $`./trees/bird_species.nwk`$adj
## [1] 0
## 
## $`./trees/bird_species.nwk`$srt
## [1] 0
## 
## $`./trees/bird_species.nwk`$no.margin
## [1] FALSE
## 
## $`./trees/bird_species.nwk`$label.offset
## [1] 0
## 
## $`./trees/bird_species.nwk`$x.lim
## [1]    0.000 1815.121
## 
## $`./trees/bird_species.nwk`$y.lim
## [1]   1 109
## 
## $`./trees/bird_species.nwk`$direction
## [1] "rightwards"
## 
## $`./trees/bird_species.nwk`$tip.color
## [1] "black"
## 
## $`./trees/bird_species.nwk`$Ntip
## [1] 109
## 
## $`./trees/bird_species.nwk`$Nnode
## [1] 108
## 
## $`./trees/bird_species.nwk`$root.time
## NULL
## 
## $`./trees/bird_species.nwk`$align.tip.label
## [1] FALSE
## 
## 
## $`./trees/fish_species.nwk`
## $`./trees/fish_species.nwk`$type
## [1] "phylogram"
## 
## $`./trees/fish_species.nwk`$use.edge.length
## [1] TRUE
## 
## $`./trees/fish_species.nwk`$node.pos
## [1] 1
## 
## $`./trees/fish_species.nwk`$node.depth
## [1] 1
## 
## $`./trees/fish_species.nwk`$show.tip.label
## [1] TRUE
## 
## $`./trees/fish_species.nwk`$show.node.label
## [1] FALSE
## 
## $`./trees/fish_species.nwk`$font
## [1] 3
## 
## $`./trees/fish_species.nwk`$cex
## [1] 1
## 
## $`./trees/fish_species.nwk`$adj
## [1] 0
## 
## $`./trees/fish_species.nwk`$srt
## [1] 0
## 
## $`./trees/fish_species.nwk`$no.margin
## [1] FALSE
## 
## $`./trees/fish_species.nwk`$label.offset
## [1] 0
## 
## $`./trees/fish_species.nwk`$x.lim
## [1]     0.00 43420.45
## 
## $`./trees/fish_species.nwk`$y.lim
## [1]  1 22
## 
## $`./trees/fish_species.nwk`$direction
## [1] "rightwards"
## 
## $`./trees/fish_species.nwk`$tip.color
## [1] "black"
## 
## $`./trees/fish_species.nwk`$Ntip
## [1] 22
## 
## $`./trees/fish_species.nwk`$Nnode
## [1] 21
## 
## $`./trees/fish_species.nwk`$root.time
## NULL
## 
## $`./trees/fish_species.nwk`$align.tip.label
## [1] FALSE
## 
## 
## $`./trees/invert_species.nwk`
## $`./trees/invert_species.nwk`$type
## [1] "phylogram"
## 
## $`./trees/invert_species.nwk`$use.edge.length
## [1] TRUE
## 
## $`./trees/invert_species.nwk`$node.pos
## [1] 1
## 
## $`./trees/invert_species.nwk`$node.depth
## [1] 1
## 
## $`./trees/invert_species.nwk`$show.tip.label
## [1] TRUE
## 
## $`./trees/invert_species.nwk`$show.node.label
## [1] FALSE
## 
## $`./trees/invert_species.nwk`$font
## [1] 3
## 
## $`./trees/invert_species.nwk`$cex
## [1] 1
## 
## $`./trees/invert_species.nwk`$adj
## [1] 0
## 
## $`./trees/invert_species.nwk`$srt
## [1] 0
## 
## $`./trees/invert_species.nwk`$no.margin
## [1] FALSE
## 
## $`./trees/invert_species.nwk`$label.offset
## [1] 0
## 
## $`./trees/invert_species.nwk`$x.lim
## [1]    0.000 8835.685
## 
## $`./trees/invert_species.nwk`$y.lim
## [1]  1 42
## 
## $`./trees/invert_species.nwk`$direction
## [1] "rightwards"
## 
## $`./trees/invert_species.nwk`$tip.color
## [1] "black"
## 
## $`./trees/invert_species.nwk`$Ntip
## [1] 42
## 
## $`./trees/invert_species.nwk`$Nnode
## [1] 41
## 
## $`./trees/invert_species.nwk`$root.time
## NULL
## 
## $`./trees/invert_species.nwk`$align.tip.label
## [1] FALSE
## 
## 
## $`./trees/mammal_species.nwk`
## $`./trees/mammal_species.nwk`$type
## [1] "phylogram"
## 
## $`./trees/mammal_species.nwk`$use.edge.length
## [1] TRUE
## 
## $`./trees/mammal_species.nwk`$node.pos
## [1] 1
## 
## $`./trees/mammal_species.nwk`$node.depth
## [1] 1
## 
## $`./trees/mammal_species.nwk`$show.tip.label
## [1] TRUE
## 
## $`./trees/mammal_species.nwk`$show.node.label
## [1] FALSE
## 
## $`./trees/mammal_species.nwk`$font
## [1] 3
## 
## $`./trees/mammal_species.nwk`$cex
## [1] 1
## 
## $`./trees/mammal_species.nwk`$adj
## [1] 0
## 
## $`./trees/mammal_species.nwk`$srt
## [1] 0
## 
## $`./trees/mammal_species.nwk`$no.margin
## [1] FALSE
## 
## $`./trees/mammal_species.nwk`$label.offset
## [1] 0
## 
## $`./trees/mammal_species.nwk`$x.lim
## [1]    0.000 1785.101
## 
## $`./trees/mammal_species.nwk`$y.lim
## [1]  1 46
## 
## $`./trees/mammal_species.nwk`$direction
## [1] "rightwards"
## 
## $`./trees/mammal_species.nwk`$tip.color
## [1] "black"
## 
## $`./trees/mammal_species.nwk`$Ntip
## [1] 46
## 
## $`./trees/mammal_species.nwk`$Nnode
## [1] 45
## 
## $`./trees/mammal_species.nwk`$root.time
## NULL
## 
## $`./trees/mammal_species.nwk`$align.tip.label
## [1] FALSE
## 
## 
## $`./trees/reptile_species.nwk`
## $`./trees/reptile_species.nwk`$type
## [1] "phylogram"
## 
## $`./trees/reptile_species.nwk`$use.edge.length
## [1] TRUE
## 
## $`./trees/reptile_species.nwk`$node.pos
## [1] 1
## 
## $`./trees/reptile_species.nwk`$node.depth
## [1] 1
## 
## $`./trees/reptile_species.nwk`$show.tip.label
## [1] TRUE
## 
## $`./trees/reptile_species.nwk`$show.node.label
## [1] FALSE
## 
## $`./trees/reptile_species.nwk`$font
## [1] 3
## 
## $`./trees/reptile_species.nwk`$cex
## [1] 1
## 
## $`./trees/reptile_species.nwk`$adj
## [1] 0
## 
## $`./trees/reptile_species.nwk`$srt
## [1] 0
## 
## $`./trees/reptile_species.nwk`$no.margin
## [1] FALSE
## 
## $`./trees/reptile_species.nwk`$label.offset
## [1] 0
## 
## $`./trees/reptile_species.nwk`$x.lim
## [1]    0.000 2764.791
## 
## $`./trees/reptile_species.nwk`$y.lim
## [1]  1 10
## 
## $`./trees/reptile_species.nwk`$direction
## [1] "rightwards"
## 
## $`./trees/reptile_species.nwk`$tip.color
## [1] "black"
## 
## $`./trees/reptile_species.nwk`$Ntip
## [1] 10
## 
## $`./trees/reptile_species.nwk`$Nnode
## [1] 9
## 
## $`./trees/reptile_species.nwk`$root.time
## NULL
## 
## $`./trees/reptile_species.nwk`$align.tip.label
## [1] FALSE
\end{verbatim}

\begin{Shaded}
\begin{Highlighting}[]
\CommentTok{# Checking trees to ensure we only include species in the current dataset}

    \CommentTok{# Check that they are ultrametric}
      \KeywordTok{lapply}\NormalTok{(trees, }\ControlFlowTok{function}\NormalTok{(x) }\KeywordTok{is.ultrametric}\NormalTok{(x))}

    \CommentTok{# Check that all names in the phylogeny are also in the data}
\NormalTok{      taxa_data_list <-}\StringTok{ }\KeywordTok{split}\NormalTok{(pers_new, pers_new}\OperatorTok{$}\NormalTok{taxo_group)}
      
\NormalTok{      other_groups <-}\StringTok{ }\KeywordTok{mapply}\NormalTok{(}\DataTypeTok{x =}\NormalTok{ taxa_data_list, }
                             \DataTypeTok{y =}\NormalTok{ trees, }
                             \ControlFlowTok{function}\NormalTok{(x,y) }\KeywordTok{tree_checks}\NormalTok{(x,y, }\StringTok{"spp_name_phylo"}\NormalTok{, }\DataTypeTok{type =} \StringTok{"checks"}\NormalTok{)) }

    \CommentTok{# Print out each taxon group}
      \ControlFlowTok{for}\NormalTok{(i }\ControlFlowTok{in} \KeywordTok{colnames}\NormalTok{(other_groups))\{}
            \KeywordTok{print}\NormalTok{(i)}
            \KeywordTok{print}\NormalTok{(other_groups[,i] )}
\NormalTok{      \}}
      

      \CommentTok{# Now to prune trees so that we get tree names that match with species in data}
\NormalTok{      pruned_trees <-}\StringTok{ }\KeywordTok{mapply}\NormalTok{(}\DataTypeTok{x =}\NormalTok{ taxa_data_list, }
                             \DataTypeTok{y =}\NormalTok{ trees, }
                             \ControlFlowTok{function}\NormalTok{(x,y) }\KeywordTok{tree_checks}\NormalTok{(x,y, }\StringTok{"spp_name_phylo"}\NormalTok{, }\DataTypeTok{type =} \StringTok{"prune"}\NormalTok{))}

      \CommentTok{# Check that this has been done correctly}
\NormalTok{        re_checks <-}\StringTok{ }\KeywordTok{mapply}\NormalTok{(}\DataTypeTok{x =}\NormalTok{ taxa_data_list, }
                             \DataTypeTok{y =}\NormalTok{ pruned_trees, }
                             \ControlFlowTok{function}\NormalTok{(x,y) }\KeywordTok{tree_checks}\NormalTok{(x,y, }\StringTok{"spp_name_phylo"}\NormalTok{, }\DataTypeTok{type =} \StringTok{"checks"}\NormalTok{))}
        
      \ControlFlowTok{for}\NormalTok{(i }\ControlFlowTok{in} \KeywordTok{colnames}\NormalTok{(re_checks))\{}
            \KeywordTok{print}\NormalTok{(i)}
            \KeywordTok{print}\NormalTok{(re_checks[,i] )}
\NormalTok{      \}}

\CommentTok{# Extract the phylogenetic correlation matrices}
\NormalTok{      phylo_vcv <-}\StringTok{ }\KeywordTok{lapply}\NormalTok{(pruned_trees, }\ControlFlowTok{function}\NormalTok{(x) }\KeywordTok{vcv}\NormalTok{(x, }\DataTypeTok{corr =} \OtherTok{TRUE}\NormalTok{)) }\CommentTok{# these matrices are used in the meta-a models}
\end{Highlighting}
\end{Shaded}

\section{Sensitivity analysis - checking score
data}\label{sensitivity-analysis---checking-score-data}

Before we begin, we need to run a sensitivity analysis to see if score
data is ok to use. With these models, we are just including score as a
moderator term to compare with the rest of the dataset (some of which
has already been transformed, we just can't do that with scores). Model
summaries are also presented in Supplementary Table S2.

Our score sensitivity model:

\begin{Shaded}
\begin{Highlighting}[]
    \CommentTok{# model:}
\NormalTok{     sensitivity_mod1_score <-}\StringTok{ }\KeywordTok{meta_model_fits}\NormalTok{(pers_new, phylo_vcv, }\DataTypeTok{type =} \StringTok{"score"}\NormalTok{)}

    \CommentTok{# Extract the SMD and lnCVR results}
\NormalTok{     smd_mods_score <-}\StringTok{ }\NormalTok{sensitivity_mod1_score[}\StringTok{"SMD"}\NormalTok{,]  }
\NormalTok{     lnCVR_mods_score <-}\StringTok{ }\NormalTok{sensitivity_mod1_score[}\StringTok{"lnCVR"}\NormalTok{,] }\CommentTok{# inverts significant}
    
    \CommentTok{# Because invert score data is significantly different, we need to remove these effect sizes before running our models}
        
      \CommentTok{# filter out invert scores from pers dataset  }
\NormalTok{            pers_new <-}\StringTok{ }\NormalTok{pers_new }\OperatorTok
\StringTok{                              }\KeywordTok{filter}\NormalTok{(score }\OperatorTok{!=}\StringTok{ "score"} \OperatorTok{|}\StringTok{ }\NormalTok{taxo_group }\OperatorTok{!=}\StringTok{ "invertebrate"}\NormalTok{) }
    
            \KeywordTok{dim}\NormalTok{(pers_new) }
            
   \CommentTok{# How many effect sizes, unique studies and different species are we left with?}
            \KeywordTok{data.frame}\NormalTok{(pers_new }\OperatorTok\StringTok{ }
\StringTok{                         }\KeywordTok{summarise}\NormalTok{(}\DataTypeTok{n =} \KeywordTok{n}\NormalTok{(), }\DataTypeTok{studies =} \KeywordTok{length}\NormalTok{(}\KeywordTok{unique}\NormalTok{(study_ID)), }\DataTypeTok{species =} \KeywordTok{length}\NormalTok{(}\KeywordTok{unique}\NormalTok{(spp_name_phylo))))}
\end{Highlighting}
\end{Shaded}

\begin{center}\rule{0.5\linewidth}{0.5pt}\end{center}

\section{Meta-Analysis Models}\label{meta-analysis-models}

Let's run the first bunch of models on the whole dataset. We'll start
off with intercept-only multi-level meta-analytic models, then move to
multi-level meta-regression models (personality traits, and SSD). The
functions in \texttt{func.R} should be consulted to see precisely what
models are being fit across the taxonomic groups.

\subsection{Intercept-only MLMA
models}\label{intercept-only-mlma-models}

Complete model summaries are also presented in Supplementary Table S14.

\begin{Shaded}
\begin{Highlighting}[]
\CommentTok{# First we will fit our MLMA intercept only models, across each taxo group. }

  \CommentTok{# we can use this function to just read the saved model output instead of re-running the model, which takes a while}
\NormalTok{rerun_models }\OperatorTok{==}\StringTok{ }\OtherTok{FALSE}
\end{Highlighting}
\end{Shaded}

\begin{verbatim}
## [1] TRUE
\end{verbatim}

\begin{Shaded}
\begin{Highlighting}[]
  \ControlFlowTok{if}\NormalTok{(rerun_models }\OperatorTok{==}\StringTok{ }\OtherTok{TRUE}\NormalTok{)\{}
\NormalTok{    MLMA_models <-}\StringTok{ }\KeywordTok{meta_model_fits}\NormalTok{(pers_new, phylo_vcv, }\DataTypeTok{type =} \StringTok{"int"}\NormalTok{)}
    \KeywordTok{saveRDS}\NormalTok{(MLMA_models, }\StringTok{"./output/MLMA_models_int"}\NormalTok{)}
\NormalTok{  \}}\ControlFlowTok{else}\NormalTok{\{}
\NormalTok{    MLMA_models <-}\StringTok{ }\KeywordTok{readRDS}\NormalTok{(}\StringTok{"./output/MLMA_models_int"}\NormalTok{)}
\NormalTok{  \}}


  \CommentTok{# View model results}
\NormalTok{  split_taxa <-}\StringTok{ }\KeywordTok{split}\NormalTok{(pers_new, pers_new}\OperatorTok{$}\NormalTok{taxo_group)}
  
\NormalTok{    smd_mods <-}\StringTok{ }\NormalTok{MLMA_models[}\StringTok{"SMD"}\NormalTok{,]}
  
\NormalTok{    lnCVR_mods <-}\StringTok{ }\NormalTok{MLMA_models[}\StringTok{"lnCVR"}\NormalTok{,]}
\end{Highlighting}
\end{Shaded}

\subsection{I2 estimates of heterogeneity - intercept
models}\label{i2-estimates-of-heterogeneity---intercept-models}

Study\_ID is the between study heterogeneity, Phylo tells us if there is
a phylogenetic signal and the strength of that signal. Total I2 is
testing how much heterogeneity we have beyond sampling variance

\begin{Shaded}
\begin{Highlighting}[]
\CommentTok{# From these models we can get I2 estimates:}
\NormalTok{  birds_smd =}\StringTok{ }\KeywordTok{I2}\NormalTok{(smd_mods[[}\DecValTok{1}\NormalTok{]], }\DataTypeTok{v =}\NormalTok{ split_taxa[[}\DecValTok{1}\NormalTok{]]}\OperatorTok{$}\NormalTok{SMD_vi, }\DataTypeTok{phylo =} \StringTok{"spp_name_phylo"}\NormalTok{, }\DataTypeTok{obs =} \StringTok{"obs"}\NormalTok{)}
     
\NormalTok{  birds_CVR =}\StringTok{ }\KeywordTok{I2}\NormalTok{(lnCVR_mods[[}\DecValTok{1}\NormalTok{]], }\DataTypeTok{v =}\NormalTok{ split_taxa[[}\DecValTok{1}\NormalTok{]]}\OperatorTok{$}\NormalTok{CVR_vi, }\DataTypeTok{phylo =} \StringTok{"spp_name_phylo"}\NormalTok{, }\DataTypeTok{obs =} \StringTok{"obs"}\NormalTok{)}
      
\NormalTok{  fish_smd =}\StringTok{ }\KeywordTok{I2}\NormalTok{(smd_mods[[}\DecValTok{2}\NormalTok{]], }\DataTypeTok{v =}\NormalTok{ split_taxa[[}\DecValTok{2}\NormalTok{]]}\OperatorTok{$}\NormalTok{SMD_vi, }\DataTypeTok{phylo =} \StringTok{"spp_name_phylo"}\NormalTok{, }\DataTypeTok{obs =} \StringTok{"obs"}\NormalTok{)}
      
\NormalTok{  fish_CVR =}\StringTok{ }\KeywordTok{I2}\NormalTok{(lnCVR_mods[[}\DecValTok{2}\NormalTok{]], }\DataTypeTok{v =}\NormalTok{ split_taxa[[}\DecValTok{2}\NormalTok{]]}\OperatorTok{$}\NormalTok{CVR_vi, }\DataTypeTok{phylo =} \StringTok{"spp_name_phylo"}\NormalTok{, }\DataTypeTok{obs =} \StringTok{"obs"}\NormalTok{)}
      
\NormalTok{  invert_smd =}\StringTok{ }\KeywordTok{I2}\NormalTok{(smd_mods[[}\DecValTok{3}\NormalTok{]], }\DataTypeTok{v =}\NormalTok{ split_taxa[[}\DecValTok{3}\NormalTok{]]}\OperatorTok{$}\NormalTok{SMD_vi, }\DataTypeTok{phylo =} \StringTok{"spp_name_phylo"}\NormalTok{, }\DataTypeTok{obs =} \StringTok{"obs"}\NormalTok{)}
      
\NormalTok{  invert_CVR =}\StringTok{ }\KeywordTok{I2}\NormalTok{(lnCVR_mods[[}\DecValTok{3}\NormalTok{]], }\DataTypeTok{v =}\NormalTok{ split_taxa[[}\DecValTok{3}\NormalTok{]]}\OperatorTok{$}\NormalTok{CVR_vi, }\DataTypeTok{phylo =} \StringTok{"spp_name_phylo"}\NormalTok{, }\DataTypeTok{obs =} \StringTok{"obs"}\NormalTok{)}
      
\NormalTok{  mammal_smd =}\StringTok{ }\KeywordTok{I2}\NormalTok{(smd_mods[[}\DecValTok{4}\NormalTok{]], }\DataTypeTok{v =}\NormalTok{ split_taxa[[}\DecValTok{4}\NormalTok{]]}\OperatorTok{$}\NormalTok{SMD_vi, }\DataTypeTok{phylo =} \StringTok{"spp_name_phylo"}\NormalTok{, }\DataTypeTok{obs =} \StringTok{"obs"}\NormalTok{)}
      
\NormalTok{  mammal_CVR =}\StringTok{ }\KeywordTok{I2}\NormalTok{(lnCVR_mods[[}\DecValTok{4}\NormalTok{]], }\DataTypeTok{v =}\NormalTok{ split_taxa[[}\DecValTok{4}\NormalTok{]]}\OperatorTok{$}\NormalTok{CVR_vi, }\DataTypeTok{phylo =} \StringTok{"spp_name_phylo"}\NormalTok{, }\DataTypeTok{obs =} \StringTok{"obs"}\NormalTok{)}
      
\NormalTok{  reptile_smd =}\StringTok{ }\KeywordTok{I2}\NormalTok{(smd_mods[[}\DecValTok{5}\NormalTok{]], }\DataTypeTok{v =}\NormalTok{ split_taxa[[}\DecValTok{5}\NormalTok{]]}\OperatorTok{$}\NormalTok{SMD_vi, }\DataTypeTok{phylo =} \StringTok{"spp_name_phylo"}\NormalTok{, }\DataTypeTok{obs =} \StringTok{"obs"}\NormalTok{)}
      
\NormalTok{  reptile_CVR =}\StringTok{ }\KeywordTok{I2}\NormalTok{(lnCVR_mods[[}\DecValTok{5}\NormalTok{]], }\DataTypeTok{v =}\NormalTok{ split_taxa[[}\DecValTok{5}\NormalTok{]]}\OperatorTok{$}\NormalTok{CVR_vi, }\DataTypeTok{phylo =} \StringTok{"spp_name_phylo"}\NormalTok{, }\DataTypeTok{obs =} \StringTok{"obs"}\NormalTok{)}
      
  
\CommentTok{# Now that we have our list of models, we can extract the estimates, CIs and prediction intervals}
\NormalTok{  MLMA_estimates_SMD <-}\StringTok{ }\NormalTok{plyr}\OperatorTok{::}\KeywordTok{ldply}\NormalTok{(}\KeywordTok{lapply}\NormalTok{(smd_mods, }\ControlFlowTok{function}\NormalTok{(x) }\KeywordTok{print}\NormalTok{(}\KeywordTok{mod_results}\NormalTok{(x, }\DataTypeTok{mod =} \StringTok{"Int"}\NormalTok{))))}
\NormalTok{      MLMA_estimates_SMD}
\end{Highlighting}
\end{Shaded}

\begin{verbatim}
##            .id    name    estimate    lowerCL   upperCL    lowerPR   upperPR
## 1         bird Intrcpt -0.11192796 -0.3499999 0.1261440 -1.8619378 1.6380819
## 2         fish Intrcpt  0.17680902 -0.2270172 0.5806353 -1.5472144 1.9008324
## 3 invertebrate Intrcpt  0.25283343 -0.1153844 0.6210513 -2.1177023 2.6233692
## 4       mammal Intrcpt  0.06839391 -0.2982885 0.4350763 -1.1341419 1.2709297
## 5     reptilia Intrcpt  0.06381860 -0.1145932 0.2422304 -0.5626372 0.6902744
\end{verbatim}

\begin{Shaded}
\begin{Highlighting}[]
\NormalTok{  MLMA_estimates_lnCVR <-}\StringTok{ }\NormalTok{plyr}\OperatorTok{::}\KeywordTok{ldply}\NormalTok{(}\KeywordTok{lapply}\NormalTok{(lnCVR_mods, }\ControlFlowTok{function}\NormalTok{(x) }\KeywordTok{print}\NormalTok{(}\KeywordTok{mod_results}\NormalTok{(x, }\DataTypeTok{mod =} \StringTok{"Int"}\NormalTok{))))}
\NormalTok{      MLMA_estimates_lnCVR}
\end{Highlighting}
\end{Shaded}

\begin{verbatim}
##            .id    name     estimate     lowerCL    upperCL     lowerPR
## 1         bird Intrcpt -0.138856029 -0.63322342 0.35551136 -1.88286875
## 2         fish Intrcpt -0.004621130 -0.08649739 0.07725513 -0.74702714
## 3 invertebrate Intrcpt -0.008634225 -0.10911299 0.09184454 -0.73732713
## 4       mammal Intrcpt  0.055549381 -0.20674669 0.31784545 -0.65499594
## 5     reptilia Intrcpt  0.037232224 -0.04236748 0.11683193 -0.04136392
##     upperPR
## 1 1.6051567
## 2 0.7377849
## 3 0.7200587
## 4 0.7660947
## 5 0.1158284
\end{verbatim}

Extract p-values from these models to use later when adjusting them for
multiple testing:

\begin{Shaded}
\begin{Highlighting}[]
\CommentTok{# taking p-values from models for False Discovery Rate p-value adjustment}
\NormalTok{  p.SMD_intercept <-}\StringTok{ }\KeywordTok{unlist}\NormalTok{(}\KeywordTok{lapply}\NormalTok{(smd_mods, }\ControlFlowTok{function}\NormalTok{(x) x}\OperatorTok{$}\NormalTok{pval))}
\NormalTok{      p.SMD_intercept}
\end{Highlighting}
\end{Shaded}

\begin{verbatim}
##         bird         fish invertebrate       mammal     reptilia 
##    0.3560563    0.3900624    0.1778479    0.7143065    0.4793203
\end{verbatim}

\begin{Shaded}
\begin{Highlighting}[]
\NormalTok{  p.lnCVR_intercept <-}\StringTok{ }\KeywordTok{unlist}\NormalTok{(}\KeywordTok{lapply}\NormalTok{(lnCVR_mods, }\ControlFlowTok{function}\NormalTok{(x) x}\OperatorTok{$}\NormalTok{pval))}
\NormalTok{      p.lnCVR_intercept}
\end{Highlighting}
\end{Shaded}

\begin{verbatim}
##         bird         fish invertebrate       mammal     reptilia 
##    0.5812731    0.9117446    0.8659517    0.6776659    0.3554154
\end{verbatim}

\subsection{Personality trait MLMR
models}\label{personality-trait-mlmr-models}

These models include personality trait type as a moderator. Please note
that we estimate the mean for each of the categorical levels because we
are not really interested in whether the means differ, but whether or
not males and females differ in any of these traits.

Complete model summaries are presented in Supplementary Table S15.

\begin{Shaded}
\begin{Highlighting}[]
\CommentTok{# we can just reload saved model outputs here to save time}
\NormalTok{rerun_models }\OperatorTok{==}\StringTok{ }\OtherTok{FALSE}
\end{Highlighting}
\end{Shaded}

\begin{verbatim}
## [1] TRUE
\end{verbatim}

\begin{Shaded}
\begin{Highlighting}[]
    \ControlFlowTok{if}\NormalTok{(rerun_models }\OperatorTok{==}\StringTok{ }\OtherTok{TRUE}\NormalTok{)\{}
\NormalTok{      MLMR_models_pers_trait <-}\StringTok{ }\KeywordTok{meta_model_fits}\NormalTok{(pers_new, phylo_vcv, }\DataTypeTok{type =} \StringTok{"pers"}\NormalTok{)}
      \KeywordTok{saveRDS}\NormalTok{(MLMR_models_pers_trait, }\StringTok{"./output/MLMR_models_pers_trait"}\NormalTok{)}
\NormalTok{    \} }\ControlFlowTok{else}\NormalTok{\{}
\NormalTok{      MLMR_models_pers_trait <-}\StringTok{ }\KeywordTok{readRDS}\NormalTok{(}\StringTok{"./output/MLMR_models_pers_trait"}\NormalTok{)}
\NormalTok{    \}}

  \CommentTok{# Extract the SMD and lnCVR results}
\NormalTok{  smd_mods_pers <-}\StringTok{ }\NormalTok{MLMR_models_pers_trait[}\StringTok{"SMD"}\NormalTok{,]}
     
\NormalTok{  lnCVR_mods_pers <-}\StringTok{ }\NormalTok{MLMR_models_pers_trait[}\StringTok{"lnCVR"}\NormalTok{,] }
     
  \CommentTok{# these model objects are used to make the orchard plots shown in Figures 2-6}
\end{Highlighting}
\end{Shaded}

Get prediction intervals for personality trait models:

\begin{Shaded}
\begin{Highlighting}[]
  \CommentTok{# Get the combined estimates from them all}
\NormalTok{  MLMA_estimates_SMD_pers <-}\StringTok{ }\NormalTok{plyr}\OperatorTok{::}\KeywordTok{ldply}\NormalTok{(}\KeywordTok{lapply}\NormalTok{(smd_mods_pers, }\ControlFlowTok{function}\NormalTok{(x) }\KeywordTok{print}\NormalTok{(}\KeywordTok{mod_results}\NormalTok{(x, }\DataTypeTok{mod =} \StringTok{"personality_trait"}\NormalTok{))))}
\NormalTok{      MLMA_estimates_SMD_pers}
\end{Highlighting}
\end{Shaded}

\begin{verbatim}
##             .id        name     estimate    lowerCL    upperCL    lowerPR
## 1          bird    Activity -0.082630938 -0.3731716  0.2079097 -1.8686217
## 2          bird  Aggression -0.150395862 -0.4380609  0.1372692 -1.9359235
## 3          bird    Boldness -0.141981559 -0.3986038  0.1146406 -1.9227956
## 4          bird Exploration  0.061120004 -0.2107229  0.3329629 -1.7219395
## 5          bird   Sociality -0.589437560 -1.0564775 -0.1223977 -2.4122911
## 6          fish    Activity  0.153275118 -0.3114801  0.6180303 -1.6341205
## 7          fish  Aggression  0.252603144 -0.1984714  0.7036777 -1.5313017
## 8          fish    Boldness  0.083577441 -0.3561599  0.5233148 -1.6975086
## 9          fish Exploration  0.265681594 -0.1976033  0.7289665 -1.5213342
## 10         fish   Sociality  0.149525877 -0.3372741  0.6363258 -1.6436989
## 11 invertebrate    Activity  0.250548279 -0.1462921  0.6473887 -2.1358328
## 12 invertebrate  Aggression  0.420366327 -0.2762252  1.1169578 -2.0333404
## 13 invertebrate    Boldness  0.242813292 -0.1483189  0.6339455 -2.1426306
## 14 invertebrate Exploration  0.039752886 -0.4098280  0.4893337 -2.3559092
## 15 invertebrate   Sociality  0.413467493 -0.2272736  1.0542085 -2.0250603
## 16       mammal    Activity -0.163065307 -0.5680824  0.2419518 -1.3909083
## 17       mammal  Aggression  0.091224955 -0.2978548  0.4803048 -1.1314722
## 18       mammal    Boldness  0.129883912 -0.2305105  0.4902783 -1.0840218
## 19       mammal Exploration  0.026387251 -0.3500445  0.4028190 -1.1923585
## 20       mammal   Sociality  0.065770946 -0.3292890  0.4608309 -1.1588354
## 21     reptilia    Activity -0.045918474 -0.4728193  0.3809823 -0.7480079
## 22     reptilia  Aggression -0.127350973 -0.4621377  0.2074358 -0.7789961
## 23     reptilia    Boldness  0.106563010 -0.1369802  0.3501062 -0.5044050
## 24     reptilia Exploration  0.251195715  0.0136709  0.4887205 -0.3574618
## 25     reptilia   Sociality  0.004637977 -0.5570016  0.5662775 -0.7843910
##      upperPR
## 1  1.7033599
## 2  1.6351318
## 3  1.6388325
## 4  1.8441795
## 5  1.2334160
## 6  1.9406707
## 7  2.0365079
## 8  1.8646635
## 9  2.0526973
## 10 1.9427507
## 11 2.6369293
## 12 2.8740730
## 13 2.6282572
## 14 2.4354150
## 15 2.8519953
## 16 1.0647777
## 17 1.3139221
## 18 1.3437897
## 19 1.2451330
## 20 1.2903773
## 21 0.6561710
## 22 0.5242942
## 23 0.7175310
## 24 0.8598532
## 25 0.7936670
\end{verbatim}

\begin{Shaded}
\begin{Highlighting}[]
\NormalTok{  MLMA_estimates_lnCVR_pers <-}\StringTok{ }\NormalTok{plyr}\OperatorTok{::}\KeywordTok{ldply}\NormalTok{(}\KeywordTok{lapply}\NormalTok{(lnCVR_mods_pers, }\ControlFlowTok{function}\NormalTok{(x) }\KeywordTok{print}\NormalTok{(}\KeywordTok{mod_results}\NormalTok{(x, }\DataTypeTok{mod =} \StringTok{"personality_trait"}\NormalTok{))))}
\NormalTok{      MLMA_estimates_lnCVR_pers}
\end{Highlighting}
\end{Shaded}

\begin{verbatim}
##             .id        name    estimate      lowerCL     upperCL      lowerPR
## 1          bird    Activity  0.06247105 -0.223810303  0.34875240 -1.547893585
## 2          bird  Aggression -0.07509972 -0.388723954  0.23852451 -1.690523480
## 3          bird    Boldness  0.04002236 -0.177573785  0.25761851 -1.559615558
## 4          bird Exploration -0.29821007 -0.550498369 -0.04592177 -1.902909359
## 5          bird   Sociality  0.14671033 -0.358195237  0.65161591 -1.516228788
## 6          fish    Activity -0.01057724 -0.164934481  0.14378000 -0.744939105
## 7          fish  Aggression -0.12076668 -0.259837252  0.01830390 -0.852083418
## 8          fish    Boldness  0.02281042 -0.082594775  0.12821562 -0.702885678
## 9          fish Exploration -0.03757428 -0.191214061  0.11606550 -0.771786428
## 10         fish   Sociality  0.25131850  0.039101979  0.46353502 -0.497274393
## 11 invertebrate    Activity -0.04431023 -0.187795258  0.09917481 -0.786780181
## 12 invertebrate  Aggression  0.03920221 -0.272921753  0.35132617 -0.753041918
## 13 invertebrate    Boldness -0.02254238 -0.149602850  0.10451808 -0.762031219
## 14 invertebrate Exploration  0.12140566 -0.069729521  0.31254083 -0.631664152
## 15 invertebrate   Sociality  0.10933426 -0.267988928  0.48665745 -0.710630523
## 16       mammal    Activity  0.10970622 -0.183407387  0.40281983 -0.617702167
## 17       mammal  Aggression  0.11346377 -0.183447978  0.41037551 -0.615477836
## 18       mammal    Boldness  0.03065931 -0.237003686  0.29832231 -0.686907726
## 19       mammal Exploration  0.05050685 -0.231448720  0.33246242 -0.672493506
## 20       mammal   Sociality  0.04407016 -0.251715474  0.33985579 -0.684415138
## 21     reptilia    Activity -0.09756095 -0.378433808  0.18331191 -0.374657513
## 22     reptilia  Aggression  0.15707735 -0.007912971  0.32206767 -0.005694702
## 23     reptilia    Boldness  0.08308063 -0.044997677  0.21115893 -0.043275684
## 24     reptilia Exploration -0.08675909 -0.237272780  0.06375460 -0.235249147
## 25     reptilia   Sociality  0.03223093 -0.530881924  0.59534378 -0.523310952
##       upperPR
## 1  1.67283568
## 2  1.54032404
## 3  1.63966028
## 4  1.30648922
## 5  1.80964946
## 6  0.72378463
## 7  0.61055007
## 8  0.74850653
## 9  0.69663786
## 10 0.99991139
## 11 0.69815973
## 12 0.83144634
## 13 0.71694645
## 14 0.87447546
## 15 0.92929904
## 16 0.83711461
## 17 0.84240537
## 18 0.74822634
## 19 0.77350721
## 20 0.77255546
## 21 0.17953562
## 22 0.31984940
## 23 0.20943694
## 24 0.06173097
## 25 0.58777281
\end{verbatim}

\begin{Shaded}
\begin{Highlighting}[]
  \CommentTok{# Add in n and k to these dataframes}
\NormalTok{  n_k<-}\StringTok{ }\NormalTok{pers_new }\OperatorTok
\StringTok{      }\KeywordTok{group_by}\NormalTok{(taxo_group, personality_trait) }\OperatorTok
\StringTok{      }\KeywordTok{summarise}\NormalTok{(}\DataTypeTok{n =} \KeywordTok{n}\NormalTok{(), }\DataTypeTok{spp =} \KeywordTok{length}\NormalTok{(}\KeywordTok{unique}\NormalTok{(spp_name_phylo)), }\DataTypeTok{k =} \KeywordTok{length}\NormalTok{(}\KeywordTok{unique}\NormalTok{(study_ID)))}
  
  \CommentTok{# Summary of model estimates with number of studies, species and effect sizes included}
\NormalTok{  MLMA_estimates_SMD_pers <-}\StringTok{ }\KeywordTok{data.frame}\NormalTok{(MLMA_estimates_SMD_pers, n_k[,}\KeywordTok{c}\NormalTok{(}\StringTok{"n"}\NormalTok{, }\StringTok{"spp"}\NormalTok{, }\StringTok{"k"}\NormalTok{)])}
\NormalTok{      MLMA_estimates_SMD_pers}
\end{Highlighting}
\end{Shaded}

\begin{verbatim}
##             .id        name     estimate    lowerCL    upperCL    lowerPR
## 1          bird    Activity -0.082630938 -0.3731716  0.2079097 -1.8686217
## 2          bird  Aggression -0.150395862 -0.4380609  0.1372692 -1.9359235
## 3          bird    Boldness -0.141981559 -0.3986038  0.1146406 -1.9227956
## 4          bird Exploration  0.061120004 -0.2107229  0.3329629 -1.7219395
## 5          bird   Sociality -0.589437560 -1.0564775 -0.1223977 -2.4122911
## 6          fish    Activity  0.153275118 -0.3114801  0.6180303 -1.6341205
## 7          fish  Aggression  0.252603144 -0.1984714  0.7036777 -1.5313017
## 8          fish    Boldness  0.083577441 -0.3561599  0.5233148 -1.6975086
## 9          fish Exploration  0.265681594 -0.1976033  0.7289665 -1.5213342
## 10         fish   Sociality  0.149525877 -0.3372741  0.6363258 -1.6436989
## 11 invertebrate    Activity  0.250548279 -0.1462921  0.6473887 -2.1358328
## 12 invertebrate  Aggression  0.420366327 -0.2762252  1.1169578 -2.0333404
## 13 invertebrate    Boldness  0.242813292 -0.1483189  0.6339455 -2.1426306
## 14 invertebrate Exploration  0.039752886 -0.4098280  0.4893337 -2.3559092
## 15 invertebrate   Sociality  0.413467493 -0.2272736  1.0542085 -2.0250603
## 16       mammal    Activity -0.163065307 -0.5680824  0.2419518 -1.3909083
## 17       mammal  Aggression  0.091224955 -0.2978548  0.4803048 -1.1314722
## 18       mammal    Boldness  0.129883912 -0.2305105  0.4902783 -1.0840218
## 19       mammal Exploration  0.026387251 -0.3500445  0.4028190 -1.1923585
## 20       mammal   Sociality  0.065770946 -0.3292890  0.4608309 -1.1588354
## 21     reptilia    Activity -0.045918474 -0.4728193  0.3809823 -0.7480079
## 22     reptilia  Aggression -0.127350973 -0.4621377  0.2074358 -0.7789961
## 23     reptilia    Boldness  0.106563010 -0.1369802  0.3501062 -0.5044050
## 24     reptilia Exploration  0.251195715  0.0136709  0.4887205 -0.3574618
## 25     reptilia   Sociality  0.004637977 -0.5570016  0.5662775 -0.7843910
##      upperPR   n spp  k
## 1  1.7033599  60   9 14
## 2  1.6351318  50  10 11
## 3  1.6388325 262  96 24
## 4  1.8441795  77   9 15
## 5  1.2334160  31   2  3
## 6  1.9406707  92   5  9
## 7  2.0365079  95  14 17
## 8  1.8646635 173  13 24
## 9  2.0526973 103   7 10
## 10 1.9427507  27   6  7
## 11 2.6369293 166  17 18
## 12 2.8740730  33   6  5
## 13 2.6282572 164  23 23
## 14 2.4354150  54   6  7
## 15 2.8519953   6   1  1
## 16 1.0647777  83  12 14
## 17 1.3139221  87  14 16
## 18 1.3437897 193  27 27
## 19 1.2451330 213  16 19
## 20 1.2903773  98  10 12
## 21 0.6561710   5   3  3
## 22 0.5242942  30   2  2
## 23 0.7175310  25   3  4
## 24 0.8598532  32   4  5
## 25 0.7936670   3   2  2
\end{verbatim}

\begin{Shaded}
\begin{Highlighting}[]
\NormalTok{  MLMA_estimates_lnCVR_pers <-}\StringTok{ }\KeywordTok{data.frame}\NormalTok{(MLMA_estimates_lnCVR_pers, n_k[,}\KeywordTok{c}\NormalTok{(}\StringTok{"n"}\NormalTok{, }\StringTok{"spp"}\NormalTok{, }\StringTok{"k"}\NormalTok{)])}
\NormalTok{      MLMA_estimates_lnCVR_pers}
\end{Highlighting}
\end{Shaded}

\begin{verbatim}
##             .id        name    estimate      lowerCL     upperCL      lowerPR
## 1          bird    Activity  0.06247105 -0.223810303  0.34875240 -1.547893585
## 2          bird  Aggression -0.07509972 -0.388723954  0.23852451 -1.690523480
## 3          bird    Boldness  0.04002236 -0.177573785  0.25761851 -1.559615558
## 4          bird Exploration -0.29821007 -0.550498369 -0.04592177 -1.902909359
## 5          bird   Sociality  0.14671033 -0.358195237  0.65161591 -1.516228788
## 6          fish    Activity -0.01057724 -0.164934481  0.14378000 -0.744939105
## 7          fish  Aggression -0.12076668 -0.259837252  0.01830390 -0.852083418
## 8          fish    Boldness  0.02281042 -0.082594775  0.12821562 -0.702885678
## 9          fish Exploration -0.03757428 -0.191214061  0.11606550 -0.771786428
## 10         fish   Sociality  0.25131850  0.039101979  0.46353502 -0.497274393
## 11 invertebrate    Activity -0.04431023 -0.187795258  0.09917481 -0.786780181
## 12 invertebrate  Aggression  0.03920221 -0.272921753  0.35132617 -0.753041918
## 13 invertebrate    Boldness -0.02254238 -0.149602850  0.10451808 -0.762031219
## 14 invertebrate Exploration  0.12140566 -0.069729521  0.31254083 -0.631664152
## 15 invertebrate   Sociality  0.10933426 -0.267988928  0.48665745 -0.710630523
## 16       mammal    Activity  0.10970622 -0.183407387  0.40281983 -0.617702167
## 17       mammal  Aggression  0.11346377 -0.183447978  0.41037551 -0.615477836
## 18       mammal    Boldness  0.03065931 -0.237003686  0.29832231 -0.686907726
## 19       mammal Exploration  0.05050685 -0.231448720  0.33246242 -0.672493506
## 20       mammal   Sociality  0.04407016 -0.251715474  0.33985579 -0.684415138
## 21     reptilia    Activity -0.09756095 -0.378433808  0.18331191 -0.374657513
## 22     reptilia  Aggression  0.15707735 -0.007912971  0.32206767 -0.005694702
## 23     reptilia    Boldness  0.08308063 -0.044997677  0.21115893 -0.043275684
## 24     reptilia Exploration -0.08675909 -0.237272780  0.06375460 -0.235249147
## 25     reptilia   Sociality  0.03223093 -0.530881924  0.59534378 -0.523310952
##       upperPR   n spp  k
## 1  1.67283568  60   9 14
## 2  1.54032404  50  10 11
## 3  1.63966028 262  96 24
## 4  1.30648922  77   9 15
## 5  1.80964946  31   2  3
## 6  0.72378463  92   5  9
## 7  0.61055007  95  14 17
## 8  0.74850653 173  13 24
## 9  0.69663786 103   7 10
## 10 0.99991139  27   6  7
## 11 0.69815973 166  17 18
## 12 0.83144634  33   6  5
## 13 0.71694645 164  23 23
## 14 0.87447546  54   6  7
## 15 0.92929904   6   1  1
## 16 0.83711461  83  12 14
## 17 0.84240537  87  14 16
## 18 0.74822634 193  27 27
## 19 0.77350721 213  16 19
## 20 0.77255546  98  10 12
## 21 0.17953562   5   3  3
## 22 0.31984940  30   2  2
## 23 0.20943694  25   3  4
## 24 0.06173097  32   4  5
## 25 0.58777281   3   2  2
\end{verbatim}

Extract p-values from models for multiple testing adjustment later:

\begin{Shaded}
\begin{Highlighting}[]
\CommentTok{# extract p-values for multiple testing }
\NormalTok{  p.SMD_pers <-}\StringTok{ }\KeywordTok{unlist}\NormalTok{(}\KeywordTok{lapply}\NormalTok{(smd_mods_pers, }\ControlFlowTok{function}\NormalTok{(x) x}\OperatorTok{$}\NormalTok{pval))}
\NormalTok{    p.SMD_pers}
\end{Highlighting}
\end{Shaded}

\begin{verbatim}
##         bird1         bird2         bird3         bird4         bird5 
##    0.57653039    0.30479363    0.27751708    0.65883812    0.01348632 
##         fish1         fish2         fish3         fish4         fish5 
##    0.51728620    0.27173344    0.70897900    0.26038453    0.54643814 
## invertebrate1 invertebrate2 invertebrate3 invertebrate4 invertebrate5 
##    0.21528997    0.23621794    0.22305017    0.86210097    0.20535070 
##       mammal1       mammal2       mammal3       mammal4       mammal5 
##    0.42949385    0.64539838    0.47941582    0.89056692    0.74385107 
##     reptilia1     reptilia2     reptilia3     reptilia4     reptilia5 
##    0.83127102    0.45179068    0.38700906    0.03843513    0.98694696
\end{verbatim}

\begin{Shaded}
\begin{Highlighting}[]
\NormalTok{  p.lnCVR_pers <-}\StringTok{ }\KeywordTok{unlist}\NormalTok{(}\KeywordTok{lapply}\NormalTok{(lnCVR_mods_pers, }\ControlFlowTok{function}\NormalTok{(x) x}\OperatorTok{$}\NormalTok{pval))}
\NormalTok{    p.lnCVR_pers}
\end{Highlighting}
\end{Shaded}

\begin{verbatim}
##         bird1         bird2         bird3         bird4         bird5 
##    0.66827246    0.63819433    0.71794905    0.02062033    0.56829592 
##         fish1         fish2         fish3         fish4         fish5 
##    0.89295129    0.08859994    0.67087141    0.63106746    0.02038126 
## invertebrate1 invertebrate2 invertebrate3 invertebrate4 invertebrate5 
##    0.54416559    0.80512014    0.72746311    0.21252880    0.56927261 
##       mammal1       mammal2       mammal3       mammal4       mammal5 
##    0.46265634    0.45330729    0.82211806    0.72515472    0.76995584 
##     reptilia1     reptilia2     reptilia3     reptilia4     reptilia5 
##    0.49192601    0.06178802    0.20080617    0.25517952    0.90971983
\end{verbatim}

\subsection{Personality trait x SSD MLMR
models}\label{personality-trait-x-ssd-mlmr-models}

Now let's look at how SSD interacts with personality trait type. Here we
are not estimating an intercept either, so each intercept varies by
trait category and each slope as well. Note that there are lots of
warnings, but these are the result of many levels not being present in
taxa groups. We chose not to scale SSD\_index because it is easier (and
biologically relevant) to interpret SSD when it is 0 (when males and
females are the same size), and when SSD is positive (when males are
larger than females).

Model summaries are presented in Supplementary Table S17.

\begin{Shaded}
\begin{Highlighting}[]
\CommentTok{# again, we can just reload our saved model output here  }
\NormalTok{rerun_models }\OperatorTok{==}\StringTok{ }\OtherTok{FALSE}
\end{Highlighting}
\end{Shaded}

\begin{verbatim}
## [1] TRUE
\end{verbatim}

\begin{Shaded}
\begin{Highlighting}[]
      \ControlFlowTok{if}\NormalTok{(rerun_models }\OperatorTok{==}\StringTok{ }\OtherTok{TRUE}\NormalTok{)\{}
\NormalTok{      MLMR_models_pers_SSD <-}\StringTok{ }\KeywordTok{meta_model_fits}\NormalTok{(pers_new, phylo_vcv, }\DataTypeTok{type =} \StringTok{"pers_SSD"}\NormalTok{)}
      \KeywordTok{saveRDS}\NormalTok{(MLMR_models_pers_SSD, }\StringTok{"./output/MLMR_models_pers_SSD"}\NormalTok{)}
\NormalTok{    \} }\ControlFlowTok{else}\NormalTok{\{}
\NormalTok{      MLMR_models_pers_SSD <-}\StringTok{ }\KeywordTok{readRDS}\NormalTok{(}\StringTok{"./output/MLMR_models_pers_SSD"}\NormalTok{)}
\NormalTok{    \}  }


  \CommentTok{# Extract the SMD and lnCVR results}
\NormalTok{  smd_mods_pers_SSD <-}\StringTok{ }\NormalTok{MLMR_models_pers_SSD[}\StringTok{"SMD"}\NormalTok{,]}
    
\NormalTok{  lnCVR_mods_pers_SSD <-}\StringTok{ }\NormalTok{MLMR_models_pers_SSD[}\StringTok{"lnCVR"}\NormalTok{,]}
\end{Highlighting}
\end{Shaded}

Get the prediction intervals for our interaction models:

\begin{Shaded}
\begin{Highlighting}[]
  \CommentTok{# extract estimates using modified function in func.R file:}
    \CommentTok{# SMD}
\NormalTok{  MLMA_estimates_SMD_SSD <-}\StringTok{ }\NormalTok{plyr}\OperatorTok{::}\KeywordTok{ldply}\NormalTok{(}\KeywordTok{lapply}\NormalTok{(smd_mods_pers_SSD, }\ControlFlowTok{function}\NormalTok{(x) }
     \KeywordTok{print}\NormalTok{(}\KeywordTok{mod_results_new}\NormalTok{(x, }\DataTypeTok{mod_cat =} \StringTok{"personality_trait"}\NormalTok{, }\DataTypeTok{mod_cont =} \StringTok{"SSD_index"}\NormalTok{, }\DataTypeTok{type =} \StringTok{"zero"}\NormalTok{))))}
    
\NormalTok{    MLMA_estimates_SMD_SSD}
\end{Highlighting}
\end{Shaded}

\begin{verbatim}
##             .id                  name     estimate      lowerCL    upperCL
## 1          bird            Aggression -0.151036532  -0.46852121 0.16644814
## 2          bird              Boldness -0.214280247  -0.50506697 0.07650648
## 3          bird           Exploration  0.074227347  -0.23570163 0.38415632
## 4          bird             Sociality -1.226866406  -2.91329598 0.45956317
## 5          bird  Aggression:SSD_index  1.157852238  -1.77508072 4.09078520
## 6          bird    Boldness:SSD_index -0.848875167  -2.27108269 0.57333236
## 7          bird Exploration:SSD_index -1.321422783  -2.83439647 0.19155090
## 8          bird   Sociality:SSD_index -3.911491684 -12.70365135 4.88066798
## 9          fish            Aggression  0.103585582  -0.23823551 0.44540667
## 10         fish              Boldness  0.140413174  -0.17920769 0.46003403
## 11         fish           Exploration  0.272900300  -0.07480852 0.62060912
## 12         fish             Sociality -0.022130752  -0.41855338 0.37429188
## 13         fish  Aggression:SSD_index  2.389762942  -0.45150502 5.23103090
## 14         fish    Boldness:SSD_index  1.286741330  -1.19395415 3.76743681
## 15         fish Exploration:SSD_index  0.607258807  -2.06820241 3.28272002
## 16         fish   Sociality:SSD_index -2.088881283  -5.26466879 1.08690623
## 17 invertebrate            Aggression  0.834112113  -0.25274144 1.92096567
## 18 invertebrate              Boldness  0.197107566  -0.23590070 0.63011583
## 19 invertebrate           Exploration -0.049461574  -0.67591471 0.57699156
## 20 invertebrate             Sociality  0.257291051  -0.39825935 0.91284145
## 21 invertebrate  Aggression:SSD_index  0.680138870  -3.87195421 5.23223195
## 22 invertebrate    Boldness:SSD_index  0.915400058  -0.22367183 2.05447195
## 23 invertebrate Exploration:SSD_index  0.654641914  -1.74659478 3.05587861
## 24       mammal            Aggression -0.068848442  -0.64585730 0.50816041
## 25       mammal              Boldness -0.004157273  -0.59355508 0.58524054
## 26       mammal           Exploration -0.153246169  -0.73241582 0.42592348
## 27       mammal             Sociality  0.066790898  -0.56353783 0.69711962
## 28       mammal  Aggression:SSD_index  2.607701753   1.05525461 4.16014890
## 29       mammal    Boldness:SSD_index  2.126650354   0.80742684 3.44587387
## 30       mammal Exploration:SSD_index  2.374650037   1.04169176 3.70760831
## 31       mammal   Sociality:SSD_index  1.789427351   0.27966189 3.29919282
## 32     reptilia            Aggression -0.066728598  -0.26994608 0.13648888
## 33     reptilia              Boldness -0.054772986  -0.40164148 0.29209551
## 34     reptilia           Exploration  0.459170599   0.20789126 0.71044993
## 35     reptilia             Sociality -0.022112770  -0.81539374 0.77116820
## 36     reptilia  Aggression:SSD_index  4.070920104   0.11347450 8.02836571
## 37     reptilia    Boldness:SSD_index  1.393904655  -2.56056843 5.34837774
## 38     reptilia Exploration:SSD_index  3.196194595  -0.57224573 6.96463492
## 39     reptilia   Sociality:SSD_index  2.747115585  -2.91125356 8.40548473
##         lowerPR   upperPR
## 1   -2.00561231 1.7035393
## 2   -2.06447260 1.6359121
## 3   -1.77906993 1.9275246
## 4   -3.71336988 1.2596371
## 5   -2.29768687 4.6133913
## 6   -3.16432956 1.4665792
## 7   -3.69370948 1.0508639
## 8  -12.89150986 5.0685265
## 9   -1.54548590 1.7526571
## 10  -1.50420003 1.7850264
## 11  -1.37740165 1.9232022
## 12  -1.68337908 1.6391176
## 13  -0.87755993 5.6570858
## 14  -1.67238780 4.2458705
## 15  -2.51695114 3.7314688
## 16  -5.65093426 1.4731717
## 17  -1.83590711 3.5041313
## 18  -2.27983574 2.6740509
## 19  -2.56743606 2.4685129
## 20  -2.26807988 2.7826620
## 21  -4.48409422 5.8443720
## 22  -1.77629769 3.6070978
## 23  -2.76788488 4.0771687
## 24  -1.47455435 1.3368575
## 25  -1.41499379 1.4066792
## 26  -1.55984042 1.2533481
## 27  -1.36162927 1.4952111
## 28   0.59445511 4.6209484
## 29   0.28724269 3.9660580
## 30   0.52536703 4.2239330
## 31  -0.19109320 3.7699479
## 32  -0.56457573 0.4311185
## 33  -0.62650034 0.5169544
## 34  -0.06015177 0.9784930
## 35  -0.93636061 0.8921351
## 36   0.08746304 8.0543772
## 37  -2.58659930 5.3744086
## 38  -0.59955261 6.9919418
## 39  -2.92947633 8.4237075
\end{verbatim}

\begin{Shaded}
\begin{Highlighting}[]
    \CommentTok{# lnCVR}
\NormalTok{  MLMA_estimates_lnCVR_SSD <-}\StringTok{ }\NormalTok{plyr}\OperatorTok{::}\KeywordTok{ldply}\NormalTok{(}\KeywordTok{lapply}\NormalTok{(lnCVR_mods_pers_SSD, }\ControlFlowTok{function}\NormalTok{(x) }
   \KeywordTok{print}\NormalTok{(}\KeywordTok{mod_results_new}\NormalTok{(x, }\DataTypeTok{mod_cat =} \StringTok{"personality_trait"}\NormalTok{, }\DataTypeTok{mod_cont =} \StringTok{"SSD_index"}\NormalTok{, }\DataTypeTok{type =} \StringTok{"zero"}\NormalTok{))))}
    
\NormalTok{    MLMA_estimates_lnCVR_SSD}
\end{Highlighting}
\end{Shaded}

\begin{verbatim}
##             .id                  name     estimate     lowerCL     upperCL
## 1          bird            Aggression -0.140533040 -0.50369345  0.22262737
## 2          bird              Boldness  0.041902624 -0.22089894  0.30470418
## 3          bird           Exploration -0.423453401 -0.72575953 -0.12114727
## 4          bird             Sociality  1.657798939 -0.47785027  3.79344814
## 5          bird  Aggression:SSD_index  0.229910712 -3.09632946  3.55615089
## 6          bird    Boldness:SSD_index  1.444566031 -0.29662027  3.18575233
## 7          bird Exploration:SSD_index  2.360671927  0.44152412  4.27981974
## 8          bird   Sociality:SSD_index  9.159821298 -1.91573451 20.23537710
## 9          fish            Aggression -0.129070115 -0.27194461  0.01380438
## 10         fish              Boldness  0.005436362 -0.10297982  0.11385254
## 11         fish           Exploration -0.008212732 -0.18136948  0.16494401
## 12         fish             Sociality  0.087624529 -0.16112582  0.33637487
## 13         fish  Aggression:SSD_index  0.530238759 -1.68577830  2.74625582
## 14         fish    Boldness:SSD_index  0.605617604 -1.48753113  2.69876634
## 15         fish Exploration:SSD_index  1.031331626 -1.22079150  3.28345475
## 16         fish   Sociality:SSD_index -1.236517607 -3.78677304  1.31373783
## 17 invertebrate            Aggression -0.203555549 -0.58086635  0.17375525
## 18 invertebrate              Boldness  0.001279023 -0.14241642  0.14497446
## 19 invertebrate           Exploration  0.059489195 -0.25144755  0.37042594
## 20 invertebrate             Sociality  0.128494477 -0.25716983  0.51415878
## 21 invertebrate  Aggression:SSD_index  1.053333652 -0.28691645  2.39358376
## 22 invertebrate    Boldness:SSD_index -0.019558502 -0.70714726  0.66803025
## 23 invertebrate Exploration:SSD_index -0.552874790 -2.17236028  1.06661070
## 24       mammal            Aggression  0.116404800 -0.20920302  0.44201262
## 25       mammal              Boldness  0.050279755 -0.28084944  0.38140895
## 26       mammal           Exploration  0.030879566 -0.29354193  0.35530106
## 27       mammal             Sociality  0.121987951 -0.23596112  0.47993703
## 28       mammal  Aggression:SSD_index -0.138782253 -1.23324702  0.95568251
## 29       mammal    Boldness:SSD_index -0.160814667 -1.01376179  0.69213246
## 30       mammal Exploration:SSD_index -0.002060956 -0.88930493  0.88518301
## 31       mammal   Sociality:SSD_index -0.357422340 -1.35708312  0.64223844
## 32     reptilia            Aggression  0.280341067  0.09692606  0.46375608
## 33     reptilia              Boldness  0.041552023 -0.19703884  0.28014288
## 34     reptilia           Exploration -0.019207567 -0.19364071  0.15522557
## 35     reptilia             Sociality  0.033598276 -1.39712811  1.46432466
## 36     reptilia  Aggression:SSD_index -1.255533966 -7.56757841  5.05651047
## 37     reptilia    Boldness:SSD_index  2.075132144 -3.85381727  8.00408156
## 38     reptilia Exploration:SSD_index  2.934481218 -2.94935384  8.81831627
## 39     reptilia   Sociality:SSD_index  2.311721057 -6.17723397 10.80067608
##        lowerPR    upperPR
## 1  -1.83260874  1.5515427
## 2  -1.63150687  1.7153121
## 3  -2.10351997  1.2566132
## 4  -1.04261440  4.3582123
## 5  -3.48426596  3.9440874
## 6  -0.95605148  3.8451835
## 7  -0.17198735  4.8933312
## 8  -2.03835583 20.3579984
## 9  -0.85781158  0.5996714
## 10 -0.71733959  0.7282123
## 11 -0.74349109  0.7270656
## 12 -0.66903110  0.8442802
## 13 -1.79814751  2.8586250
## 14 -1.60615152  2.8173867
## 15 -1.33144419  3.3941074
## 16 -3.88499902  1.4119638
## 17 -1.02495455  0.6178435
## 18 -0.74234807  0.7449061
## 19 -0.73361517  0.8525936
## 20 -0.69677507  0.9537640
## 21 -0.47264250  2.5793098
## 22 -1.02211083  0.9829938
## 23 -2.32912537  1.2233758
## 24 -0.63758614  0.8703957
## 25 -0.70611196  0.8066715
## 26 -0.72259982  0.7843590
## 27 -0.64652322  0.8904991
## 28 -1.42732213  1.1497576
## 29 -1.25168619  0.9300569
## 30 -1.11995350  1.1158316
## 31 -1.56647303  0.8516283
## 32  0.09686176  0.4638204
## 33 -0.19708827  0.2801923
## 34 -0.19370831  0.1552932
## 35 -1.39713636  1.4643329
## 36 -7.56758027  5.0565123
## 37 -3.85381926  8.0040835
## 38 -2.94935584  8.8183183
## 39 -6.17723536 10.8006775
\end{verbatim}

\begin{Shaded}
\begin{Highlighting}[]
  \CommentTok{# Table to get species numbers, no. studies and no. effect sizes:}
  \KeywordTok{data.frame}\NormalTok{(pers_new }\OperatorTok
\StringTok{  }\KeywordTok{group_by}\NormalTok{(taxo_group, personality_trait) }\OperatorTok
\StringTok{  }\KeywordTok{filter}\NormalTok{(}\OperatorTok{!}\KeywordTok{is.na}\NormalTok{(SSD_index))}\OperatorTok
\StringTok{  }\KeywordTok{summarise}\NormalTok{(}\DataTypeTok{n =} \KeywordTok{n}\NormalTok{(), }\DataTypeTok{N_spp =} \KeywordTok{length}\NormalTok{(}\KeywordTok{unique}\NormalTok{(spp_name_phylo)), }\DataTypeTok{N_studies =} \KeywordTok{length}\NormalTok{(}\KeywordTok{unique}\NormalTok{(study_ID))))}
\end{Highlighting}
\end{Shaded}

\begin{verbatim}
##      taxo_group personality_trait   n N_spp N_studies
## 1          bird          activity  60     9        14
## 2          bird        aggression  41     8         9
## 3          bird          boldness 234    78        21
## 4          bird       exploration  77     9        15
## 5          bird         sociality  31     2         3
## 6          fish          activity  92     5         9
## 7          fish        aggression  93    13        16
## 8          fish          boldness 171    12        23
## 9          fish       exploration 101     6         9
## 10         fish         sociality  27     6         7
## 11 invertebrate          activity 165    16        18
## 12 invertebrate        aggression  32     5         5
## 13 invertebrate          boldness 164    23        23
## 14 invertebrate       exploration  54     6         7
## 15 invertebrate         sociality   6     1         1
## 16       mammal          activity  83    12        14
## 17       mammal        aggression  85    13        15
## 18       mammal          boldness 163    26        26
## 19       mammal       exploration 213    16        19
## 20       mammal         sociality  89     9        11
## 21     reptilia          activity   5     3         3
## 22     reptilia        aggression  30     2         2
## 23     reptilia          boldness  25     3         4
## 24     reptilia       exploration  32     4         5
## 25     reptilia         sociality   3     2         2
\end{verbatim}

\subsection{SSD subset models}\label{ssd-subset-models}

Because we aren't really interested in how each trait type differs from
each other, we need to run our SSD models on subsets of the data where
we can get the mean estimates for individual trait types and for SSD.
Model summaries are presented in Supplementary Table S16.

NOTE: Since we are conducting our meta-regression at the species level
(the level at which we can assume effect sizes are independent), any
personality trait with fewer than 10 species needs to be dropped to look
at interactions between SSD and personality. Having a minimum of 10
studies etc. is the rule of thumb for meta-regressions (e.g.~see
Borenstein et al Intro to Meta-A)

\subsubsection{NEED TO DROP:}\label{need-to-drop}

\begin{enumerate}
\def\labelenumi{\arabic{enumi}.}
\tightlist
\item
  ALL REPTILES - GONE, NOT ENOUGH SPECIES
\item
  BIRDS - EVERYTHING BUT BOLDNESS
\item
  FISH - ACTIVITY, EXPLORATION \& SOCIALITY
\item
  INVERTS - SOCIALITY, EXPLORATION \& AGGRESSION
\item
  MAMMALS - SOCIALITY
\end{enumerate}

\subsubsection{Mammals}\label{mammals}

SSD for activity, boldness, aggression and exploration.

\begin{Shaded}
\begin{Highlighting}[]
\CommentTok{# 1. MAMMALS}

\CommentTok{# First, we need to subset our pers dataset by taxo group to drop the unwanted levels.}
  \CommentTok{# a. activity}
\NormalTok{    pers_new_mammal_activity <-}\StringTok{ }\KeywordTok{as.data.frame}\NormalTok{(pers_new }\OperatorTok
\StringTok{    }\KeywordTok{filter}\NormalTok{(personality_trait }\OperatorTok{==}\StringTok{ "activity"}\NormalTok{) }\OperatorTok
\StringTok{    }\KeywordTok{filter}\NormalTok{(taxo_group }\OperatorTok{==}\StringTok{ "mammal"}\NormalTok{)) }
  \CommentTok{# b. boldness}
\NormalTok{    pers_new_mammal_boldness <-}\StringTok{ }\KeywordTok{as.data.frame}\NormalTok{(pers_new }\OperatorTok
\StringTok{    }\KeywordTok{filter}\NormalTok{(personality_trait }\OperatorTok{==}\StringTok{ "boldness"}\NormalTok{) }\OperatorTok
\StringTok{    }\KeywordTok{filter}\NormalTok{(taxo_group }\OperatorTok{==}\StringTok{ "mammal"}\NormalTok{)) }
  \CommentTok{# c. aggression}
\NormalTok{    pers_new_mammal_aggression <-}\StringTok{ }\KeywordTok{as.data.frame}\NormalTok{(pers_new }\OperatorTok
\StringTok{    }\KeywordTok{filter}\NormalTok{(personality_trait }\OperatorTok{==}\StringTok{ "aggression"}\NormalTok{) }\OperatorTok
\StringTok{    }\KeywordTok{filter}\NormalTok{(taxo_group }\OperatorTok{==}\StringTok{ "mammal"}\NormalTok{)) }
  \CommentTok{# d. exploration}
\NormalTok{    pers_new_mammal_exploration <-}\StringTok{ }\KeywordTok{as.data.frame}\NormalTok{(pers_new }\OperatorTok
\StringTok{    }\KeywordTok{filter}\NormalTok{(personality_trait }\OperatorTok{==}\StringTok{ "exploration"}\NormalTok{) }\OperatorTok
\StringTok{    }\KeywordTok{filter}\NormalTok{(taxo_group }\OperatorTok{==}\StringTok{ "mammal"}\NormalTok{)) }

  \CommentTok{# Extract the phylogenetic correlation matrices}
\NormalTok{     phylo_vcv_mammal <-}\StringTok{ }\NormalTok{phylo_vcv[[}\DecValTok{4}\NormalTok{]] }
\end{Highlighting}
\end{Shaded}

Activity:

\begin{Shaded}
\begin{Highlighting}[]
    \CommentTok{# a. activity}
    \CommentTok{#SMD}
\NormalTok{    MLMR_mods_pers_SSD_mammal_activity_SMD <-}\StringTok{ }\KeywordTok{rma.mv}\NormalTok{(SMD_yi_flip }\OperatorTok{~}\StringTok{ }\NormalTok{SSD_index, }\DataTypeTok{V =}\NormalTok{ SMD_vi, }
                                          \DataTypeTok{random =} \KeywordTok{list}\NormalTok{(}\OperatorTok{~}\DecValTok{1}\OperatorTok{|}\NormalTok{study_ID, }\OperatorTok{~}\DecValTok{1}\OperatorTok{|}\NormalTok{spp_name_phylo, }\OperatorTok{~}\DecValTok{1}\OperatorTok{|}\NormalTok{obs), }
                                          \DataTypeTok{R =} \KeywordTok{list}\NormalTok{(}\DataTypeTok{spp_name_phylo=}\NormalTok{phylo_vcv_mammal), }\DataTypeTok{control=}\KeywordTok{list}\NormalTok{(}\DataTypeTok{optimizer=}\StringTok{"optim"}\NormalTok{), }
                                          \DataTypeTok{test =} \StringTok{"t"}\NormalTok{, }\DataTypeTok{data =}\NormalTok{ pers_new_mammal_activity)}
\end{Highlighting}
\end{Shaded}

\begin{verbatim}
## Warning in rma.mv(SMD_yi_flip ~ SSD_index, V = SMD_vi, random = list(~1 | :
## There are rows/columns in the 'R' matrix for 'spp_name_phylo' for which there
## are no data.
\end{verbatim}

\begin{Shaded}
\begin{Highlighting}[]
\NormalTok{    MLMR_mods_pers_SSD_mammal_activity_SMD}
\end{Highlighting}
\end{Shaded}

\begin{verbatim}
## 
## Multivariate Meta-Analysis Model (k = 83; method: REML)
## 
## Variance Components:
## 
##             estim    sqrt  nlvls  fixed          factor    R 
## sigma^2.1  0.1000  0.3163     14     no        study_ID   no 
## sigma^2.2  2.8075  1.6755     12     no  spp_name_phylo  yes 
## sigma^2.3  0.1914  0.4375     83     no             obs   no 
## 
## Test for Residual Heterogeneity:
## QE(df = 81) = 320.2151, p-val < .0001
## 
## Test of Moderators (coefficient 2):
## F(df1 = 1, df2 = 81) = 5.0697, p-val = 0.0271
## 
## Model Results:
## 
##            estimate      se     tval  df    pval    ci.lb    ci.ub 
## intrcpt      0.5040  1.2531   0.4022  81  0.6886  -1.9892   2.9972    
## SSD_index   -2.2054  0.9795  -2.2516  81  0.0271  -4.1543  -0.2565  * 
## 
## ---
## Signif. codes:  0 '***' 0.001 '**' 0.01 '*' 0.05 '.' 0.1 ' ' 1
\end{verbatim}

\begin{Shaded}
\begin{Highlighting}[]
    \CommentTok{#lnCVR}
\NormalTok{    MLMR_mods_pers_SSD_mammal_activity_lncvr <-}\StringTok{ }\KeywordTok{rma.mv}\NormalTok{(CVR_yi }\OperatorTok{~}\StringTok{ }\NormalTok{SSD_index, }\DataTypeTok{V =}\NormalTok{ CVR_vi, }
                                            \DataTypeTok{random =} \KeywordTok{list}\NormalTok{(}\OperatorTok{~}\DecValTok{1}\OperatorTok{|}\NormalTok{study_ID, }\OperatorTok{~}\DecValTok{1}\OperatorTok{|}\NormalTok{spp_name_phylo, }\OperatorTok{~}\DecValTok{1}\OperatorTok{|}\NormalTok{obs), }
                                            \DataTypeTok{R =} \KeywordTok{list}\NormalTok{(}\DataTypeTok{spp_name_phylo=}\NormalTok{phylo_vcv_mammal), }\DataTypeTok{control=}\KeywordTok{list}\NormalTok{(}\DataTypeTok{optimizer=}\StringTok{"optim"}\NormalTok{), }
                                            \DataTypeTok{test =} \StringTok{"t"}\NormalTok{, }\DataTypeTok{data =}\NormalTok{ pers_new_mammal_activity)}
\end{Highlighting}
\end{Shaded}

\begin{verbatim}
## Warning in rma.mv(CVR_yi ~ SSD_index, V = CVR_vi, random = list(~1 | study_ID, :
## There are rows/columns in the 'R' matrix for 'spp_name_phylo' for which there
## are no data.
\end{verbatim}

\begin{Shaded}
\begin{Highlighting}[]
\NormalTok{    MLMR_mods_pers_SSD_mammal_activity_lncvr}
\end{Highlighting}
\end{Shaded}

\begin{verbatim}
## 
## Multivariate Meta-Analysis Model (k = 83; method: REML)
## 
## Variance Components:
## 
##             estim    sqrt  nlvls  fixed          factor    R 
## sigma^2.1  0.0293  0.1710     14     no        study_ID   no 
## sigma^2.2  0.0001  0.0079     12     no  spp_name_phylo  yes 
## sigma^2.3  0.0607  0.2465     83     no             obs   no 
## 
## Test for Residual Heterogeneity:
## QE(df = 81) = 146.2596, p-val < .0001
## 
## Test of Moderators (coefficient 2):
## F(df1 = 1, df2 = 81) = 0.1324, p-val = 0.7169
## 
## Model Results:
## 
##            estimate      se    tval  df    pval    ci.lb   ci.ub 
## intrcpt      0.0518  0.0988  0.5236  81  0.6020  -0.1449  0.2484    
## SSD_index    0.1248  0.3430  0.3638  81  0.7169  -0.5577  0.8073    
## 
## ---
## Signif. codes:  0 '***' 0.001 '**' 0.01 '*' 0.05 '.' 0.1 ' ' 1
\end{verbatim}

Boldness:

\begin{Shaded}
\begin{Highlighting}[]
    \CommentTok{# b. boldness}
    \CommentTok{#SMD}
\NormalTok{    MLMR_mods_pers_SSD_mammal_bold_SMD <-}\StringTok{ }\KeywordTok{rma.mv}\NormalTok{(SMD_yi_flip }\OperatorTok{~}\StringTok{ }\NormalTok{SSD_index, }\DataTypeTok{V =}\NormalTok{ SMD_vi, }
                                              \DataTypeTok{random =} \KeywordTok{list}\NormalTok{(}\OperatorTok{~}\DecValTok{1}\OperatorTok{|}\NormalTok{study_ID, }\OperatorTok{~}\DecValTok{1}\OperatorTok{|}\NormalTok{spp_name_phylo, }\OperatorTok{~}\DecValTok{1}\OperatorTok{|}\NormalTok{obs), }
                                              \DataTypeTok{R =} \KeywordTok{list}\NormalTok{(}\DataTypeTok{spp_name_phylo=}\NormalTok{phylo_vcv_mammal), }\DataTypeTok{control=}\KeywordTok{list}\NormalTok{(}\DataTypeTok{optimizer=}\StringTok{"optim"}\NormalTok{), }
                                              \DataTypeTok{test =} \StringTok{"t"}\NormalTok{, }\DataTypeTok{data =}\NormalTok{ pers_new_mammal_boldness)}
\end{Highlighting}
\end{Shaded}

\begin{verbatim}
## Warning in rma.mv(SMD_yi_flip ~ SSD_index, V = SMD_vi, random = list(~1 | :
## There are rows/columns in the 'R' matrix for 'spp_name_phylo' for which there
## are no data.
\end{verbatim}

\begin{verbatim}
## Warning: Rows with NAs omitted from model fitting.
\end{verbatim}

\begin{Shaded}
\begin{Highlighting}[]
\NormalTok{    MLMR_mods_pers_SSD_mammal_bold_SMD    }
\end{Highlighting}
\end{Shaded}

\begin{verbatim}
## 
## Multivariate Meta-Analysis Model (k = 163; method: REML)
## 
## Variance Components:
## 
##             estim    sqrt  nlvls  fixed          factor    R 
## sigma^2.1  0.0088  0.0938     26     no        study_ID   no 
## sigma^2.2  0.0000  0.0050     26     no  spp_name_phylo  yes 
## sigma^2.3  0.1707  0.4132    163     no             obs   no 
## 
## Test for Residual Heterogeneity:
## QE(df = 161) = 405.7659, p-val < .0001
## 
## Test of Moderators (coefficient 2):
## F(df1 = 1, df2 = 161) = 1.3101, p-val = 0.2541
## 
## Model Results:
## 
##            estimate      se     tval   df    pval    ci.lb   ci.ub 
## intrcpt      0.0739  0.0774   0.9547  161  0.3412  -0.0789  0.2267    
## SSD_index   -0.1686  0.1473  -1.1446  161  0.2541  -0.4594  0.1223    
## 
## ---
## Signif. codes:  0 '***' 0.001 '**' 0.01 '*' 0.05 '.' 0.1 ' ' 1
\end{verbatim}

\begin{Shaded}
\begin{Highlighting}[]
    \CommentTok{#lnCVR}
\NormalTok{    MLMR_mods_pers_SSD_mammal_bold_lncvr <-}\StringTok{ }\KeywordTok{rma.mv}\NormalTok{(CVR_yi }\OperatorTok{~}\StringTok{ }\NormalTok{SSD_index, }\DataTypeTok{V =}\NormalTok{ CVR_vi, }
                                            \DataTypeTok{random =} \KeywordTok{list}\NormalTok{(}\OperatorTok{~}\DecValTok{1}\OperatorTok{|}\NormalTok{study_ID, }\OperatorTok{~}\DecValTok{1}\OperatorTok{|}\NormalTok{spp_name_phylo, }\OperatorTok{~}\DecValTok{1}\OperatorTok{|}\NormalTok{obs), }
                                            \DataTypeTok{R =} \KeywordTok{list}\NormalTok{(}\DataTypeTok{spp_name_phylo=}\NormalTok{phylo_vcv_mammal), }\DataTypeTok{control=}\KeywordTok{list}\NormalTok{(}\DataTypeTok{optimizer=}\StringTok{"optim"}\NormalTok{), }
                                            \DataTypeTok{test =} \StringTok{"t"}\NormalTok{, }\DataTypeTok{data =}\NormalTok{ pers_new_mammal_boldness)}
\end{Highlighting}
\end{Shaded}

\begin{verbatim}
## Warning in rma.mv(CVR_yi ~ SSD_index, V = CVR_vi, random = list(~1 | study_ID, :
## There are rows/columns in the 'R' matrix for 'spp_name_phylo' for which there
## are no data.

## Warning in rma.mv(CVR_yi ~ SSD_index, V = CVR_vi, random = list(~1 | study_ID, :
## Rows with NAs omitted from model fitting.
\end{verbatim}

\begin{Shaded}
\begin{Highlighting}[]
\NormalTok{    MLMR_mods_pers_SSD_mammal_bold_lncvr}
\end{Highlighting}
\end{Shaded}

\begin{verbatim}
## 
## Multivariate Meta-Analysis Model (k = 163; method: REML)
## 
## Variance Components:
## 
##             estim    sqrt  nlvls  fixed          factor    R 
## sigma^2.1  0.0029  0.0543     26     no        study_ID   no 
## sigma^2.2  0.0000  0.0027     26     no  spp_name_phylo  yes 
## sigma^2.3  0.0211  0.1452    163     no             obs   no 
## 
## Test for Residual Heterogeneity:
## QE(df = 161) = 177.7499, p-val = 0.1737
## 
## Test of Moderators (coefficient 2):
## F(df1 = 1, df2 = 161) = 2.1066, p-val = 0.1486
## 
## Model Results:
## 
##            estimate      se    tval   df    pval    ci.lb   ci.ub 
## intrcpt      0.0114  0.0513  0.2230  161  0.8238  -0.0899  0.1128    
## SSD_index    0.1316  0.0907  1.4514  161  0.1486  -0.0475  0.3106    
## 
## ---
## Signif. codes:  0 '***' 0.001 '**' 0.01 '*' 0.05 '.' 0.1 ' ' 1
\end{verbatim}

Aggression:

\begin{Shaded}
\begin{Highlighting}[]
    \CommentTok{# c. aggression}
    \CommentTok{#SMD}
\NormalTok{    MLMR_mods_pers_SSD_mammal_aggression_SMD <-}\StringTok{ }\KeywordTok{rma.mv}\NormalTok{(SMD_yi_flip }\OperatorTok{~}\StringTok{ }\NormalTok{SSD_index, }\DataTypeTok{V =}\NormalTok{ SMD_vi, }
                                          \DataTypeTok{random =} \KeywordTok{list}\NormalTok{(}\OperatorTok{~}\DecValTok{1}\OperatorTok{|}\NormalTok{study_ID, }\OperatorTok{~}\DecValTok{1}\OperatorTok{|}\NormalTok{spp_name_phylo, }\OperatorTok{~}\DecValTok{1}\OperatorTok{|}\NormalTok{obs), }
                                          \DataTypeTok{R =} \KeywordTok{list}\NormalTok{(}\DataTypeTok{spp_name_phylo=}\NormalTok{phylo_vcv_mammal), }\DataTypeTok{control=}\KeywordTok{list}\NormalTok{(}\DataTypeTok{optimizer=}\StringTok{"optim"}\NormalTok{), }
                                          \DataTypeTok{test =} \StringTok{"t"}\NormalTok{, }\DataTypeTok{data =}\NormalTok{ pers_new_mammal_aggression)}
\end{Highlighting}
\end{Shaded}

\begin{verbatim}
## Warning in rma.mv(SMD_yi_flip ~ SSD_index, V = SMD_vi, random = list(~1 | :
## There are rows/columns in the 'R' matrix for 'spp_name_phylo' for which there
## are no data.
\end{verbatim}

\begin{verbatim}
## Warning: Rows with NAs omitted from model fitting.
\end{verbatim}

\begin{Shaded}
\begin{Highlighting}[]
\NormalTok{    MLMR_mods_pers_SSD_mammal_aggression_SMD}
\end{Highlighting}
\end{Shaded}

\begin{verbatim}
## 
## Multivariate Meta-Analysis Model (k = 85; method: REML)
## 
## Variance Components:
## 
##             estim    sqrt  nlvls  fixed          factor    R 
## sigma^2.1  0.0000  0.0039     15     no        study_ID   no 
## sigma^2.2  0.6852  0.8277     13     no  spp_name_phylo  yes 
## sigma^2.3  0.1430  0.3781     85     no             obs   no 
## 
## Test for Residual Heterogeneity:
## QE(df = 83) = 312.3189, p-val < .0001
## 
## Test of Moderators (coefficient 2):
## F(df1 = 1, df2 = 83) = 4.2481, p-val = 0.0424
## 
## Model Results:
## 
##            estimate      se     tval  df    pval    ci.lb   ci.ub 
## intrcpt     -0.1036  0.6014  -0.1722  83  0.8637  -1.2997  1.0926    
## SSD_index    1.4134  0.6858   2.0611  83  0.0424   0.0495  2.7774  * 
## 
## ---
## Signif. codes:  0 '***' 0.001 '**' 0.01 '*' 0.05 '.' 0.1 ' ' 1
\end{verbatim}

\begin{Shaded}
\begin{Highlighting}[]
    \CommentTok{#lnCVR}
\NormalTok{    MLMR_mods_pers_SSD_mammal_aggression_lncvr <-}\StringTok{ }\KeywordTok{rma.mv}\NormalTok{(CVR_yi }\OperatorTok{~}\StringTok{ }\NormalTok{SSD_index, }\DataTypeTok{V =}\NormalTok{ CVR_vi, }
                                            \DataTypeTok{random =} \KeywordTok{list}\NormalTok{(}\OperatorTok{~}\DecValTok{1}\OperatorTok{|}\NormalTok{study_ID, }\OperatorTok{~}\DecValTok{1}\OperatorTok{|}\NormalTok{spp_name_phylo, }\OperatorTok{~}\DecValTok{1}\OperatorTok{|}\NormalTok{obs), }
                                            \DataTypeTok{R =} \KeywordTok{list}\NormalTok{(}\DataTypeTok{spp_name_phylo=}\NormalTok{phylo_vcv_mammal), }\DataTypeTok{control=}\KeywordTok{list}\NormalTok{(}\DataTypeTok{optimizer=}\StringTok{"optim"}\NormalTok{), }
                                            \DataTypeTok{test =} \StringTok{"t"}\NormalTok{, }\DataTypeTok{data =}\NormalTok{ pers_new_mammal_aggression)}
\end{Highlighting}
\end{Shaded}

\begin{verbatim}
## Warning in rma.mv(CVR_yi ~ SSD_index, V = CVR_vi, random = list(~1 | study_ID, :
## There are rows/columns in the 'R' matrix for 'spp_name_phylo' for which there
## are no data.

## Warning in rma.mv(CVR_yi ~ SSD_index, V = CVR_vi, random = list(~1 | study_ID, :
## Rows with NAs omitted from model fitting.
\end{verbatim}

\begin{Shaded}
\begin{Highlighting}[]
\NormalTok{    MLMR_mods_pers_SSD_mammal_aggression_lncvr}
\end{Highlighting}
\end{Shaded}

\begin{verbatim}
## 
## Multivariate Meta-Analysis Model (k = 85; method: REML)
## 
## Variance Components:
## 
##             estim    sqrt  nlvls  fixed          factor    R 
## sigma^2.1  0.1790  0.4231     15     no        study_ID   no 
## sigma^2.2  0.0000  0.0052     13     no  spp_name_phylo  yes 
## sigma^2.3  0.1539  0.3922     85     no             obs   no 
## 
## Test for Residual Heterogeneity:
## QE(df = 83) = 202.3514, p-val < .0001
## 
## Test of Moderators (coefficient 2):
## F(df1 = 1, df2 = 83) = 0.0118, p-val = 0.9138
## 
## Model Results:
## 
##            estimate      se     tval  df    pval    ci.lb   ci.ub 
## intrcpt      0.0992  0.1534   0.6467  83  0.5196  -0.2059  0.4042    
## SSD_index   -0.0756  0.6959  -0.1086  83  0.9138  -1.4596  1.3084    
## 
## ---
## Signif. codes:  0 '***' 0.001 '**' 0.01 '*' 0.05 '.' 0.1 ' ' 1
\end{verbatim}

Exploration:

\begin{Shaded}
\begin{Highlighting}[]
  \CommentTok{# d. exploration}
     \CommentTok{#SMD}
\NormalTok{    MLMR_mods_pers_SSD_mammal_explore_SMD <-}\StringTok{ }\KeywordTok{rma.mv}\NormalTok{(SMD_yi_flip }\OperatorTok{~}\StringTok{ }\NormalTok{SSD_index, }\DataTypeTok{V =}\NormalTok{ SMD_vi, }
                                          \DataTypeTok{random =} \KeywordTok{list}\NormalTok{(}\OperatorTok{~}\DecValTok{1}\OperatorTok{|}\NormalTok{study_ID, }\OperatorTok{~}\DecValTok{1}\OperatorTok{|}\NormalTok{spp_name_phylo, }\OperatorTok{~}\DecValTok{1}\OperatorTok{|}\NormalTok{obs), }
                                          \DataTypeTok{R =} \KeywordTok{list}\NormalTok{(}\DataTypeTok{spp_name_phylo=}\NormalTok{phylo_vcv_mammal), }\DataTypeTok{control=}\KeywordTok{list}\NormalTok{(}\DataTypeTok{optimizer=}\StringTok{"optim"}\NormalTok{), }
                                          \DataTypeTok{test =} \StringTok{"t"}\NormalTok{, }\DataTypeTok{data =}\NormalTok{ pers_new_mammal_exploration)}
\end{Highlighting}
\end{Shaded}

\begin{verbatim}
## Warning in rma.mv(SMD_yi_flip ~ SSD_index, V = SMD_vi, random = list(~1 | :
## There are rows/columns in the 'R' matrix for 'spp_name_phylo' for which there
## are no data.
\end{verbatim}

\begin{Shaded}
\begin{Highlighting}[]
\NormalTok{    MLMR_mods_pers_SSD_mammal_explore_SMD}
\end{Highlighting}
\end{Shaded}

\begin{verbatim}
## 
## Multivariate Meta-Analysis Model (k = 213; method: REML)
## 
## Variance Components:
## 
##             estim    sqrt  nlvls  fixed          factor    R 
## sigma^2.1  0.0504  0.2244     19     no        study_ID   no 
## sigma^2.2  0.0000  0.0048     16     no  spp_name_phylo  yes 
## sigma^2.3  0.1331  0.3649    213     no             obs   no 
## 
## Test for Residual Heterogeneity:
## QE(df = 211) = 658.4587, p-val < .0001
## 
## Test of Moderators (coefficient 2):
## F(df1 = 1, df2 = 211) = 0.0351, p-val = 0.8516
## 
## Model Results:
## 
##            estimate      se     tval   df    pval    ci.lb   ci.ub 
## intrcpt     -0.0016  0.0914  -0.0173  211  0.9862  -0.1817  0.1786    
## SSD_index   -0.0522  0.2786  -0.1873  211  0.8516  -0.6015  0.4971    
## 
## ---
## Signif. codes:  0 '***' 0.001 '**' 0.01 '*' 0.05 '.' 0.1 ' ' 1
\end{verbatim}

\begin{Shaded}
\begin{Highlighting}[]
    \CommentTok{#lnCVR}
\NormalTok{    MLMR_mods_pers_SSD_mammal_explore_lncvr <-}\StringTok{ }\KeywordTok{rma.mv}\NormalTok{(CVR_yi }\OperatorTok{~}\StringTok{ }\NormalTok{SSD_index, }\DataTypeTok{V =}\NormalTok{ CVR_vi, }
                                            \DataTypeTok{random =} \KeywordTok{list}\NormalTok{(}\OperatorTok{~}\DecValTok{1}\OperatorTok{|}\NormalTok{study_ID, }\OperatorTok{~}\DecValTok{1}\OperatorTok{|}\NormalTok{spp_name_phylo, }\OperatorTok{~}\DecValTok{1}\OperatorTok{|}\NormalTok{obs), }
                                            \DataTypeTok{R =} \KeywordTok{list}\NormalTok{(}\DataTypeTok{spp_name_phylo=}\NormalTok{phylo_vcv_mammal), }\DataTypeTok{control=}\KeywordTok{list}\NormalTok{(}\DataTypeTok{optimizer=}\StringTok{"optim"}\NormalTok{), }
                                            \DataTypeTok{test =} \StringTok{"t"}\NormalTok{, }\DataTypeTok{data =}\NormalTok{ pers_new_mammal_exploration)}
\end{Highlighting}
\end{Shaded}

\begin{verbatim}
## Warning in rma.mv(CVR_yi ~ SSD_index, V = CVR_vi, random = list(~1 | study_ID, :
## There are rows/columns in the 'R' matrix for 'spp_name_phylo' for which there
## are no data.
\end{verbatim}

\begin{Shaded}
\begin{Highlighting}[]
\NormalTok{    MLMR_mods_pers_SSD_mammal_explore_lncvr}
\end{Highlighting}
\end{Shaded}

\begin{verbatim}
## 
## Multivariate Meta-Analysis Model (k = 213; method: REML)
## 
## Variance Components:
## 
##             estim    sqrt  nlvls  fixed          factor    R 
## sigma^2.1  0.0198  0.1406     19     no        study_ID   no 
## sigma^2.2  0.0265  0.1628     16     no  spp_name_phylo  yes 
## sigma^2.3  0.0323  0.1799    213     no             obs   no 
## 
## Test for Residual Heterogeneity:
## QE(df = 211) = 361.1620, p-val < .0001
## 
## Test of Moderators (coefficient 2):
## F(df1 = 1, df2 = 211) = 0.2658, p-val = 0.6067
## 
## Model Results:
## 
##            estimate      se     tval   df    pval    ci.lb   ci.ub 
## intrcpt     -0.0595  0.1507  -0.3951  211  0.6932  -0.3566  0.2375    
## SSD_index    0.1324  0.2567   0.5156  211  0.6067  -0.3737  0.6384    
## 
## ---
## Signif. codes:  0 '***' 0.001 '**' 0.01 '*' 0.05 '.' 0.1 ' ' 1
\end{verbatim}

\subsubsection{Birds}\label{birds}

SSD for boldness only.

\begin{Shaded}
\begin{Highlighting}[]
\CommentTok{# 2. BIRDS}

  \CommentTok{# subset dataset}
\NormalTok{  pers_new_bird <-}\StringTok{ }\KeywordTok{as.data.frame}\NormalTok{(pers_new }\OperatorTok
\StringTok{    }\KeywordTok{filter}\NormalTok{(personality_trait }\OperatorTok{==}\StringTok{ "boldness"} \OperatorTok{&}\StringTok{ }\NormalTok{taxo_group }\OperatorTok{==}\StringTok{ "bird"}\NormalTok{))}

  \CommentTok{# phylo_vcv birds only}
\NormalTok{  phylo_vcv_bird <-}\StringTok{ }\NormalTok{phylo_vcv[[}\DecValTok{1}\NormalTok{]]}
\end{Highlighting}
\end{Shaded}

Boldness:

\begin{Shaded}
\begin{Highlighting}[]
    \CommentTok{# SMD}
\NormalTok{    MLMR_mods_pers_SSD_bird_SMD <-}\StringTok{ }\KeywordTok{rma.mv}\NormalTok{(SMD_yi_flip }\OperatorTok{~}\StringTok{ }\NormalTok{SSD_index, }\DataTypeTok{V =}\NormalTok{ SMD_vi, }
                                          \DataTypeTok{random =} \KeywordTok{list}\NormalTok{(}\OperatorTok{~}\DecValTok{1}\OperatorTok{|}\NormalTok{study_ID, }\OperatorTok{~}\DecValTok{1}\OperatorTok{|}\NormalTok{spp_name_phylo, }\OperatorTok{~}\DecValTok{1}\OperatorTok{|}\NormalTok{obs), }
                                          \DataTypeTok{R =} \KeywordTok{list}\NormalTok{(}\DataTypeTok{spp_name_phylo=}\NormalTok{phylo_vcv_bird), }\DataTypeTok{control=}\KeywordTok{list}\NormalTok{(}\DataTypeTok{optimizer=}\StringTok{"optim"}\NormalTok{), }
                                          \DataTypeTok{test =} \StringTok{"t"}\NormalTok{, }\DataTypeTok{data =}\NormalTok{ pers_new_bird)}
\end{Highlighting}
\end{Shaded}

\begin{verbatim}
## Warning in rma.mv(SMD_yi_flip ~ SSD_index, V = SMD_vi, random = list(~1 | :
## There are rows/columns in the 'R' matrix for 'spp_name_phylo' for which there
## are no data.
\end{verbatim}

\begin{verbatim}
## Warning: Rows with NAs omitted from model fitting.
\end{verbatim}

\begin{Shaded}
\begin{Highlighting}[]
\NormalTok{    MLMR_mods_pers_SSD_bird_SMD}
\end{Highlighting}
\end{Shaded}

\begin{verbatim}
## 
## Multivariate Meta-Analysis Model (k = 234; method: REML)
## 
## Variance Components:
## 
##             estim    sqrt  nlvls  fixed          factor    R 
## sigma^2.1  1.9496  1.3963     21     no        study_ID   no 
## sigma^2.2  0.0001  0.0074     78     no  spp_name_phylo  yes 
## sigma^2.3  0.0925  0.3042    234     no             obs   no 
## 
## Test for Residual Heterogeneity:
## QE(df = 232) = 1579.6588, p-val < .0001
## 
## Test of Moderators (coefficient 2):
## F(df1 = 1, df2 = 232) = 0.1117, p-val = 0.7385
## 
## Model Results:
## 
##            estimate      se     tval   df    pval    ci.lb   ci.ub 
## intrcpt     -0.2211  0.3139  -0.7043  232  0.4819  -0.8397  0.3974    
## SSD_index   -0.2015  0.6028  -0.3342  232  0.7385  -1.3890  0.9861    
## 
## ---
## Signif. codes:  0 '***' 0.001 '**' 0.01 '*' 0.05 '.' 0.1 ' ' 1
\end{verbatim}

\begin{Shaded}
\begin{Highlighting}[]
    \CommentTok{# lnCVR}
\NormalTok{    MLMR_mods_pers_SSD_bird_lncvr <-}\StringTok{ }\KeywordTok{rma.mv}\NormalTok{(CVR_yi }\OperatorTok{~}\StringTok{ }\NormalTok{SSD_index, }\DataTypeTok{V =}\NormalTok{ CVR_vi, }
                                            \DataTypeTok{random =} \KeywordTok{list}\NormalTok{(}\OperatorTok{~}\DecValTok{1}\OperatorTok{|}\NormalTok{study_ID, }\OperatorTok{~}\DecValTok{1}\OperatorTok{|}\NormalTok{spp_name_phylo, }\OperatorTok{~}\DecValTok{1}\OperatorTok{|}\NormalTok{obs), }
                                            \DataTypeTok{R =} \KeywordTok{list}\NormalTok{(}\DataTypeTok{spp_name_phylo=}\NormalTok{phylo_vcv_bird), }\DataTypeTok{control=}\KeywordTok{list}\NormalTok{(}\DataTypeTok{optimizer=}\StringTok{"optim"}\NormalTok{), }
                                            \DataTypeTok{test =} \StringTok{"t"}\NormalTok{, }\DataTypeTok{data =}\NormalTok{ pers_new_bird)}
\end{Highlighting}
\end{Shaded}

\begin{verbatim}
## Warning in rma.mv(CVR_yi ~ SSD_index, V = CVR_vi, random = list(~1 | study_ID, :
## There are rows/columns in the 'R' matrix for 'spp_name_phylo' for which there
## are no data.

## Warning in rma.mv(CVR_yi ~ SSD_index, V = CVR_vi, random = list(~1 | study_ID, :
## Rows with NAs omitted from model fitting.
\end{verbatim}

\begin{Shaded}
\begin{Highlighting}[]
\NormalTok{    MLMR_mods_pers_SSD_bird_lncvr}
\end{Highlighting}
\end{Shaded}

\begin{verbatim}
## 
## Multivariate Meta-Analysis Model (k = 234; method: REML)
## 
## Variance Components:
## 
##             estim    sqrt  nlvls  fixed          factor    R 
## sigma^2.1  0.0000  0.0003     21     no        study_ID   no 
## sigma^2.2  0.0029  0.0537     78     no  spp_name_phylo  yes 
## sigma^2.3  0.0000  0.0003    234     no             obs   no 
## 
## Test for Residual Heterogeneity:
## QE(df = 232) = 244.9667, p-val = 0.2670
## 
## Test of Moderators (coefficient 2):
## F(df1 = 1, df2 = 232) = 0.6126, p-val = 0.4346
## 
## Model Results:
## 
##            estimate      se    tval   df    pval    ci.lb   ci.ub 
## intrcpt      0.0428  0.0361  1.1851  232  0.2372  -0.0283  0.1139    
## SSD_index    0.1023  0.1307  0.7827  232  0.4346  -0.1553  0.3599    
## 
## ---
## Signif. codes:  0 '***' 0.001 '**' 0.01 '*' 0.05 '.' 0.1 ' ' 1
\end{verbatim}

\subsubsection{Fish}\label{fish}

SSD for aggression and boldness.

\begin{Shaded}
\begin{Highlighting}[]
\CommentTok{# 3. FISH}
  
  \CommentTok{# subset by trait type}
  \CommentTok{# a. aggression}
\NormalTok{    pers_new_fish_aggression <-}\StringTok{ }\KeywordTok{as.data.frame}\NormalTok{(pers_new }\OperatorTok
\StringTok{    }\KeywordTok{filter}\NormalTok{(personality_trait }\OperatorTok{==}\StringTok{ "aggression"}\NormalTok{) }\OperatorTok
\StringTok{    }\KeywordTok{filter}\NormalTok{(taxo_group }\OperatorTok{==}\StringTok{ "fish"}\NormalTok{)) }
  \CommentTok{# b. boldness}
\NormalTok{    pers_new_fish_bold <-}\StringTok{ }\KeywordTok{as.data.frame}\NormalTok{(pers_new }\OperatorTok
\StringTok{    }\KeywordTok{filter}\NormalTok{(personality_trait }\OperatorTok{==}\StringTok{ "boldness"}\NormalTok{) }\OperatorTok
\StringTok{    }\KeywordTok{filter}\NormalTok{(taxo_group }\OperatorTok{==}\StringTok{ "fish"}\NormalTok{)) }
    
  \CommentTok{# phylo}
\NormalTok{    phylo_vcv_fish <-}\StringTok{ }\NormalTok{phylo_vcv[[}\DecValTok{2}\NormalTok{]]}
\end{Highlighting}
\end{Shaded}

Aggression:

\begin{Shaded}
\begin{Highlighting}[]
  \CommentTok{# a. aggression}
    \CommentTok{# SMD}
\NormalTok{      MLMR_mods_pers_SSD_fish_aggression_SMD <-}\StringTok{ }\KeywordTok{rma.mv}\NormalTok{(SMD_yi_flip }\OperatorTok{~}\StringTok{ }\NormalTok{SSD_index, }\DataTypeTok{V =}\NormalTok{ SMD_vi, }
                                          \DataTypeTok{random =} \KeywordTok{list}\NormalTok{(}\OperatorTok{~}\DecValTok{1}\OperatorTok{|}\NormalTok{study_ID, }\OperatorTok{~}\DecValTok{1}\OperatorTok{|}\NormalTok{spp_name_phylo, }\OperatorTok{~}\DecValTok{1}\OperatorTok{|}\NormalTok{obs), }
                                          \DataTypeTok{R =} \KeywordTok{list}\NormalTok{(}\DataTypeTok{spp_name_phylo=}\NormalTok{phylo_vcv_fish), }\DataTypeTok{control=}\KeywordTok{list}\NormalTok{(}\DataTypeTok{optimizer=}\StringTok{"optim"}\NormalTok{), }
                                          \DataTypeTok{test =} \StringTok{"t"}\NormalTok{, }\DataTypeTok{data =}\NormalTok{ pers_new_fish_aggression)}
\end{Highlighting}
\end{Shaded}

\begin{verbatim}
## Warning in rma.mv(SMD_yi_flip ~ SSD_index, V = SMD_vi, random = list(~1 | :
## There are rows/columns in the 'R' matrix for 'spp_name_phylo' for which there
## are no data.
\end{verbatim}

\begin{verbatim}
## Warning: Rows with NAs omitted from model fitting.
\end{verbatim}

\begin{Shaded}
\begin{Highlighting}[]
\NormalTok{    MLMR_mods_pers_SSD_fish_aggression_SMD}
\end{Highlighting}
\end{Shaded}

\begin{verbatim}
## 
## Multivariate Meta-Analysis Model (k = 93; method: REML)
## 
## Variance Components:
## 
##             estim    sqrt  nlvls  fixed          factor    R 
## sigma^2.1  0.0194  0.1395     16     no        study_ID   no 
## sigma^2.2  0.3329  0.5770     13     no  spp_name_phylo  yes 
## sigma^2.3  0.1704  0.4128     93     no             obs   no 
## 
## Test for Residual Heterogeneity:
## QE(df = 91) = 334.1728, p-val < .0001
## 
## Test of Moderators (coefficient 2):
## F(df1 = 1, df2 = 91) = 0.2301, p-val = 0.6326
## 
## Model Results:
## 
##            estimate      se     tval  df    pval    ci.lb   ci.ub 
## intrcpt     -0.1643  0.3987  -0.4120  91  0.6813  -0.9562  0.6277    
## SSD_index    0.2659  0.5544   0.4797  91  0.6326  -0.8352  1.3671    
## 
## ---
## Signif. codes:  0 '***' 0.001 '**' 0.01 '*' 0.05 '.' 0.1 ' ' 1
\end{verbatim}

\begin{Shaded}
\begin{Highlighting}[]
    \CommentTok{# lnCVR}
\NormalTok{      MLMR_mods_pers_SSD_fish_aggression_lncvr <-}\StringTok{ }\KeywordTok{rma.mv}\NormalTok{(CVR_yi }\OperatorTok{~}\StringTok{ }\NormalTok{SSD_index, }\DataTypeTok{V =}\NormalTok{ CVR_vi, }
                                            \DataTypeTok{random =} \KeywordTok{list}\NormalTok{(}\OperatorTok{~}\DecValTok{1}\OperatorTok{|}\NormalTok{study_ID, }\OperatorTok{~}\DecValTok{1}\OperatorTok{|}\NormalTok{spp_name_phylo, }\OperatorTok{~}\DecValTok{1}\OperatorTok{|}\NormalTok{obs), }
                                            \DataTypeTok{R =} \KeywordTok{list}\NormalTok{(}\DataTypeTok{spp_name_phylo=}\NormalTok{phylo_vcv_fish), }\DataTypeTok{control=}\KeywordTok{list}\NormalTok{(}\DataTypeTok{optimizer=}\StringTok{"optim"}\NormalTok{), }
                                            \DataTypeTok{test =} \StringTok{"t"}\NormalTok{, }\DataTypeTok{data =}\NormalTok{ pers_new_fish_aggression)}
\end{Highlighting}
\end{Shaded}

\begin{verbatim}
## Warning in rma.mv(CVR_yi ~ SSD_index, V = CVR_vi, random = list(~1 | study_ID, :
## There are rows/columns in the 'R' matrix for 'spp_name_phylo' for which there
## are no data.

## Warning in rma.mv(CVR_yi ~ SSD_index, V = CVR_vi, random = list(~1 | study_ID, :
## Rows with NAs omitted from model fitting.
\end{verbatim}

\begin{Shaded}
\begin{Highlighting}[]
\NormalTok{    MLMR_mods_pers_SSD_fish_aggression_lncvr}
\end{Highlighting}
\end{Shaded}

\begin{verbatim}
## 
## Multivariate Meta-Analysis Model (k = 93; method: REML)
## 
## Variance Components:
## 
##             estim    sqrt  nlvls  fixed          factor    R 
## sigma^2.1  0.0210  0.1450     16     no        study_ID   no 
## sigma^2.2  0.0000  0.0021     13     no  spp_name_phylo  yes 
## sigma^2.3  0.0000  0.0012     93     no             obs   no 
## 
## Test for Residual Heterogeneity:
## QE(df = 91) = 68.2701, p-val = 0.9640
## 
## Test of Moderators (coefficient 2):
## F(df1 = 1, df2 = 91) = 0.1495, p-val = 0.6999
## 
## Model Results:
## 
##            estimate      se     tval  df    pval    ci.lb   ci.ub 
## intrcpt     -0.1163  0.0597  -1.9483  91  0.0545  -0.2348  0.0023  . 
## SSD_index   -0.1323  0.3423  -0.3866  91  0.6999  -0.8122  0.5476    
## 
## ---
## Signif. codes:  0 '***' 0.001 '**' 0.01 '*' 0.05 '.' 0.1 ' ' 1
\end{verbatim}

Boldness:

\begin{Shaded}
\begin{Highlighting}[]
    \CommentTok{# SMD}
\NormalTok{      MLMR_mods_pers_SSD_fish_bold_SMD <-}\StringTok{ }\KeywordTok{rma.mv}\NormalTok{(SMD_yi_flip }\OperatorTok{~}\StringTok{ }\NormalTok{SSD_index, }\DataTypeTok{V =}\NormalTok{ SMD_vi, }
                                          \DataTypeTok{random =} \KeywordTok{list}\NormalTok{(}\OperatorTok{~}\DecValTok{1}\OperatorTok{|}\NormalTok{study_ID, }\OperatorTok{~}\DecValTok{1}\OperatorTok{|}\NormalTok{spp_name_phylo, }\OperatorTok{~}\DecValTok{1}\OperatorTok{|}\NormalTok{obs), }
                                          \DataTypeTok{R =} \KeywordTok{list}\NormalTok{(}\DataTypeTok{spp_name_phylo=}\NormalTok{phylo_vcv_fish), }\DataTypeTok{control=}\KeywordTok{list}\NormalTok{(}\DataTypeTok{optimizer=}\StringTok{"optim"}\NormalTok{), }
                                          \DataTypeTok{test =} \StringTok{"t"}\NormalTok{, }\DataTypeTok{data =}\NormalTok{ pers_new_fish_bold)}
\end{Highlighting}
\end{Shaded}

\begin{verbatim}
## Warning in rma.mv(SMD_yi_flip ~ SSD_index, V = SMD_vi, random = list(~1 | :
## There are rows/columns in the 'R' matrix for 'spp_name_phylo' for which there
## are no data.
\end{verbatim}

\begin{verbatim}
## Warning: Rows with NAs omitted from model fitting.
\end{verbatim}

\begin{Shaded}
\begin{Highlighting}[]
\NormalTok{    MLMR_mods_pers_SSD_fish_bold_SMD}
\end{Highlighting}
\end{Shaded}

\begin{verbatim}
## 
## Multivariate Meta-Analysis Model (k = 171; method: REML)
## 
## Variance Components:
## 
##             estim    sqrt  nlvls  fixed          factor    R 
## sigma^2.1  0.1717  0.4143     23     no        study_ID   no 
## sigma^2.2  0.0279  0.1671     12     no  spp_name_phylo  yes 
## sigma^2.3  0.1634  0.4042    171     no             obs   no 
## 
## Test for Residual Heterogeneity:
## QE(df = 169) = 614.1157, p-val < .0001
## 
## Test of Moderators (coefficient 2):
## F(df1 = 1, df2 = 169) = 0.3196, p-val = 0.5726
## 
## Model Results:
## 
##            estimate      se     tval   df    pval    ci.lb   ci.ub 
## intrcpt      0.1883  0.1764   1.0678  169  0.2871  -0.1599  0.5366    
## SSD_index   -0.2571  0.4548  -0.5653  169  0.5726  -1.1550  0.6408    
## 
## ---
## Signif. codes:  0 '***' 0.001 '**' 0.01 '*' 0.05 '.' 0.1 ' ' 1
\end{verbatim}

\begin{Shaded}
\begin{Highlighting}[]
    \CommentTok{# lnCVR}
\NormalTok{      MLMR_mods_pers_SSD_fish_bold_lncvr <-}\StringTok{ }\KeywordTok{rma.mv}\NormalTok{(CVR_yi }\OperatorTok{~}\StringTok{ }\NormalTok{SSD_index, }\DataTypeTok{V =}\NormalTok{ CVR_vi, }
                                            \DataTypeTok{random =} \KeywordTok{list}\NormalTok{(}\OperatorTok{~}\DecValTok{1}\OperatorTok{|}\NormalTok{study_ID, }\OperatorTok{~}\DecValTok{1}\OperatorTok{|}\NormalTok{spp_name_phylo, }\OperatorTok{~}\DecValTok{1}\OperatorTok{|}\NormalTok{obs), }
                                            \DataTypeTok{R =} \KeywordTok{list}\NormalTok{(}\DataTypeTok{spp_name_phylo=}\NormalTok{phylo_vcv_fish), }\DataTypeTok{control=}\KeywordTok{list}\NormalTok{(}\DataTypeTok{optimizer=}\StringTok{"optim"}\NormalTok{), }
                                            \DataTypeTok{test =} \StringTok{"t"}\NormalTok{, }\DataTypeTok{data =}\NormalTok{ pers_new_fish_bold)  }
\end{Highlighting}
\end{Shaded}

\begin{verbatim}
## Warning in rma.mv(CVR_yi ~ SSD_index, V = CVR_vi, random = list(~1 | study_ID, :
## There are rows/columns in the 'R' matrix for 'spp_name_phylo' for which there
## are no data.

## Warning in rma.mv(CVR_yi ~ SSD_index, V = CVR_vi, random = list(~1 | study_ID, :
## Rows with NAs omitted from model fitting.
\end{verbatim}

\begin{Shaded}
\begin{Highlighting}[]
\NormalTok{    MLMR_mods_pers_SSD_fish_bold_lncvr}
\end{Highlighting}
\end{Shaded}

\begin{verbatim}
## 
## Multivariate Meta-Analysis Model (k = 171; method: REML)
## 
## Variance Components:
## 
##             estim    sqrt  nlvls  fixed          factor    R 
## sigma^2.1  0.0445  0.2109     23     no        study_ID   no 
## sigma^2.2  0.0184  0.1356     12     no  spp_name_phylo  yes 
## sigma^2.3  0.0881  0.2968    171     no             obs   no 
## 
## Test for Residual Heterogeneity:
## QE(df = 169) = 395.3375, p-val < .0001
## 
## Test of Moderators (coefficient 2):
## F(df1 = 1, df2 = 169) = 0.3068, p-val = 0.5804
## 
## Model Results:
## 
##            estimate      se     tval   df    pval    ci.lb   ci.ub 
## intrcpt      0.0063  0.1310   0.0482  169  0.9616  -0.2523  0.2649    
## SSD_index   -0.1507  0.2721  -0.5539  169  0.5804  -0.6878  0.3864    
## 
## ---
## Signif. codes:  0 '***' 0.001 '**' 0.01 '*' 0.05 '.' 0.1 ' ' 1
\end{verbatim}

\subsubsection{Inverts}\label{inverts}

SSD for activity and boldness.

\begin{Shaded}
\begin{Highlighting}[]
\CommentTok{# 4. INVERTS}
  \CommentTok{# subset dataset}
  \CommentTok{# a. activity}
\NormalTok{    invert_activity <-}\StringTok{ }\KeywordTok{as.data.frame}\NormalTok{(pers_new }\OperatorTok
\StringTok{    }\KeywordTok{filter}\NormalTok{(personality_trait }\OperatorTok{==}\StringTok{ "activity"}\NormalTok{) }\OperatorTok
\StringTok{    }\KeywordTok{filter}\NormalTok{(taxo_group }\OperatorTok{==}\StringTok{ "invertebrate"}\NormalTok{)) }
  \CommentTok{# b. boldness}
\NormalTok{    invert_bold <-}\StringTok{ }\KeywordTok{as.data.frame}\NormalTok{(pers_new }\OperatorTok
\StringTok{    }\KeywordTok{filter}\NormalTok{(personality_trait }\OperatorTok{==}\StringTok{ "boldness"}\NormalTok{) }\OperatorTok
\StringTok{    }\KeywordTok{filter}\NormalTok{(taxo_group }\OperatorTok{==}\StringTok{ "invertebrate"}\NormalTok{))}
  
  \CommentTok{# phylo}
\NormalTok{    phylo_vcv_invert <-}\StringTok{ }\NormalTok{phylo_vcv[[}\DecValTok{3}\NormalTok{]]}
\end{Highlighting}
\end{Shaded}

Activity:

\begin{Shaded}
\begin{Highlighting}[]
  \CommentTok{# rerun models}
    \CommentTok{# a. activity}
    \CommentTok{# SMD}
\NormalTok{      MLMR_mods_pers_SSD_invert_activity_SMD <-}\StringTok{ }\KeywordTok{rma.mv}\NormalTok{(SMD_yi_flip }\OperatorTok{~}\StringTok{ }\NormalTok{SSD_index, }\DataTypeTok{V =}\NormalTok{ SMD_vi, }
                                          \DataTypeTok{random =} \KeywordTok{list}\NormalTok{(}\OperatorTok{~}\DecValTok{1}\OperatorTok{|}\NormalTok{study_ID, }\OperatorTok{~}\DecValTok{1}\OperatorTok{|}\NormalTok{spp_name_phylo, }\OperatorTok{~}\DecValTok{1}\OperatorTok{|}\NormalTok{obs), }
                                          \DataTypeTok{R =} \KeywordTok{list}\NormalTok{(}\DataTypeTok{spp_name_phylo=}\NormalTok{phylo_vcv_invert), }\DataTypeTok{control=}\KeywordTok{list}\NormalTok{(}\DataTypeTok{optimizer=}\StringTok{"optim"}\NormalTok{), }
                                          \DataTypeTok{test =} \StringTok{"t"}\NormalTok{, }\DataTypeTok{data =}\NormalTok{ invert_activity)}
\end{Highlighting}
\end{Shaded}

\begin{verbatim}
## Warning in rma.mv(SMD_yi_flip ~ SSD_index, V = SMD_vi, random = list(~1 | :
## There are rows/columns in the 'R' matrix for 'spp_name_phylo' for which there
## are no data.
\end{verbatim}

\begin{verbatim}
## Warning: Rows with NAs omitted from model fitting.
\end{verbatim}

\begin{Shaded}
\begin{Highlighting}[]
\NormalTok{    MLMR_mods_pers_SSD_invert_activity_SMD}
\end{Highlighting}
\end{Shaded}

\begin{verbatim}
## 
## Multivariate Meta-Analysis Model (k = 165; method: REML)
## 
## Variance Components:
## 
##             estim    sqrt  nlvls  fixed          factor    R 
## sigma^2.1  2.1874  1.4790     18     no        study_ID   no 
## sigma^2.2  0.0001  0.0104     16     no  spp_name_phylo  yes 
## sigma^2.3  0.1562  0.3952    165     no             obs   no 
## 
## Test for Residual Heterogeneity:
## QE(df = 163) = 1081.7241, p-val < .0001
## 
## Test of Moderators (coefficient 2):
## F(df1 = 1, df2 = 163) = 0.7140, p-val = 0.3993
## 
## Model Results:
## 
##            estimate      se     tval   df    pval    ci.lb   ci.ub 
## intrcpt      0.3479  0.3670   0.9480  163  0.3445  -0.3767  1.0725    
## SSD_index   -0.6862  0.8120  -0.8450  163  0.3993  -2.2896  0.9173    
## 
## ---
## Signif. codes:  0 '***' 0.001 '**' 0.01 '*' 0.05 '.' 0.1 ' ' 1
\end{verbatim}

\begin{Shaded}
\begin{Highlighting}[]
    \CommentTok{# lnCVR}
\NormalTok{      MLMR_mods_pers_SSD_invert_activity_lncvr <-}\StringTok{ }\KeywordTok{rma.mv}\NormalTok{(CVR_yi }\OperatorTok{~}\StringTok{ }\NormalTok{SSD_index, }\DataTypeTok{V =}\NormalTok{ CVR_vi, }
                                            \DataTypeTok{random =} \KeywordTok{list}\NormalTok{(}\OperatorTok{~}\DecValTok{1}\OperatorTok{|}\NormalTok{study_ID, }\OperatorTok{~}\DecValTok{1}\OperatorTok{|}\NormalTok{spp_name_phylo, }\OperatorTok{~}\DecValTok{1}\OperatorTok{|}\NormalTok{obs), }
                                            \DataTypeTok{R =} \KeywordTok{list}\NormalTok{(}\DataTypeTok{spp_name_phylo=}\NormalTok{phylo_vcv_invert), }\DataTypeTok{control=}\KeywordTok{list}\NormalTok{(}\DataTypeTok{optimizer=}\StringTok{"optim"}\NormalTok{), }
                                            \DataTypeTok{test =} \StringTok{"t"}\NormalTok{, }\DataTypeTok{data =}\NormalTok{ invert_activity) }
\end{Highlighting}
\end{Shaded}

\begin{verbatim}
## Warning in rma.mv(CVR_yi ~ SSD_index, V = CVR_vi, random = list(~1 | study_ID, :
## There are rows/columns in the 'R' matrix for 'spp_name_phylo' for which there
## are no data.

## Warning in rma.mv(CVR_yi ~ SSD_index, V = CVR_vi, random = list(~1 | study_ID, :
## Rows with NAs omitted from model fitting.
\end{verbatim}

\begin{Shaded}
\begin{Highlighting}[]
\NormalTok{      MLMR_mods_pers_SSD_invert_activity_lncvr}
\end{Highlighting}
\end{Shaded}

\begin{verbatim}
## 
## Multivariate Meta-Analysis Model (k = 165; method: REML)
## 
## Variance Components:
## 
##             estim    sqrt  nlvls  fixed          factor    R 
## sigma^2.1  0.1193  0.3454     18     no        study_ID   no 
## sigma^2.2  0.0000  0.0035     16     no  spp_name_phylo  yes 
## sigma^2.3  0.0591  0.2431    165     no             obs   no 
## 
## Test for Residual Heterogeneity:
## QE(df = 163) = 486.7410, p-val < .0001
## 
## Test of Moderators (coefficient 2):
## F(df1 = 1, df2 = 163) = 0.4392, p-val = 0.5085
## 
## Model Results:
## 
##            estimate      se     tval   df    pval    ci.lb   ci.ub 
## intrcpt     -0.0278  0.1074  -0.2589  163  0.7960  -0.2400  0.1843    
## SSD_index    0.2685  0.4052   0.6627  163  0.5085  -0.5315  1.0685    
## 
## ---
## Signif. codes:  0 '***' 0.001 '**' 0.01 '*' 0.05 '.' 0.1 ' ' 1
\end{verbatim}

Boldness:

\begin{Shaded}
\begin{Highlighting}[]
   \CommentTok{# SMD}
\NormalTok{      MLMR_mods_pers_SSD_invert_bold_SMD <-}\StringTok{ }\KeywordTok{rma.mv}\NormalTok{(SMD_yi_flip }\OperatorTok{~}\StringTok{ }\NormalTok{SSD_index, }\DataTypeTok{V =}\NormalTok{ SMD_vi, }
                                          \DataTypeTok{random =} \KeywordTok{list}\NormalTok{(}\OperatorTok{~}\DecValTok{1}\OperatorTok{|}\NormalTok{study_ID, }\OperatorTok{~}\DecValTok{1}\OperatorTok{|}\NormalTok{spp_name_phylo, }\OperatorTok{~}\DecValTok{1}\OperatorTok{|}\NormalTok{obs), }
                                          \DataTypeTok{R =} \KeywordTok{list}\NormalTok{(}\DataTypeTok{spp_name_phylo=}\NormalTok{phylo_vcv_invert), }\DataTypeTok{control=}\KeywordTok{list}\NormalTok{(}\DataTypeTok{optimizer=}\StringTok{"optim"}\NormalTok{), }
                                          \DataTypeTok{test =} \StringTok{"t"}\NormalTok{, }\DataTypeTok{data =}\NormalTok{ invert_bold)}
\end{Highlighting}
\end{Shaded}

\begin{verbatim}
## Warning in rma.mv(SMD_yi_flip ~ SSD_index, V = SMD_vi, random = list(~1 | :
## There are rows/columns in the 'R' matrix for 'spp_name_phylo' for which there
## are no data.
\end{verbatim}

\begin{Shaded}
\begin{Highlighting}[]
\NormalTok{    MLMR_mods_pers_SSD_invert_bold_SMD}
\end{Highlighting}
\end{Shaded}

\begin{verbatim}
## 
## Multivariate Meta-Analysis Model (k = 164; method: REML)
## 
## Variance Components:
## 
##             estim    sqrt  nlvls  fixed          factor    R 
## sigma^2.1  0.0822  0.2867     23     no        study_ID   no 
## sigma^2.2  0.0000  0.0020     23     no  spp_name_phylo  yes 
## sigma^2.3  0.0650  0.2550    164     no             obs   no 
## 
## Test for Residual Heterogeneity:
## QE(df = 162) = 513.4222, p-val < .0001
## 
## Test of Moderators (coefficient 2):
## F(df1 = 1, df2 = 162) = 0.1533, p-val = 0.6959
## 
## Model Results:
## 
##            estimate      se    tval   df    pval    ci.lb   ci.ub 
## intrcpt      0.0985  0.0823  1.1967  162  0.2332  -0.0640  0.2611    
## SSD_index    0.1313  0.3354  0.3915  162  0.6959  -0.5310  0.7936    
## 
## ---
## Signif. codes:  0 '***' 0.001 '**' 0.01 '*' 0.05 '.' 0.1 ' ' 1
\end{verbatim}

\begin{Shaded}
\begin{Highlighting}[]
    \CommentTok{# lnCVR}
\NormalTok{      MLMR_mods_pers_SSD_invert_bold_lncvr <-}\StringTok{ }\KeywordTok{rma.mv}\NormalTok{(CVR_yi }\OperatorTok{~}\StringTok{ }\NormalTok{SSD_index, }\DataTypeTok{V =}\NormalTok{ CVR_vi, }
                                            \DataTypeTok{random =} \KeywordTok{list}\NormalTok{(}\OperatorTok{~}\DecValTok{1}\OperatorTok{|}\NormalTok{study_ID, }\OperatorTok{~}\DecValTok{1}\OperatorTok{|}\NormalTok{spp_name_phylo, }\OperatorTok{~}\DecValTok{1}\OperatorTok{|}\NormalTok{obs), }
                                            \DataTypeTok{R =} \KeywordTok{list}\NormalTok{(}\DataTypeTok{spp_name_phylo=}\NormalTok{phylo_vcv_invert), }\DataTypeTok{control=}\KeywordTok{list}\NormalTok{(}\DataTypeTok{optimizer=}\StringTok{"optim"}\NormalTok{), }
                                            \DataTypeTok{test =} \StringTok{"t"}\NormalTok{, }\DataTypeTok{data =}\NormalTok{ invert_bold)  }
\end{Highlighting}
\end{Shaded}

\begin{verbatim}
## Warning in rma.mv(CVR_yi ~ SSD_index, V = CVR_vi, random = list(~1 | study_ID, :
## There are rows/columns in the 'R' matrix for 'spp_name_phylo' for which there
## are no data.
\end{verbatim}

\begin{Shaded}
\begin{Highlighting}[]
\NormalTok{    MLMR_mods_pers_SSD_invert_bold_lncvr}
\end{Highlighting}
\end{Shaded}

\begin{verbatim}
## 
## Multivariate Meta-Analysis Model (k = 164; method: REML)
## 
## Variance Components:
## 
##             estim    sqrt  nlvls  fixed          factor    R 
## sigma^2.1  0.0383  0.1957     23     no        study_ID   no 
## sigma^2.2  0.0000  0.0015     23     no  spp_name_phylo  yes 
## sigma^2.3  0.0378  0.1945    164     no             obs   no 
## 
## Test for Residual Heterogeneity:
## QE(df = 162) = 380.7049, p-val < .0001
## 
## Test of Moderators (coefficient 2):
## F(df1 = 1, df2 = 162) = 0.0013, p-val = 0.9716
## 
## Model Results:
## 
##            estimate      se     tval   df    pval    ci.lb   ci.ub 
## intrcpt     -0.0145  0.0607  -0.2384  162  0.8119  -0.1344  0.1055    
## SSD_index   -0.0089  0.2503  -0.0356  162  0.9716  -0.5032  0.4854    
## 
## ---
## Signif. codes:  0 '***' 0.001 '**' 0.01 '*' 0.05 '.' 0.1 ' ' 1
\end{verbatim}

\subsection{Multiple testing}\label{multiple-testing}

Now we can extract the p-values from our intercept models, personality
trait models, and SSD subset models to adjust p-values using the false
discovery rate method. This method uses the p.adjust function to adjust
p-values to account for multiple testing.

\begin{Shaded}
\begin{Highlighting}[]
\NormalTok{## Extract p-values from SSD subset models (reported in main text)}

\CommentTok{# list}
\NormalTok{  p.SMD_SSD <-}\StringTok{ }\KeywordTok{c}\NormalTok{(}\FloatTok{0.69}\NormalTok{, }\FloatTok{0.03}\NormalTok{, }\FloatTok{0.86}\NormalTok{, }\FloatTok{0.04}\NormalTok{, }\FloatTok{0.34}\NormalTok{, }\FloatTok{0.25}\NormalTok{, }\FloatTok{0.99}\NormalTok{, }\FloatTok{0.85}\NormalTok{, }\FloatTok{0.48}\NormalTok{, }\FloatTok{0.74}\NormalTok{, }\FloatTok{0.68}\NormalTok{, }\FloatTok{0.63}\NormalTok{, }\FloatTok{0.29}\NormalTok{, }\FloatTok{0.57}\NormalTok{, }\FloatTok{0.34}\NormalTok{, }\FloatTok{0.40}\NormalTok{, }\FloatTok{0.23}\NormalTok{, }\FloatTok{0.70}\NormalTok{)}
\NormalTok{  p.lnCVR_SSD <-}\StringTok{ }\KeywordTok{c}\NormalTok{(}\FloatTok{0.60}\NormalTok{, }\FloatTok{0.72}\NormalTok{, }\FloatTok{0.52}\NormalTok{, }\FloatTok{0.91}\NormalTok{, }\FloatTok{0.82}\NormalTok{, }\FloatTok{0.15}\NormalTok{, }\FloatTok{0.69}\NormalTok{, }\FloatTok{0.61}\NormalTok{, }\FloatTok{0.24}\NormalTok{, }\FloatTok{0.43}\NormalTok{, }\FloatTok{0.05}\NormalTok{, }\FloatTok{0.70}\NormalTok{, }\FloatTok{0.96}\NormalTok{, }\FloatTok{0.58}\NormalTok{, }\FloatTok{0.80}\NormalTok{, }\FloatTok{0.51}\NormalTok{, }\FloatTok{0.81}\NormalTok{, }\FloatTok{0.97}\NormalTok{)}
  
  \CommentTok{# p adjustment on our 3 hypothesis-testing models}
  
    \CommentTok{#SMD}
  \KeywordTok{p.adjust}\NormalTok{(}\DataTypeTok{p =} \KeywordTok{c}\NormalTok{(p.SMD_intercept, p.SMD_pers, p.SMD_SSD), }\DataTypeTok{method =} \StringTok{"fdr"}\NormalTok{) }
\end{Highlighting}
\end{Shaded}

\begin{verbatim}
##          bird          fish  invertebrate        mammal      reptilia 
##     0.8533333     0.8533333     0.8533333     0.8708500     0.8533333 
##         bird1         bird2         bird3         bird4         bird5 
##     0.8708500     0.8533333     0.8533333     0.8708500     0.4800000 
##         fish1         fish2         fish3         fish4         fish5 
##     0.8708500     0.8533333     0.8708500     0.8533333     0.8708500 
## invertebrate1 invertebrate2 invertebrate3 invertebrate4 invertebrate5 
##     0.8533333     0.8533333     0.8533333     0.9195744     0.8533333 
##       mammal1       mammal2       mammal3       mammal4       mammal5 
##     0.8533333     0.8708500     0.8533333     0.9292872     0.8708500 
##     reptilia1     reptilia2     reptilia3     reptilia4     reptilia5 
##     0.9195744     0.8533333     0.8533333     0.4800000     0.9900000 
##                                                                       
##     0.8708500     0.4800000     0.9195744     0.4800000     0.8533333 
##                                                                       
##     0.8533333     0.9900000     0.9195744     0.8533333     0.8708500 
##                                                                       
##     0.8708500     0.8708500     0.8533333     0.8708500     0.8533333 
##                                           
##     0.8533333     0.8533333     0.8708500
\end{verbatim}

\begin{Shaded}
\begin{Highlighting}[]
    \CommentTok{#lncvr}
  \KeywordTok{p.adjust}\NormalTok{(}\DataTypeTok{p =} \KeywordTok{c}\NormalTok{(p.lnCVR_intercept, p.lnCVR_pers, p.lnCVR_SSD), }\DataTypeTok{method =} \StringTok{"fdr"}\NormalTok{) }
\end{Highlighting}
\end{Shaded}

\begin{verbatim}
##          bird          fish  invertebrate        mammal      reptilia 
##     0.9513857     0.9513857     0.9513857     0.9513857     0.9513857 
##         bird1         bird2         bird3         bird4         bird5 
##     0.9513857     0.9513857     0.9513857     0.4948879     0.9513857 
##         fish1         fish2         fish3         fish4         fish5 
##     0.9513857     0.8505594     0.9513857     0.9513857     0.4948879 
## invertebrate1 invertebrate2 invertebrate3 invertebrate4 invertebrate5 
##     0.9513857     0.9513857     0.9513857     0.9513857     0.9513857 
##       mammal1       mammal2       mammal3       mammal4       mammal5 
##     0.9513857     0.9513857     0.9513857     0.9513857     0.9513857 
##     reptilia1     reptilia2     reptilia3     reptilia4     reptilia5 
##     0.9513857     0.7414563     0.9513857     0.9513857     0.9513857 
##                                                                       
##     0.9513857     0.9513857     0.9513857     0.9513857     0.9513857 
##                                                                       
##     0.9513857     0.9513857     0.9513857     0.9513857     0.9513857 
##                                                                       
##     0.7414563     0.9513857     0.9700000     0.9513857     0.9513857 
##                                           
##     0.9513857     0.9513857     0.9700000
\end{verbatim}

\begin{Shaded}
\begin{Highlighting}[]
  \CommentTok{# these p-values are in the order presented in tables, so easy to replace old p-values with new ones}
\end{Highlighting}
\end{Shaded}

\section{Exploratory analyses}\label{exploratory-analyses}

We collected some additional information from the literature (mating
system) and from studies that we expected would influence sex
differences. These analyses are strictly exploratory and just compare
categorical moderator terms.

\subsection{mating system}\label{mating-system}

Do effect sizes from monogamous or multiply-mating species differ? Model
summaries presented in Supplementary Table S3.

\begin{Shaded}
\begin{Highlighting}[]
\CommentTok{# what have we got to work with?}
\NormalTok{    pers_new }\OperatorTok
\StringTok{    }\KeywordTok{group_by}\NormalTok{(taxo_group, mating_system) }\OperatorTok
\StringTok{    }\KeywordTok{filter}\NormalTok{(}\OperatorTok{!}\KeywordTok{is.na}\NormalTok{(mating_system))}\OperatorTok
\StringTok{    }\KeywordTok{summarise}\NormalTok{(}\DataTypeTok{n =} \KeywordTok{n}\NormalTok{(), }\DataTypeTok{studies =} \KeywordTok{length}\NormalTok{(}\KeywordTok{unique}\NormalTok{(study_ID)), }\DataTypeTok{species =} \KeywordTok{length}\NormalTok{(}\KeywordTok{unique}\NormalTok{(spp_name_phylo))) }\CommentTok{# make a table of numbers}
\end{Highlighting}
\end{Shaded}

\begin{verbatim}
## `summarise()` has grouped output by 'taxo_group'. You can override using the `.groups` argument.
\end{verbatim}

\begin{verbatim}
## # A tibble: 10 x 5
## # Groups:   taxo_group [5]
##    taxo_group   mating_system       n studies species
##    <chr>        <chr>           <int>   <int>   <int>
##  1 bird         monogamy          370      43      92
##  2 bird         multiple mating   107       9      12
##  3 fish         monogamy           65       8       5
##  4 fish         multiple mating   411      34      15
##  5 invertebrate monogamy           22       3       3
##  6 invertebrate multiple mating   369      35      29
##  7 mammal       monogamy          105       9       8
##  8 mammal       multiple mating   517      52      33
##  9 reptilia     monogamy            2       1       1
## 10 reptilia     multiple mating    53       7       6
\end{verbatim}

\begin{Shaded}
\begin{Highlighting}[]
\CommentTok{# reload model output}
\NormalTok{rerun_models }\OperatorTok{==}\StringTok{ }\OtherTok{FALSE}
\end{Highlighting}
\end{Shaded}

\begin{verbatim}
## [1] TRUE
\end{verbatim}

\begin{Shaded}
\begin{Highlighting}[]
    \ControlFlowTok{if}\NormalTok{(rerun_models }\OperatorTok{==}\StringTok{ }\OtherTok{TRUE}\NormalTok{)\{}
\NormalTok{      MLMR_models_pers_mating_system <-}\StringTok{ }\KeywordTok{meta_model_fits}\NormalTok{(pers_new, phylo_vcv, }\DataTypeTok{type =} \StringTok{"pers_mate"}\NormalTok{)}
      \KeywordTok{saveRDS}\NormalTok{(MLMR_models_pers_mating_system, }\StringTok{"./output/MLMR_models_pers_mating_system"}\NormalTok{)}
\NormalTok{    \} }\ControlFlowTok{else}\NormalTok{\{}
\NormalTok{     MLMR_models_pers_mating_system <-}\StringTok{ }\KeywordTok{readRDS}\NormalTok{(}\StringTok{"./output/MLMR_models_pers_mating_system"}\NormalTok{)}
\NormalTok{    \}}

\CommentTok{# Extract the SMD and lnCVR results}
\NormalTok{    smd_mods_mating_system <-}\StringTok{ }\NormalTok{MLMR_models_pers_mating_system[}\StringTok{"SMD"}\NormalTok{,]}
      
\NormalTok{    lnCVR_mods_mating_system <-}\StringTok{ }\NormalTok{MLMR_models_pers_mating_system[}\StringTok{"lnCVR"}\NormalTok{,]}
\end{Highlighting}
\end{Shaded}

\subsection{age}\label{age}

Do effect sizes from adults (sexually mature) or juveniles differ? Model
summaries presented in Supplementary Table S4

\begin{Shaded}
\begin{Highlighting}[]
\CommentTok{# make a table}
  \KeywordTok{data.frame}\NormalTok{(pers_new }\OperatorTok
\StringTok{  }\KeywordTok{group_by}\NormalTok{(taxo_group, age) }\OperatorTok
\StringTok{  }\KeywordTok{summarise}\NormalTok{(}\DataTypeTok{n=} \KeywordTok{n}\NormalTok{(), }\DataTypeTok{N_spp =} \KeywordTok{length}\NormalTok{(}\KeywordTok{unique}\NormalTok{(spp_name_phylo)), }\DataTypeTok{N_studies =} \KeywordTok{length}\NormalTok{(}\KeywordTok{unique}\NormalTok{(study_ID))))}
\end{Highlighting}
\end{Shaded}

\begin{verbatim}
## `summarise()` has grouped output by 'taxo_group'. You can override using the `.groups` argument.
\end{verbatim}

\begin{verbatim}
##      taxo_group      age   n N_spp N_studies
## 1          bird    adult 323   105        43
## 2          bird juvenile 157    10        13
## 3          fish    adult 483    22        43
## 4          fish juvenile   7     3         3
## 5  invertebrate    adult 384    36        37
## 6  invertebrate juvenile  39     3         3
## 7        mammal    adult 470    38        48
## 8        mammal juvenile 204    18        19
## 9      reptilia    adult  93     9        10
## 10     reptilia juvenile   2     1         1
\end{verbatim}

\begin{Shaded}
\begin{Highlighting}[]
\CommentTok{# reload model output:}
\NormalTok{rerun_models }\OperatorTok{==}\StringTok{ }\OtherTok{FALSE}
\end{Highlighting}
\end{Shaded}

\begin{verbatim}
## [1] TRUE
\end{verbatim}

\begin{Shaded}
\begin{Highlighting}[]
  \ControlFlowTok{if}\NormalTok{(rerun_models }\OperatorTok{==}\StringTok{ }\OtherTok{TRUE}\NormalTok{)\{}
\NormalTok{      MLMR_models_pers_age <-}\StringTok{ }\KeywordTok{meta_model_fits}\NormalTok{(pers_new, phylo_vcv, }\DataTypeTok{type =} \StringTok{"age"}\NormalTok{)}
      \KeywordTok{saveRDS}\NormalTok{(MLMR_models_pers_age, }\StringTok{"./output/MLMR_models_pers_age"}\NormalTok{)}
\NormalTok{    \} }\ControlFlowTok{else}\NormalTok{\{}
\NormalTok{     MLMR_models_pers_age <-}\StringTok{ }\KeywordTok{readRDS}\NormalTok{(}\StringTok{"./output/MLMR_models_pers_age"}\NormalTok{)}
\NormalTok{    \}}


\CommentTok{# Extract the SMD and lnCVR results}
\NormalTok{  smd_mods_pers_age <-}\StringTok{ }\NormalTok{MLMR_models_pers_age[}\StringTok{"SMD"}\NormalTok{,]}
    
\NormalTok{  lnCVR_mods_pers_age <-}\StringTok{ }\NormalTok{MLMR_models_pers_age[}\StringTok{"lnCVR"}\NormalTok{,]}
\end{Highlighting}
\end{Shaded}

\subsection{population}\label{population}

Do effect sizes from wild animals or lab animals differ? Model summaries
presented in Supplementary Table S5.

\begin{Shaded}
\begin{Highlighting}[]
\CommentTok{# table}
  \KeywordTok{data.frame}\NormalTok{(pers_new }\OperatorTok
\StringTok{  }\KeywordTok{group_by}\NormalTok{(taxo_group, population) }\OperatorTok
\StringTok{  }\KeywordTok{summarise}\NormalTok{(}\DataTypeTok{n =} \KeywordTok{n}\NormalTok{(), }\DataTypeTok{N_spp =} \KeywordTok{length}\NormalTok{(}\KeywordTok{unique}\NormalTok{(spp_name_phylo)), }\DataTypeTok{N_studies =} \KeywordTok{length}\NormalTok{(}\KeywordTok{unique}\NormalTok{(study_ID))))}
\end{Highlighting}
\end{Shaded}

\begin{verbatim}
## `summarise()` has grouped output by 'taxo_group'. You can override using the `.groups` argument.
\end{verbatim}

\begin{verbatim}
##      taxo_group population   n N_spp N_studies
## 1          bird      field 263   100        34
## 2          bird        lab 217     9        16
## 3          fish      field 189    13        17
## 4          fish        lab 301    12        28
## 5  invertebrate      field 176    24        21
## 6  invertebrate        lab 247    13        17
## 7        mammal      field 181    23        26
## 8        mammal        lab 493    26        38
## 9      reptilia      field  81     9        10
## 10     reptilia        lab  14     2         2
\end{verbatim}

\begin{Shaded}
\begin{Highlighting}[]
\CommentTok{# reload model output:}
\NormalTok{rerun_models }\OperatorTok{==}\StringTok{ }\OtherTok{FALSE}
\end{Highlighting}
\end{Shaded}

\begin{verbatim}
## [1] TRUE
\end{verbatim}

\begin{Shaded}
\begin{Highlighting}[]
  \ControlFlowTok{if}\NormalTok{(rerun_models }\OperatorTok{==}\StringTok{ }\OtherTok{TRUE}\NormalTok{)\{}
\NormalTok{      MLMR_models_pers_pop <-}\StringTok{ }\KeywordTok{meta_model_fits}\NormalTok{(pers_new, phylo_vcv, }\DataTypeTok{type =} \StringTok{"pop"}\NormalTok{)}
      \KeywordTok{saveRDS}\NormalTok{(MLMR_models_pers_pop, }\StringTok{"./output/MLMR_models_pers_pop"}\NormalTok{)}
\NormalTok{    \} }\ControlFlowTok{else}\NormalTok{\{}
\NormalTok{     MLMR_models_pers_pop <-}\StringTok{ }\KeywordTok{readRDS}\NormalTok{(}\StringTok{"./output/MLMR_models_pers_pop"}\NormalTok{)}
\NormalTok{    \}}

\CommentTok{# Extract the SMD and lnCVR results}
\NormalTok{  smd_mods_pers_pop <-}\StringTok{ }\NormalTok{MLMR_models_pers_pop[}\StringTok{"SMD"}\NormalTok{,]}
    
\NormalTok{  lnCVR_mods_pers_pop <-}\StringTok{ }\NormalTok{MLMR_models_pers_pop[}\StringTok{"lnCVR"}\NormalTok{,]}
\end{Highlighting}
\end{Shaded}

\subsection{study environment}\label{study-environment}

Do effect sizes collected in the wild or the lab differ? Model summaries
presented in Supplementary Table S6.

\begin{Shaded}
\begin{Highlighting}[]
\CommentTok{# table}
  \KeywordTok{data.frame}\NormalTok{(pers_new }\OperatorTok
\StringTok{  }\KeywordTok{group_by}\NormalTok{(taxo_group, study_environment) }\OperatorTok
\StringTok{  }\KeywordTok{summarise}\NormalTok{(}\DataTypeTok{n =} \KeywordTok{n}\NormalTok{(), }\DataTypeTok{N_spp =} \KeywordTok{length}\NormalTok{(}\KeywordTok{unique}\NormalTok{(spp_name_phylo)), }\DataTypeTok{N_studies =} \KeywordTok{length}\NormalTok{(}\KeywordTok{unique}\NormalTok{(study_ID))))}
\end{Highlighting}
\end{Shaded}

\begin{verbatim}
## `summarise()` has grouped output by 'taxo_group'. You can override using the `.groups` argument.
\end{verbatim}

\begin{verbatim}
##      taxo_group study_environment   n N_spp N_studies
## 1          bird             field 224   100        29
## 2          bird               lab 256    11        22
## 3          fish             field  68     5         5
## 4          fish               lab 422    17        39
## 5  invertebrate             field  14     4         1
## 6  invertebrate               lab 409    35        38
## 7        mammal             field 115    18        22
## 8        mammal               lab 559    30        40
## 9      reptilia             field   5     2         2
## 10     reptilia               lab  90     8         9
\end{verbatim}

\begin{Shaded}
\begin{Highlighting}[]
\CommentTok{# reload model output:}
\NormalTok{rerun_models }\OperatorTok{==}\StringTok{ }\OtherTok{FALSE}
\end{Highlighting}
\end{Shaded}

\begin{verbatim}
## [1] TRUE
\end{verbatim}

\begin{Shaded}
\begin{Highlighting}[]
  \ControlFlowTok{if}\NormalTok{(rerun_models }\OperatorTok{==}\StringTok{ }\OtherTok{TRUE}\NormalTok{)\{}
\NormalTok{      MLMR_models_pers_environ <-}\StringTok{ }\KeywordTok{meta_model_fits}\NormalTok{(pers_new, phylo_vcv, }\DataTypeTok{type =} \StringTok{"environ"}\NormalTok{)}
      \KeywordTok{saveRDS}\NormalTok{(MLMR_models_pers_environ, }\StringTok{"./output/MLMR_models_pers_environ"}\NormalTok{)}
\NormalTok{    \} }\ControlFlowTok{else}\NormalTok{\{}
\NormalTok{     MLMR_models_pers_environ <-}\StringTok{ }\KeywordTok{readRDS}\NormalTok{(}\StringTok{"./output/MLMR_models_pers_environ"}\NormalTok{)}
\NormalTok{    \}}

\CommentTok{# Extract the SMD and lnCVR results}
\NormalTok{  smd_mods_pers_enviro <-}\StringTok{ }\NormalTok{MLMR_models_pers_environ[}\StringTok{"SMD"}\NormalTok{,]}
      
\NormalTok{  lnCVR_mods_pers_enviro <-}\StringTok{ }\NormalTok{MLMR_models_pers_environ[}\StringTok{"lnCVR"}\NormalTok{,]}
\end{Highlighting}
\end{Shaded}

\subsection{study type}\label{study-type}

Do effect sizes from observational or experimental study design differ?
Model summaries presented in Supplementary Table S7.

\begin{Shaded}
\begin{Highlighting}[]
\CommentTok{# Let's see what we have to work with}
  \KeywordTok{data.frame}\NormalTok{(pers_new }\OperatorTok
\StringTok{  }\KeywordTok{group_by}\NormalTok{(taxo_group, study_type) }\OperatorTok
\StringTok{  }\KeywordTok{summarise}\NormalTok{(}\DataTypeTok{N_spp =} \KeywordTok{length}\NormalTok{(}\KeywordTok{unique}\NormalTok{(spp_name_phylo)), }\DataTypeTok{N_studies =} \KeywordTok{length}\NormalTok{(}\KeywordTok{unique}\NormalTok{(study_ID))))}
\end{Highlighting}
\end{Shaded}

\begin{verbatim}
## `summarise()` has grouped output by 'taxo_group'. You can override using the `.groups` argument.
\end{verbatim}

\begin{verbatim}
##     taxo_group   study_type N_spp N_studies
## 1         bird experimental    17        32
## 2         bird  observation    94        18
## 3         fish experimental    19        41
## 4         fish  observation     3         3
## 5 invertebrate experimental    37        38
## 6       mammal experimental    37        47
## 7       mammal  observation    12        14
## 8     reptilia experimental     8         9
## 9     reptilia  observation     2         2
\end{verbatim}

\begin{Shaded}
\begin{Highlighting}[]
    \CommentTok{# inverts only have experimental observations, so need to exclude inverts from this analysis}
    \CommentTok{# because our phylo_vcv matrix is in a list that is hard to drop elements from, let's just run each model individually }

\CommentTok{# 1. Mammals}
  \CommentTok{# Subset data}
\NormalTok{    pers_new_mammal <-}\StringTok{ }\KeywordTok{as.data.frame}\NormalTok{(pers_new }\OperatorTok
\StringTok{    }\KeywordTok{filter}\NormalTok{(taxo_group }\OperatorTok{==}\StringTok{ "mammal"}\NormalTok{))}

  \CommentTok{# Run models with just study type as moderator:}
    \CommentTok{#SMD}
\NormalTok{    MLMR_mods_pers_studytype_mammal_SMD <-}\StringTok{ }\KeywordTok{rma.mv}\NormalTok{(SMD_yi_flip }\OperatorTok{~}\StringTok{ }\NormalTok{study_type, }\DataTypeTok{V =}\NormalTok{ SMD_vi, }
                                                   \DataTypeTok{random =} \KeywordTok{list}\NormalTok{(}\OperatorTok{~}\DecValTok{1}\OperatorTok{|}\NormalTok{study_ID, }\OperatorTok{~}\DecValTok{1}\OperatorTok{|}\NormalTok{spp_name_phylo, }\OperatorTok{~}\DecValTok{1}\OperatorTok{|}\NormalTok{obs), }
                                                   \DataTypeTok{R =} \KeywordTok{list}\NormalTok{(}\DataTypeTok{spp_name_phylo=}\NormalTok{phylo_vcv_mammal), }\DataTypeTok{control=}\KeywordTok{list}\NormalTok{(}\DataTypeTok{optimizer=}\StringTok{"optim"}\NormalTok{), }
                                                   \DataTypeTok{test =} \StringTok{"t"}\NormalTok{, }\DataTypeTok{data =}\NormalTok{ pers_new_mammal)}

\NormalTok{    MLMR_mods_pers_studytype_mammal_SMD}
\end{Highlighting}
\end{Shaded}

\begin{verbatim}
## 
## Multivariate Meta-Analysis Model (k = 674; method: REML)
## 
## Variance Components:
## 
##             estim    sqrt  nlvls  fixed          factor    R 
## sigma^2.1  0.1051  0.3242     61     no        study_ID   no 
## sigma^2.2  0.0094  0.0971     45     no  spp_name_phylo  yes 
## sigma^2.3  0.1570  0.3963    674     no             obs   no 
## 
## Test for Residual Heterogeneity:
## QE(df = 672) = 2198.5222, p-val < .0001
## 
## Test of Moderators (coefficient 2):
## F(df1 = 1, df2 = 672) = 9.6851, p-val = 0.0019
## 
## Model Results:
## 
##                        estimate      se     tval   df    pval    ci.lb   ci.ub 
## intrcpt                 -0.0067  0.0913  -0.0734  672  0.9415  -0.1860  0.1726 
## study_typeobservation    0.4092  0.1315   3.1121  672  0.0019   0.1510  0.6674 
##  
## intrcpt 
## study_typeobservation  ** 
## 
## ---
## Signif. codes:  0 '***' 0.001 '**' 0.01 '*' 0.05 '.' 0.1 ' ' 1
\end{verbatim}

\begin{Shaded}
\begin{Highlighting}[]
    \CommentTok{#lnCVR}
\NormalTok{    MLMR_mods_pers_studytype_mammal_lncvr <-}\StringTok{ }\KeywordTok{rma.mv}\NormalTok{(CVR_yi }\OperatorTok{~}\StringTok{ }\NormalTok{study_type, }\DataTypeTok{V =}\NormalTok{ CVR_vi, }
                                                   \DataTypeTok{random =} \KeywordTok{list}\NormalTok{(}\OperatorTok{~}\DecValTok{1}\OperatorTok{|}\NormalTok{study_ID, }\OperatorTok{~}\DecValTok{1}\OperatorTok{|}\NormalTok{spp_name_phylo, }\OperatorTok{~}\DecValTok{1}\OperatorTok{|}\NormalTok{obs), }
                                                   \DataTypeTok{R =} \KeywordTok{list}\NormalTok{(}\DataTypeTok{spp_name_phylo=}\NormalTok{phylo_vcv_mammal), }\DataTypeTok{control=}\KeywordTok{list}\NormalTok{(}\DataTypeTok{optimizer=}\StringTok{"optim"}\NormalTok{), }
                                                   \DataTypeTok{test =} \StringTok{"t"}\NormalTok{, }\DataTypeTok{data =}\NormalTok{ pers_new_mammal)}
    
\NormalTok{    MLMR_mods_pers_studytype_mammal_lncvr}
\end{Highlighting}
\end{Shaded}

\begin{verbatim}
## 
## Multivariate Meta-Analysis Model (k = 674; method: REML)
## 
## Variance Components:
## 
##             estim    sqrt  nlvls  fixed          factor    R 
## sigma^2.1  0.0356  0.1888     61     no        study_ID   no 
## sigma^2.2  0.0436  0.2088     45     no  spp_name_phylo  yes 
## sigma^2.3  0.0338  0.1837    674     no             obs   no 
## 
## Test for Residual Heterogeneity:
## QE(df = 672) = 1061.6294, p-val < .0001
## 
## Test of Moderators (coefficient 2):
## F(df1 = 1, df2 = 672) = 0.5661, p-val = 0.4521
## 
## Model Results:
## 
##                        estimate      se    tval   df    pval    ci.lb   ci.ub 
## intrcpt                  0.0298  0.1361  0.2188  672  0.8269  -0.2375  0.2970 
## study_typeobservation    0.0771  0.1024  0.7524  672  0.4521  -0.1240  0.2781 
##  
## intrcpt 
## study_typeobservation 
## 
## ---
## Signif. codes:  0 '***' 0.001 '**' 0.01 '*' 0.05 '.' 0.1 ' ' 1
\end{verbatim}

\begin{Shaded}
\begin{Highlighting}[]
\CommentTok{# 2. BIRDS}
  \CommentTok{# subset dataset}
\NormalTok{      pers_new_bird <-}\StringTok{ }\KeywordTok{as.data.frame}\NormalTok{(pers_new }\OperatorTok
\StringTok{      }\KeywordTok{filter}\NormalTok{(taxo_group }\OperatorTok{==}\StringTok{ "bird"}\NormalTok{))}

  \CommentTok{# rerun models}
      \CommentTok{#SMD}
\NormalTok{       MLMR_mods_pers_studytype_bird_SMD <-}\StringTok{ }\KeywordTok{rma.mv}\NormalTok{(SMD_yi_flip }\OperatorTok{~}\StringTok{ }\NormalTok{study_type, }\DataTypeTok{V =}\NormalTok{ SMD_vi, }
                                                   \DataTypeTok{random =} \KeywordTok{list}\NormalTok{(}\OperatorTok{~}\DecValTok{1}\OperatorTok{|}\NormalTok{study_ID, }\OperatorTok{~}\DecValTok{1}\OperatorTok{|}\NormalTok{spp_name_phylo, }\OperatorTok{~}\DecValTok{1}\OperatorTok{|}\NormalTok{obs), }
                                                   \DataTypeTok{R =} \KeywordTok{list}\NormalTok{(}\DataTypeTok{spp_name_phylo=}\NormalTok{phylo_vcv_bird), }\DataTypeTok{control=}\KeywordTok{list}\NormalTok{(}\DataTypeTok{optimizer=}\StringTok{"optim"}\NormalTok{), }
                                                   \DataTypeTok{test =} \StringTok{"t"}\NormalTok{, }\DataTypeTok{data =}\NormalTok{ pers_new_bird)}

\NormalTok{       MLMR_mods_pers_studytype_bird_SMD}
\end{Highlighting}
\end{Shaded}

\begin{verbatim}
## 
## Multivariate Meta-Analysis Model (k = 480; method: REML)
## 
## Variance Components:
## 
##             estim    sqrt  nlvls  fixed          factor    R 
## sigma^2.1  0.6650  0.8155     50     no        study_ID   no 
## sigma^2.2  0.0000  0.0038    106     no  spp_name_phylo  yes 
## sigma^2.3  0.1203  0.3469    480     no             obs   no 
## 
## Test for Residual Heterogeneity:
## QE(df = 478) = 2378.8387, p-val < .0001
## 
## Test of Moderators (coefficient 2):
## F(df1 = 1, df2 = 478) = 1.0068, p-val = 0.3162
## 
## Model Results:
## 
##                        estimate      se     tval   df    pval    ci.lb   ci.ub 
## intrcpt                 -0.0210  0.1515  -0.1389  478  0.8896  -0.3187  0.2766 
## study_typeobservation   -0.2540  0.2532  -1.0034  478  0.3162  -0.7515  0.2434 
##  
## intrcpt 
## study_typeobservation 
## 
## ---
## Signif. codes:  0 '***' 0.001 '**' 0.01 '*' 0.05 '.' 0.1 ' ' 1
\end{verbatim}

\begin{Shaded}
\begin{Highlighting}[]
      \CommentTok{#lnCVR}
\NormalTok{      MLMR_mods_pers_studytype_bird_lncvr <-}\StringTok{ }\KeywordTok{rma.mv}\NormalTok{(CVR_yi }\OperatorTok{~}\StringTok{ }\NormalTok{study_type, }\DataTypeTok{V =}\NormalTok{ CVR_vi, }
                                                   \DataTypeTok{random =} \KeywordTok{list}\NormalTok{(}\OperatorTok{~}\DecValTok{1}\OperatorTok{|}\NormalTok{study_ID, }\OperatorTok{~}\DecValTok{1}\OperatorTok{|}\NormalTok{spp_name_phylo, }\OperatorTok{~}\DecValTok{1}\OperatorTok{|}\NormalTok{obs), }
                                                   \DataTypeTok{R =} \KeywordTok{list}\NormalTok{(}\DataTypeTok{spp_name_phylo=}\NormalTok{phylo_vcv_bird), }\DataTypeTok{control=}\KeywordTok{list}\NormalTok{(}\DataTypeTok{optimizer=}\StringTok{"optim"}\NormalTok{), }
                                                   \DataTypeTok{test =} \StringTok{"t"}\NormalTok{, }\DataTypeTok{data =}\NormalTok{ pers_new_bird)}

\NormalTok{      MLMR_mods_pers_studytype_bird_lncvr}
\end{Highlighting}
\end{Shaded}

\begin{verbatim}
## 
## Multivariate Meta-Analysis Model (k = 480; method: REML)
## 
## Variance Components:
## 
##             estim    sqrt  nlvls  fixed          factor    R 
## sigma^2.1  0.2537  0.5036     50     no        study_ID   no 
## sigma^2.2  0.0001  0.0086    106     no  spp_name_phylo  yes 
## sigma^2.3  0.3766  0.6137    480     no             obs   no 
## 
## Test for Residual Heterogeneity:
## QE(df = 478) = 3186.9869, p-val < .0001
## 
## Test of Moderators (coefficient 2):
## F(df1 = 1, df2 = 478) = 1.7076, p-val = 0.1919
## 
## Model Results:
## 
##                        estimate      se     tval   df    pval    ci.lb   ci.ub 
## intrcpt                  0.0436  0.1109   0.3929  478  0.6946  -0.1744  0.2615 
## study_typeobservation   -0.2451  0.1876  -1.3067  478  0.1919  -0.6138  0.1235 
##  
## intrcpt 
## study_typeobservation 
## 
## ---
## Signif. codes:  0 '***' 0.001 '**' 0.01 '*' 0.05 '.' 0.1 ' ' 1
\end{verbatim}

\begin{Shaded}
\begin{Highlighting}[]
\CommentTok{# 3. FISH}
  \CommentTok{# subset dataset}
\NormalTok{      pers_new_fish <-}\StringTok{ }\KeywordTok{as.data.frame}\NormalTok{(pers_new }\OperatorTok
\StringTok{      }\KeywordTok{filter}\NormalTok{(taxo_group }\OperatorTok{==}\StringTok{ "fish"}\NormalTok{)) }

  \CommentTok{# rerun models}
      \CommentTok{#SMD}
\NormalTok{      MLMR_mods_pers_studytype_fish_SMD <-}\StringTok{ }\KeywordTok{rma.mv}\NormalTok{(SMD_yi_flip }\OperatorTok{~}\StringTok{ }\NormalTok{study_type, }\DataTypeTok{V =}\NormalTok{ SMD_vi, }
                                          \DataTypeTok{random =} \KeywordTok{list}\NormalTok{(}\OperatorTok{~}\DecValTok{1}\OperatorTok{|}\NormalTok{study_ID, }\OperatorTok{~}\DecValTok{1}\OperatorTok{|}\NormalTok{spp_name_phylo, }\OperatorTok{~}\DecValTok{1}\OperatorTok{|}\NormalTok{obs), }
                                          \DataTypeTok{R =} \KeywordTok{list}\NormalTok{(}\DataTypeTok{spp_name_phylo=}\NormalTok{phylo_vcv_fish), }\DataTypeTok{control=}\KeywordTok{list}\NormalTok{(}\DataTypeTok{optimizer=}\StringTok{"optim"}\NormalTok{), }
                                          \DataTypeTok{test =} \StringTok{"t"}\NormalTok{, }\DataTypeTok{data =}\NormalTok{ pers_new_fish)}

\NormalTok{      MLMR_mods_pers_studytype_fish_SMD}
\end{Highlighting}
\end{Shaded}

\begin{verbatim}
## 
## Multivariate Meta-Analysis Model (k = 490; method: REML)
## 
## Variance Components:
## 
##             estim    sqrt  nlvls  fixed          factor    R 
## sigma^2.1  0.5997  0.7744     44     no        study_ID   no 
## sigma^2.2  0.0440  0.2098     22     no  spp_name_phylo  yes 
## sigma^2.3  0.1091  0.3303    490     no             obs   no 
## 
## Test for Residual Heterogeneity:
## QE(df = 488) = 1523.5807, p-val < .0001
## 
## Test of Moderators (coefficient 2):
## F(df1 = 1, df2 = 488) = 0.0188, p-val = 0.8911
## 
## Model Results:
## 
##                        estimate      se     tval   df    pval    ci.lb   ci.ub 
## intrcpt                  0.1825  0.2106   0.8669  488  0.3864  -0.2312  0.5962 
## study_typeobservation   -0.0657  0.4798  -0.1369  488  0.8911  -1.0085  0.8771 
##  
## intrcpt 
## study_typeobservation 
## 
## ---
## Signif. codes:  0 '***' 0.001 '**' 0.01 '*' 0.05 '.' 0.1 ' ' 1
\end{verbatim}

\begin{Shaded}
\begin{Highlighting}[]
      \CommentTok{#lnCVR}
\NormalTok{      MLMR_mods_pers_studytype_fish_lncvr <-}\StringTok{ }\KeywordTok{rma.mv}\NormalTok{(CVR_yi }\OperatorTok{~}\StringTok{ }\NormalTok{study_type, }\DataTypeTok{V =}\NormalTok{ CVR_vi, }
                                            \DataTypeTok{random =} \KeywordTok{list}\NormalTok{(}\OperatorTok{~}\DecValTok{1}\OperatorTok{|}\NormalTok{study_ID, }\OperatorTok{~}\DecValTok{1}\OperatorTok{|}\NormalTok{spp_name_phylo, }\OperatorTok{~}\DecValTok{1}\OperatorTok{|}\NormalTok{obs), }
                                            \DataTypeTok{R =} \KeywordTok{list}\NormalTok{(}\DataTypeTok{spp_name_phylo=}\NormalTok{phylo_vcv_fish), }\DataTypeTok{control=}\KeywordTok{list}\NormalTok{(}\DataTypeTok{optimizer=}\StringTok{"optim"}\NormalTok{), }
                                            \DataTypeTok{test =} \StringTok{"t"}\NormalTok{, }\DataTypeTok{data =}\NormalTok{ pers_new_fish)}

\NormalTok{      MLMR_mods_pers_studytype_fish_lncvr}
\end{Highlighting}
\end{Shaded}

\begin{verbatim}
## 
## Multivariate Meta-Analysis Model (k = 490; method: REML)
## 
## Variance Components:
## 
##             estim    sqrt  nlvls  fixed          factor    R 
## sigma^2.1  0.0352  0.1875     44     no        study_ID   no 
## sigma^2.2  0.0017  0.0408     22     no  spp_name_phylo  yes 
## sigma^2.3  0.1094  0.3307    490     no             obs   no 
## 
## Test for Residual Heterogeneity:
## QE(df = 488) = 1122.6615, p-val < .0001
## 
## Test of Moderators (coefficient 2):
## F(df1 = 1, df2 = 488) = 0.0722, p-val = 0.7882
## 
## Model Results:
## 
##                        estimate      se     tval   df    pval    ci.lb   ci.ub 
## intrcpt                 -0.0025  0.0536  -0.0460  488  0.9633  -0.1078  0.1029 
## study_typeobservation   -0.0399  0.1486  -0.2687  488  0.7882  -0.3319  0.2520 
##  
## intrcpt 
## study_typeobservation 
## 
## ---
## Signif. codes:  0 '***' 0.001 '**' 0.01 '*' 0.05 '.' 0.1 ' ' 1
\end{verbatim}

\begin{Shaded}
\begin{Highlighting}[]
\CommentTok{# 4. Reptiles}
  \CommentTok{# subset dataset}
\NormalTok{      pers_new_reptile <-}\StringTok{ }\KeywordTok{as.data.frame}\NormalTok{(pers_new }\OperatorTok
\StringTok{      }\KeywordTok{filter}\NormalTok{(taxo_group }\OperatorTok{==}\StringTok{ "reptilia"}\NormalTok{)) }
 
  \CommentTok{# phylo}
\NormalTok{      phylo_vcv_reptile <-}\StringTok{ }\NormalTok{phylo_vcv[[}\DecValTok{5}\NormalTok{]]}
  
  \CommentTok{# rerun models}
      \CommentTok{#SMD}
\NormalTok{      MLMR_mods_pers_studytype_rep_SMD <-}\StringTok{ }\KeywordTok{rma.mv}\NormalTok{(SMD_yi_flip }\OperatorTok{~}\StringTok{ }\NormalTok{study_type, }\DataTypeTok{V =}\NormalTok{ SMD_vi, }
                                                \DataTypeTok{random =} \KeywordTok{list}\NormalTok{(}\OperatorTok{~}\DecValTok{1}\OperatorTok{|}\NormalTok{study_ID, }\OperatorTok{~}\DecValTok{1}\OperatorTok{|}\NormalTok{spp_name_phylo, }\OperatorTok{~}\DecValTok{1}\OperatorTok{|}\NormalTok{obs), }
                                          \DataTypeTok{R =} \KeywordTok{list}\NormalTok{(}\DataTypeTok{spp_name_phylo=}\NormalTok{phylo_vcv_reptile), }\DataTypeTok{control=}\KeywordTok{list}\NormalTok{(}\DataTypeTok{optimizer=}\StringTok{"optim"}\NormalTok{), }
                                          \DataTypeTok{test =} \StringTok{"t"}\NormalTok{, }\DataTypeTok{data =}\NormalTok{ pers_new_reptile)}

\NormalTok{      MLMR_mods_pers_studytype_rep_SMD      }
\end{Highlighting}
\end{Shaded}

\begin{verbatim}
## 
## Multivariate Meta-Analysis Model (k = 95; method: REML)
## 
## Variance Components:
## 
##             estim    sqrt  nlvls  fixed          factor    R 
## sigma^2.1  0.0000  0.0008     11     no        study_ID   no 
## sigma^2.2  0.0730  0.2702     10     no  spp_name_phylo  yes 
## sigma^2.3  0.0426  0.2063     95     no             obs   no 
## 
## Test for Residual Heterogeneity:
## QE(df = 93) = 159.2776, p-val < .0001
## 
## Test of Moderators (coefficient 2):
## F(df1 = 1, df2 = 93) = 3.4462, p-val = 0.0666
## 
## Model Results:
## 
##                        estimate      se     tval  df    pval    ci.lb   ci.ub 
## intrcpt                  0.1334  0.1531   0.8715  93  0.3857  -0.1706  0.4374 
## study_typeobservation   -0.5085  0.2739  -1.8564  93  0.0666  -1.0524  0.0354 
##  
## intrcpt 
## study_typeobservation  . 
## 
## ---
## Signif. codes:  0 '***' 0.001 '**' 0.01 '*' 0.05 '.' 0.1 ' ' 1
\end{verbatim}

\begin{Shaded}
\begin{Highlighting}[]
      \CommentTok{#lnCVR}
\NormalTok{      MLMR_mods_pers_studytype_rep_lncvr <-}\StringTok{ }\KeywordTok{rma.mv}\NormalTok{(CVR_yi }\OperatorTok{~}\StringTok{ }\NormalTok{study_type, }\DataTypeTok{V =}\NormalTok{ CVR_vi, }
                                            \DataTypeTok{random =} \KeywordTok{list}\NormalTok{(}\OperatorTok{~}\DecValTok{1}\OperatorTok{|}\NormalTok{study_ID, }\OperatorTok{~}\DecValTok{1}\OperatorTok{|}\NormalTok{spp_name_phylo, }\OperatorTok{~}\DecValTok{1}\OperatorTok{|}\NormalTok{obs), }
                                            \DataTypeTok{R =} \KeywordTok{list}\NormalTok{(}\DataTypeTok{spp_name_phylo=}\NormalTok{phylo_vcv_reptile), }\DataTypeTok{control=}\KeywordTok{list}\NormalTok{(}\DataTypeTok{optimizer=}\StringTok{"optim"}\NormalTok{), }
                                            \DataTypeTok{test =} \StringTok{"t"}\NormalTok{, }\DataTypeTok{data =}\NormalTok{ pers_new_reptile)}
      
\NormalTok{      MLMR_mods_pers_studytype_rep_lncvr}
\end{Highlighting}
\end{Shaded}

\begin{verbatim}
## 
## Multivariate Meta-Analysis Model (k = 95; method: REML)
## 
## Variance Components:
## 
##             estim    sqrt  nlvls  fixed          factor    R 
## sigma^2.1  0.0000  0.0003     11     no        study_ID   no 
## sigma^2.2  0.0000  0.0012     10     no  spp_name_phylo  yes 
## sigma^2.3  0.0000  0.0003     95     no             obs   no 
## 
## Test for Residual Heterogeneity:
## QE(df = 93) = 58.0505, p-val = 0.9983
## 
## Test of Moderators (coefficient 2):
## F(df1 = 1, df2 = 93) = 0.1931, p-val = 0.6613
## 
## Model Results:
## 
##                        estimate      se     tval  df    pval    ci.lb   ci.ub 
## intrcpt                  0.0427  0.0420   1.0173  93  0.3116  -0.0407  0.1261 
## study_typeobservation   -0.0621  0.1413  -0.4395  93  0.6613  -0.3426  0.2184 
##  
## intrcpt 
## study_typeobservation 
## 
## ---
## Signif. codes:  0 '***' 0.001 '**' 0.01 '*' 0.05 '.' 0.1 ' ' 1
\end{verbatim}

\section{Sensitivity analyses - Dependency matrix
models}\label{sensitivity-analyses---dependency-matrix-models}

We need to refit our 3 main models accounting for any dependency
resulting from the same traits measured on the same animals (likely a
big source of non-independence) and any other shared covariance. We
added the D matrices to the residual variance matrix as opposed to the
sampling covariance. We chose to set 3 different levels of dependency
(rho): 0.3, 0.5 an 0.8.

Model summaries are also presented in Supplementary Tables S8-S13.

\begin{Shaded}
\begin{Highlighting}[]
 \CommentTok{# Create the dependency matrices; try 3 levels of rho = 0.3, 0.5, 0.8}
      
\NormalTok{      pers_new <-}\StringTok{ }\KeywordTok{data.frame}\NormalTok{(pers_new }\OperatorTok
\StringTok{      }\KeywordTok{group_by}\NormalTok{(taxo_group) }\OperatorTok
\StringTok{      }\KeywordTok{mutate}\NormalTok{(}\DataTypeTok{depend_n =} \KeywordTok{paste0}\NormalTok{(study_ID, }\StringTok{"_"}\NormalTok{, depend)))}
    
\NormalTok{      split_taxa <-}\StringTok{ }\KeywordTok{split}\NormalTok{(pers_new, pers_new}\OperatorTok{$}\NormalTok{taxo_group)}
    
  \CommentTok{# 0.3 rho:}
\NormalTok{    D_matrices_}\FloatTok{0.3}\NormalTok{ <-}\StringTok{ }\KeywordTok{lapply}\NormalTok{(split_taxa, }\ControlFlowTok{function}\NormalTok{(x) }\KeywordTok{make_VCV_matrix}\NormalTok{(x, }\DataTypeTok{V =}\NormalTok{ x}\OperatorTok{$}\NormalTok{SMD_vi, }\DataTypeTok{cluster =} \StringTok{"depend_n"}\NormalTok{, }
                                                                     \DataTypeTok{obs =} \StringTok{"obs"}\NormalTok{, }\DataTypeTok{type =} \StringTok{"cor"}\NormalTok{, }\DataTypeTok{rho =} \FloatTok{0.3}\NormalTok{))}
  \CommentTok{# 0.5 rho:}
\NormalTok{    D_matrices_}\FloatTok{0.5}\NormalTok{ <-}\StringTok{ }\KeywordTok{lapply}\NormalTok{(split_taxa, }\ControlFlowTok{function}\NormalTok{(x) }\KeywordTok{make_VCV_matrix}\NormalTok{(x, }\DataTypeTok{V =}\NormalTok{ x}\OperatorTok{$}\NormalTok{SMD_vi, }\DataTypeTok{cluster =} \StringTok{"depend_n"}\NormalTok{, }
                                                                     \DataTypeTok{obs =} \StringTok{"obs"}\NormalTok{, }\DataTypeTok{type =} \StringTok{"cor"}\NormalTok{, }\DataTypeTok{rho =} \FloatTok{0.5}\NormalTok{))}
  \CommentTok{# 0.8 rho:   }
\NormalTok{    D_matrices_}\FloatTok{0.8}\NormalTok{ <-}\StringTok{ }\KeywordTok{lapply}\NormalTok{(split_taxa, }\ControlFlowTok{function}\NormalTok{(x) }\KeywordTok{make_VCV_matrix}\NormalTok{(x, }\DataTypeTok{V =}\NormalTok{ x}\OperatorTok{$}\NormalTok{SMD_vi, }\DataTypeTok{cluster =} \StringTok{"depend_n"}\NormalTok{, }
                                                                     \DataTypeTok{obs =} \StringTok{"obs"}\NormalTok{, }\DataTypeTok{type =} \StringTok{"cor"}\NormalTok{, }\DataTypeTok{rho =} \FloatTok{0.8}\NormalTok{))}
\end{Highlighting}
\end{Shaded}

\subsection{Intercept-only models with D
matrices}\label{intercept-only-models-with-d-matrices}

Model output is presented in Supplementary Tables S8-S10 in the
Supporting Information.

\begin{Shaded}
\begin{Highlighting}[]
\CommentTok{# 1. Intercept only models}
    \CommentTok{# rho = 0.3}
\NormalTok{    int_}\FloatTok{0.3}\NormalTok{ <-}\StringTok{ }\KeywordTok{fit_int_MLMAmodel_D}\NormalTok{(pers_new, phylo_vcv, D_matrices_}\FloatTok{0.3}\NormalTok{)}
    
\NormalTok{    smd_mods_D_}\FloatTok{0.3}\NormalTok{ <-}\StringTok{ }\NormalTok{int_}\FloatTok{0.3}\NormalTok{[[}\StringTok{"SMD"}\NormalTok{]] }
\NormalTok{    lnCVR_mods_D_}\FloatTok{0.3}\NormalTok{ <-}\StringTok{ }\NormalTok{int_}\FloatTok{0.3}\NormalTok{[[}\StringTok{"lnCVR"}\NormalTok{]] }
  
   \CommentTok{# prediction intervals}
\NormalTok{    MLMA_estimates_SMD_D_}\FloatTok{0.3}\NormalTok{ <-}\StringTok{ }\NormalTok{plyr}\OperatorTok{::}\KeywordTok{ldply}\NormalTok{(}\KeywordTok{lapply}\NormalTok{(smd_mods_D_}\FloatTok{0.3}\NormalTok{, }
                           \ControlFlowTok{function}\NormalTok{(x) }\KeywordTok{print}\NormalTok{(}\KeywordTok{mod_results}\NormalTok{(x, }\DataTypeTok{mod =} \StringTok{"Int"}\NormalTok{)))) }
\NormalTok{    MLMA_estimates_lnCVR_D_}\FloatTok{0.3}\NormalTok{ <-}\StringTok{ }\NormalTok{plyr}\OperatorTok{::}\KeywordTok{ldply}\NormalTok{(}\KeywordTok{lapply}\NormalTok{(lnCVR_mods_D_}\FloatTok{0.3}\NormalTok{, }
                              \ControlFlowTok{function}\NormalTok{(x) }\KeywordTok{print}\NormalTok{(}\KeywordTok{mod_results}\NormalTok{(x, }\DataTypeTok{mod =} \StringTok{"Int"}\NormalTok{))))   }

    \CommentTok{# rho = 0.5}
\NormalTok{    int_}\FloatTok{0.5}\NormalTok{ <-}\StringTok{ }\KeywordTok{fit_int_MLMAmodel_D}\NormalTok{(pers_new, phylo_vcv, D_matrices_}\FloatTok{0.5}\NormalTok{)}
    
\NormalTok{    smd_mods_D_}\FloatTok{0.5}\NormalTok{ <-}\StringTok{ }\NormalTok{int_}\FloatTok{0.5}\NormalTok{[[}\StringTok{"SMD"}\NormalTok{]] }
\NormalTok{    lnCVR_mods_D_}\FloatTok{0.5}\NormalTok{ <-}\StringTok{ }\NormalTok{int_}\FloatTok{0.5}\NormalTok{[[}\StringTok{"lnCVR"}\NormalTok{]] }
  
    \CommentTok{# prediction intervals}
\NormalTok{    MLMA_estimates_SMD_D_}\FloatTok{0.5}\NormalTok{ <-}\StringTok{ }\NormalTok{plyr}\OperatorTok{::}\KeywordTok{ldply}\NormalTok{(}\KeywordTok{lapply}\NormalTok{(smd_mods_D_}\FloatTok{0.5}\NormalTok{, }
                             \ControlFlowTok{function}\NormalTok{(x) }\KeywordTok{print}\NormalTok{(}\KeywordTok{mod_results}\NormalTok{(x, }\DataTypeTok{mod =} \StringTok{"Int"}\NormalTok{)))) }
\NormalTok{    MLMA_estimates_lnCVR_D_}\FloatTok{0.5}\NormalTok{ <-}\StringTok{ }\NormalTok{plyr}\OperatorTok{::}\KeywordTok{ldply}\NormalTok{(}\KeywordTok{lapply}\NormalTok{(lnCVR_mods_D_}\FloatTok{0.5}\NormalTok{, }
                              \ControlFlowTok{function}\NormalTok{(x) }\KeywordTok{print}\NormalTok{(}\KeywordTok{mod_results}\NormalTok{(x, }\DataTypeTok{mod =} \StringTok{"Int"}\NormalTok{))))}
    
    \CommentTok{# rho = 0.8}
\NormalTok{    int_}\FloatTok{0.8}\NormalTok{ <-}\StringTok{ }\KeywordTok{fit_int_MLMAmodel_D}\NormalTok{(pers_new, phylo_vcv, D_matrices_}\FloatTok{0.8}\NormalTok{) }
    
\NormalTok{    smd_mods_D_}\FloatTok{0.8}\NormalTok{ <-}\StringTok{ }\NormalTok{int_}\FloatTok{0.8}\NormalTok{[[}\StringTok{"SMD"}\NormalTok{]] }
\NormalTok{    lnCVR_mods_D_}\FloatTok{0.8}\NormalTok{ <-}\StringTok{ }\NormalTok{int_}\FloatTok{0.8}\NormalTok{[[}\StringTok{"lnCVR"}\NormalTok{]] }
    
    \CommentTok{# prediction intervals}
\NormalTok{    MLMA_estimates_SMD_D_}\FloatTok{0.8}\NormalTok{ <-}\StringTok{ }\NormalTok{plyr}\OperatorTok{::}\KeywordTok{ldply}\NormalTok{(}\KeywordTok{lapply}\NormalTok{(smd_mods_D_}\FloatTok{0.8}\NormalTok{, }
                           \ControlFlowTok{function}\NormalTok{(x) }\KeywordTok{print}\NormalTok{(}\KeywordTok{mod_results}\NormalTok{(x, }\DataTypeTok{mod =} \StringTok{"Int"}\NormalTok{)))) }
\NormalTok{    MLMA_estimates_lnCVR_D_}\FloatTok{0.8}\NormalTok{ <-}\StringTok{ }\NormalTok{plyr}\OperatorTok{::}\KeywordTok{ldply}\NormalTok{(}\KeywordTok{lapply}\NormalTok{(lnCVR_mods_D_}\FloatTok{0.8}\NormalTok{, }
                              \ControlFlowTok{function}\NormalTok{(x) }\KeywordTok{print}\NormalTok{(}\KeywordTok{mod_results}\NormalTok{(x, }\DataTypeTok{mod =} \StringTok{"Int"}\NormalTok{))))}
\end{Highlighting}
\end{Shaded}

\subsection{Personality trait models with D
matrices}\label{personality-trait-models-with-d-matrices}

Model output is presented in Supplementary Tables S11-S13 in the
Supporting Information.

\begin{Shaded}
\begin{Highlighting}[]
\CommentTok{# 2. Personality Trait models}

  \CommentTok{# rho = 0.3}
\NormalTok{    pers_}\FloatTok{0.3}\NormalTok{ <-}\StringTok{ }\KeywordTok{fit_int_MLMAmodel_D_pers}\NormalTok{(pers_new, phylo_vcv, D_matrices_}\FloatTok{0.3}\NormalTok{)}
    
\NormalTok{    smd_mods_D_pers_}\FloatTok{0.3}\NormalTok{ <-}\StringTok{ }\NormalTok{pers_}\FloatTok{0.3}\NormalTok{[[}\StringTok{"SMD"}\NormalTok{]] }
\NormalTok{    lnCVR_mods_D_pers_}\FloatTok{0.3}\NormalTok{ <-}\StringTok{ }\NormalTok{pers_}\FloatTok{0.3}\NormalTok{[[}\StringTok{"lnCVR"}\NormalTok{]] }
  
    \CommentTok{# prediction intervals}
\NormalTok{    MLMA_estimates_SMD_pers_D_}\FloatTok{0.3}\NormalTok{ <-}\StringTok{ }\NormalTok{plyr}\OperatorTok{::}\KeywordTok{ldply}\NormalTok{(}\KeywordTok{lapply}\NormalTok{(smd_mods_D_pers_}\FloatTok{0.3}\NormalTok{, }
                           \ControlFlowTok{function}\NormalTok{(x) }\KeywordTok{print}\NormalTok{(}\KeywordTok{mod_results}\NormalTok{(x, }\DataTypeTok{mod =} \StringTok{"personality_trait"}\NormalTok{)))) }
\NormalTok{    MLMA_estimates_lnCVR_pers_D_}\FloatTok{0.3}\NormalTok{ <-}\StringTok{ }\NormalTok{plyr}\OperatorTok{::}\KeywordTok{ldply}\NormalTok{(}\KeywordTok{lapply}\NormalTok{(lnCVR_mods_D_pers_}\FloatTok{0.3}\NormalTok{, }
                              \ControlFlowTok{function}\NormalTok{(x) }\KeywordTok{print}\NormalTok{(}\KeywordTok{mod_results}\NormalTok{(x, }\DataTypeTok{mod =} \StringTok{"personality_trait"}\NormalTok{)))) }

  \CommentTok{# rho = 0.5}
\NormalTok{    pers_}\FloatTok{0.5}\NormalTok{ <-}\StringTok{ }\KeywordTok{fit_int_MLMAmodel_D_pers}\NormalTok{(pers_new, phylo_vcv, D_matrices_}\FloatTok{0.5}\NormalTok{)}
    
\NormalTok{    smd_mods_D_pers_}\FloatTok{0.5}\NormalTok{ <-}\StringTok{ }\NormalTok{pers_}\FloatTok{0.5}\NormalTok{[[}\StringTok{"SMD"}\NormalTok{]] }
\NormalTok{    lnCVR_mods_D_pers_}\FloatTok{0.5}\NormalTok{ <-}\StringTok{ }\NormalTok{pers_}\FloatTok{0.5}\NormalTok{[[}\StringTok{"lnCVR"}\NormalTok{]] }
  
    \CommentTok{# prediction intervals}
\NormalTok{    MLMA_estimates_SMD_pers_D_}\FloatTok{0.5}\NormalTok{ <-}\StringTok{ }\NormalTok{plyr}\OperatorTok{::}\KeywordTok{ldply}\NormalTok{(}\KeywordTok{lapply}\NormalTok{(smd_mods_D_pers_}\FloatTok{0.5}\NormalTok{, }
                           \ControlFlowTok{function}\NormalTok{(x) }\KeywordTok{print}\NormalTok{(}\KeywordTok{mod_results}\NormalTok{(x, }\DataTypeTok{mod =} \StringTok{"personality_trait"}\NormalTok{)))) }
\NormalTok{    MLMA_estimates_lnCVR_pers_D_}\FloatTok{0.5}\NormalTok{ <-}\StringTok{ }\NormalTok{plyr}\OperatorTok{::}\KeywordTok{ldply}\NormalTok{(}\KeywordTok{lapply}\NormalTok{(lnCVR_mods_D_pers_}\FloatTok{0.5}\NormalTok{, }
                              \ControlFlowTok{function}\NormalTok{(x) }\KeywordTok{print}\NormalTok{(}\KeywordTok{mod_results}\NormalTok{(x, }\DataTypeTok{mod =} \StringTok{"personality_trait"}\NormalTok{)))) }

  \CommentTok{# rho = 0.8}
\NormalTok{    pers_}\FloatTok{0.8}\NormalTok{ <-}\StringTok{ }\KeywordTok{fit_int_MLMAmodel_D_pers}\NormalTok{(pers_new, phylo_vcv, D_matrices_}\FloatTok{0.8}\NormalTok{)}
    
\NormalTok{    smd_mods_D_pers_}\FloatTok{0.8}\NormalTok{ <-}\StringTok{ }\NormalTok{pers_}\FloatTok{0.8}\NormalTok{[[}\StringTok{"SMD"}\NormalTok{]] }
\NormalTok{    lnCVR_mods_D_pers_}\FloatTok{0.8}\NormalTok{ <-}\StringTok{ }\NormalTok{pers_}\FloatTok{0.8}\NormalTok{[[}\StringTok{"lnCVR"}\NormalTok{]] }
  
    \CommentTok{# prediction intervals}
\NormalTok{    MLMA_estimates_SMD_pers_D_}\FloatTok{0.8}\NormalTok{ <-}\StringTok{ }\NormalTok{plyr}\OperatorTok{::}\KeywordTok{ldply}\NormalTok{(}\KeywordTok{lapply}\NormalTok{(smd_mods_D_pers_}\FloatTok{0.8}\NormalTok{, }
                           \ControlFlowTok{function}\NormalTok{(x) }\KeywordTok{print}\NormalTok{(}\KeywordTok{mod_results}\NormalTok{(x, }\DataTypeTok{mod =} \StringTok{"personality_trait"}\NormalTok{)))) }
\NormalTok{    MLMA_estimates_lnCVR_pers_D_}\FloatTok{0.8}\NormalTok{ <-}\StringTok{ }\NormalTok{plyr}\OperatorTok{::}\KeywordTok{ldply}\NormalTok{(}\KeywordTok{lapply}\NormalTok{(lnCVR_mods_D_pers_}\FloatTok{0.8}\NormalTok{, }
                              \ControlFlowTok{function}\NormalTok{(x) }\KeywordTok{print}\NormalTok{(}\KeywordTok{mod_results}\NormalTok{(x, }\DataTypeTok{mod =} \StringTok{"personality_trait"}\NormalTok{)))) }
\end{Highlighting}
\end{Shaded}

\subsection{Personality * SSD models}\label{personality-ssd-models}

These models were just to check since we don't really interpret the
interaction models.

\begin{Shaded}
\begin{Highlighting}[]
\CommentTok{# 3. Pers Trait * SSD models }
\CommentTok{# just use the full interaction models here since this is just a check}
\CommentTok{# won't bother with prediction intervals here since these models aren't really for interpretation}

    \CommentTok{# rho = 0.3}
\NormalTok{      ssd_}\FloatTok{0.3}\NormalTok{ <-}\StringTok{ }\KeywordTok{fit_int_MLMAmodel_D_pers_ssd}\NormalTok{(pers_new, phylo_vcv, D_matrices_}\FloatTok{0.3}\NormalTok{)}
      
\NormalTok{      split_taxa <-}\StringTok{ }\KeywordTok{split}\NormalTok{(pers_new, pers_new}\OperatorTok{$}\NormalTok{taxo_group)}
\NormalTok{      smd_mods_D_pers_ssd_}\FloatTok{0.3}\NormalTok{ <-}\StringTok{ }\NormalTok{ssd_}\FloatTok{0.3}\NormalTok{[[}\StringTok{"SMD"}\NormalTok{]] }
\NormalTok{      lnCVR_mods_D_pers_ssd_}\FloatTok{0.3}\NormalTok{ <-}\StringTok{ }\NormalTok{ssd_}\FloatTok{0.3}\NormalTok{[[}\StringTok{"lnCVR"}\NormalTok{]] }
  
    \CommentTok{# rho = 0.5}
\NormalTok{      ssd_}\FloatTok{0.5}\NormalTok{ <-}\StringTok{ }\KeywordTok{fit_int_MLMAmodel_D_pers_ssd}\NormalTok{(pers_new, phylo_vcv, D_matrices_}\FloatTok{0.5}\NormalTok{)}
    
\NormalTok{      split_taxa <-}\StringTok{ }\KeywordTok{split}\NormalTok{(pers_new, pers_new}\OperatorTok{$}\NormalTok{taxo_group)}
\NormalTok{      smd_mods_D_pers_ssd_}\FloatTok{0.5}\NormalTok{ <-}\StringTok{ }\NormalTok{ssd_}\FloatTok{0.5}\NormalTok{[[}\StringTok{"SMD"}\NormalTok{]] }
\NormalTok{      lnCVR_mods_D_pers_ssd_}\FloatTok{0.5}\NormalTok{ <-}\StringTok{ }\NormalTok{ssd_}\FloatTok{0.5}\NormalTok{[[}\StringTok{"lnCVR"}\NormalTok{]]   }
     
    \CommentTok{# rho = 0.8}
\NormalTok{      ssd_}\FloatTok{0.8}\NormalTok{ <-}\StringTok{ }\KeywordTok{fit_int_MLMAmodel_D_pers_ssd}\NormalTok{(pers_new, phylo_vcv, D_matrices_}\FloatTok{0.8}\NormalTok{)}
    
\NormalTok{      split_taxa <-}\StringTok{ }\KeywordTok{split}\NormalTok{(pers_new, pers_new}\OperatorTok{$}\NormalTok{taxo_group)}
\NormalTok{      smd_mods_D_pers_ssd_}\FloatTok{0.8}\NormalTok{ <-}\StringTok{ }\NormalTok{ssd_}\FloatTok{0.8}\NormalTok{[[}\StringTok{"SMD"}\NormalTok{]] }
\NormalTok{      lnCVR_mods_D_pers_ssd_}\FloatTok{0.8}\NormalTok{ <-}\StringTok{ }\NormalTok{ssd_}\FloatTok{0.8}\NormalTok{[[}\StringTok{"lnCVR"}\NormalTok{]] }
\end{Highlighting}
\end{Shaded}

\section{Sensitivity analyses - Publication
Bias}\label{sensitivity-analyses---publication-bias}

We can use: 1) funnel plots to look for asymmetry across all effect
sizes for both SMD and lnCVR, and 2) Egger's test which performs a
regression test on our funnel plots \ldots{} but is not useful when
there is high heterogeneity NOT caused by publication bias (which is the
case for our data).

Since our data has very high heterogeneity, we instead included the
inverse of the `effective sample size' as a moderator term in our full
model (personality trait model) to see if study precision is driving
effect size patterns. The logic here is that studies with low or high
precision can have a significant influence and so including precision as
a moderator will allow us to see if precision is significant (and which
direction). See Nakagawa et al. 2021 for more info (reference in main
text).

Model summaries are presented in Supplementary Table S18.

\begin{Shaded}
\begin{Highlighting}[]
\NormalTok{### NEW METHOD OF PUBLICATION BIAS FROM NAKAGAWA ET AL 2021 - PREPRINT}

    \CommentTok{# calculating the inverse of the 'effective sample size' to account for unbalanced sampling}
\NormalTok{    pers_new}\OperatorTok{$}\NormalTok{inv_n_tilda <-}\StringTok{ }\KeywordTok{with}\NormalTok{(pers_new, ((female_n }\OperatorTok{+}\StringTok{ }\NormalTok{male_n)}\OperatorTok{/}\NormalTok{(female_n}\OperatorTok{*}\NormalTok{male_n)))}
\NormalTok{    pers_new}\OperatorTok{$}\NormalTok{sqrt_inv_n_tilda <-}\StringTok{ }\KeywordTok{with}\NormalTok{(pers_new, (}\KeywordTok{sqrt}\NormalTok{(inv_n_tilda))) }\CommentTok{# use this in the model}

    \ControlFlowTok{if}\NormalTok{(rerun_models }\OperatorTok{==}\StringTok{ }\OtherTok{TRUE}\NormalTok{)\{}
\NormalTok{      MLMR_models_pers_pubbias <-}\StringTok{ }\KeywordTok{meta_model_fits}\NormalTok{(pers_new, phylo_vcv, }\DataTypeTok{type =} \StringTok{"pubbias"}\NormalTok{)}
      \KeywordTok{saveRDS}\NormalTok{(MLMR_models_pers_pubbias, }\StringTok{"./output/MLMR_models_pers_pubbias"}\NormalTok{)}
\NormalTok{    \} }\ControlFlowTok{else}\NormalTok{\{}
\NormalTok{     MLMR_models_pers_pubbias <-}\StringTok{ }\KeywordTok{readRDS}\NormalTok{(}\StringTok{"./output/MLMR_models_pers_pubbias"}\NormalTok{)}
\NormalTok{    \}}

    \CommentTok{# Extract the SMD and lnCVR results}
\NormalTok{    smd_mods_pubbias <-}\StringTok{ }\NormalTok{MLMR_models_pers_pubbias[}\StringTok{"SMD"}\NormalTok{,] }
   
\NormalTok{    lnCVR_mods_pubbias <-}\StringTok{ }\NormalTok{MLMR_models_pers_pubbias[}\StringTok{"lnCVR"}\NormalTok{,] }
\end{Highlighting}
\end{Shaded}

\section{Exploratory analyses - Heterogamety and taxo
group}\label{exploratory-analyses---heterogamety-and-taxo-group}

There was a trend for male mammals to be more variable than females and
female birds to be more variable than males, for some of the five
personality traits. To better compare the direction of these effect
sizes we decided post hoc to conduct an exploratory analysis with
personality trait type and taxonomic group as moderator terms to compare
birds and mammals (males homogametic or heterogametic, respectively). To
do this, we first combined the bird and mammal phylo correlation
matrices together (assuming no phylo heritability across the taxo groups
- since phylo did not really explain heterogeneity it shouldn't matter).
We then created an interaction MLMR model with personality trait * taxa
(no intercept) to get slope estimates for each of the traits for mammals
and birds seperately.

From this model, we then compared each of the five traits for birds and
mammals using a post hoc Tukey pairwise comparison to test whether birds
and mammals were significantly different from each other.

Model summaries are presented in Supplementary Table S19.

\begin{Shaded}
\begin{Highlighting}[]
\CommentTok{# install packages to make diagonal matrix and to make multiple comparisons }
  \KeywordTok{library}\NormalTok{(multcomp)}
  \KeywordTok{library}\NormalTok{(Matrix)}

\CommentTok{# Create block diag phylogeny}
\NormalTok{  phylogeny <-}\StringTok{ }\NormalTok{Matrix}\OperatorTok{::}\KeywordTok{bdiag}\NormalTok{(phylo_vcv_bird, phylo_vcv_mammal) }\CommentTok{# use this as the phylo vcv in the model}

    \CommentTok{# needs to have colnames for use in random effects model}
    \KeywordTok{dimnames}\NormalTok{(phylogeny) <-}\StringTok{ }\KeywordTok{Map}\NormalTok{(c, }\KeywordTok{dimnames}\NormalTok{(phylo_vcv_bird), }\KeywordTok{dimnames}\NormalTok{(phylo_vcv_mammal))}

    \CommentTok{# only include bird and mammal data}
\NormalTok{    pers_new_contrast <-}\StringTok{ }\KeywordTok{as.data.frame}\NormalTok{(pers_new }\OperatorTok
\StringTok{                                        }\KeywordTok{filter}\NormalTok{(taxo_group }\OperatorTok{==}\StringTok{"mammal"} \OperatorTok{|}\StringTok{ }\NormalTok{taxo_group }\OperatorTok{==}\StringTok{ "bird"}\NormalTok{) }\OperatorTok\StringTok{ }
\StringTok{                                        }\KeywordTok{mutate}\NormalTok{(}\DataTypeTok{sp_pers =} \KeywordTok{interaction}\NormalTok{(personality_trait,taxo_group)))}
    
    \CommentTok{# 1. intercept only model}
\NormalTok{    contrast_birdmammal_lncvr_int <-}\StringTok{ }\KeywordTok{rma.mv}\NormalTok{(CVR_yi }\OperatorTok{~}\StringTok{ }\NormalTok{taxo_group, }\DataTypeTok{V =}\NormalTok{ CVR_vi, }
                                                 \DataTypeTok{random =} \KeywordTok{list}\NormalTok{(}\OperatorTok{~}\DecValTok{1}\OperatorTok{|}\NormalTok{study_ID, }\OperatorTok{~}\DecValTok{1}\OperatorTok{|}\NormalTok{spp_name_phylo, }\OperatorTok{~}\DecValTok{1}\OperatorTok{|}\NormalTok{obs), }
                                                 \DataTypeTok{R =} \KeywordTok{list}\NormalTok{(}\DataTypeTok{spp_name_phylo=}\NormalTok{phylogeny), }\DataTypeTok{control=}\KeywordTok{list}\NormalTok{(}\DataTypeTok{optimizer=}\StringTok{"optim"}\NormalTok{), }
                                                 \DataTypeTok{test =} \StringTok{"t"}\NormalTok{, }\DataTypeTok{data =}\NormalTok{ pers_new_contrast) }
    

    \CommentTok{# 2. personality trait model  }
    \CommentTok{# creating the model - with pers trait and taxo group as mods }

    \CommentTok{#lnCVR model only}
\NormalTok{    contrast_birdmammal_lncvr <-}\StringTok{ }\KeywordTok{rma.mv}\NormalTok{(CVR_yi }\OperatorTok{~}\StringTok{ }\NormalTok{personality_trait}\OperatorTok{*}\NormalTok{taxo_group, }\DataTypeTok{V =}\NormalTok{ CVR_vi, }
                                                 \DataTypeTok{random =} \KeywordTok{list}\NormalTok{(}\OperatorTok{~}\DecValTok{1}\OperatorTok{|}\NormalTok{study_ID, }\OperatorTok{~}\DecValTok{1}\OperatorTok{|}\NormalTok{spp_name_phylo, }\OperatorTok{~}\DecValTok{1}\OperatorTok{|}\NormalTok{obs), }
                                                 \DataTypeTok{R =} \KeywordTok{list}\NormalTok{(}\DataTypeTok{spp_name_phylo=}\NormalTok{phylogeny), }\DataTypeTok{control=}\KeywordTok{list}\NormalTok{(}\DataTypeTok{optimizer=}\StringTok{"optim"}\NormalTok{), }
                                                 \DataTypeTok{test =} \StringTok{"t"}\NormalTok{, }\DataTypeTok{data =}\NormalTok{ pers_new_contrast)}
    
    \CommentTok{# model with interaction only to check output of model above}
\NormalTok{    contrast_birdmammal_lncvr_}\DecValTok{2}\NormalTok{ <-}\StringTok{ }\KeywordTok{rma.mv}\NormalTok{(CVR_yi }\OperatorTok{~}\StringTok{ }\NormalTok{sp_pers }\OperatorTok{-}\DecValTok{1}\NormalTok{, }\DataTypeTok{V =}\NormalTok{ CVR_vi, }
                                                 \DataTypeTok{random =} \KeywordTok{list}\NormalTok{(}\OperatorTok{~}\DecValTok{1}\OperatorTok{|}\NormalTok{study_ID, }\OperatorTok{~}\DecValTok{1}\OperatorTok{|}\NormalTok{spp_name_phylo, }\OperatorTok{~}\DecValTok{1}\OperatorTok{|}\NormalTok{obs), }
                                                 \DataTypeTok{R =} \KeywordTok{list}\NormalTok{(}\DataTypeTok{spp_name_phylo=}\NormalTok{phylogeny), }\DataTypeTok{control=}\KeywordTok{list}\NormalTok{(}\DataTypeTok{optimizer=}\StringTok{"optim"}\NormalTok{), }
                                                 \DataTypeTok{test =} \StringTok{"t"}\NormalTok{, }\DataTypeTok{data =}\NormalTok{ pers_new_contrast)}

\CommentTok{# multiple comparison using Tukey test}
  \KeywordTok{summary}\NormalTok{(}\KeywordTok{glht}\NormalTok{(contrast_birdmammal_lncvr, }\DataTypeTok{linfct =} \KeywordTok{cbind}\NormalTok{(}\KeywordTok{contrMat}\NormalTok{(}\KeywordTok{rep}\NormalTok{(}\DecValTok{1}\OperatorTok{:}\DecValTok{10}\NormalTok{), }\DataTypeTok{type =} \StringTok{"Tukey"}\NormalTok{))), }\DataTypeTok{test=}\KeywordTok{adjusted}\NormalTok{(}\StringTok{"fdr"}\NormalTok{))}

\CommentTok{# here we are only interested in the comparisons between mammals and birds, so: 1-6 (activity), 2-7 (aggression), 3-8 (boldness), 4-9 (exploration), and 5-10 (sociality)}
\end{Highlighting}
\end{Shaded}

\begin{center}\rule{0.5\linewidth}{0.5pt}\end{center}

\section{Plots}\label{plots}

\subsection{Orchard plots of effect sizes from personality trait
models}\label{orchard-plots-of-effect-sizes-from-personality-trait-models}

These plots use the orchaRd package to generate pretty plots where each
effect size (k) is a point on the plot.

\subsubsection{lnCVR}\label{lncvr}

\begin{Shaded}
\begin{Highlighting}[]
\CommentTok{# create objects of each of the models first}

\CommentTok{# lnCVR}

  \CommentTok{# Bird lnCVR}
\NormalTok{  bird_lncvr <-}\StringTok{ }\KeywordTok{orchard_plot}\NormalTok{(lnCVR_mods_pers[[}\DecValTok{1}\NormalTok{]], }\DataTypeTok{mod =} \StringTok{"personality_trait"}\NormalTok{, }\DataTypeTok{xlab =} \StringTok{"log Coefficient of Variance (lnCVR)"}\NormalTok{, }\DataTypeTok{angle =} \DecValTok{45}\NormalTok{, }\DataTypeTok{alpha =} \FloatTok{0.5}\NormalTok{, }\DataTypeTok{transfm =} \StringTok{"none"}\NormalTok{)}
  \CommentTok{# Fish lnCVR}
\NormalTok{  fish_lncvr <-}\StringTok{ }\KeywordTok{orchard_plot}\NormalTok{(lnCVR_mods_pers[[}\DecValTok{2}\NormalTok{]], }\DataTypeTok{mod =} \StringTok{"personality_trait"}\NormalTok{, }\DataTypeTok{xlab =} \StringTok{"log Coefficient of Variance (lnCVR)"}\NormalTok{, }\DataTypeTok{angle =} \DecValTok{45}\NormalTok{, }\DataTypeTok{alpha =} \FloatTok{0.5}\NormalTok{, }\DataTypeTok{transfm =} \StringTok{"none"}\NormalTok{)}
  \CommentTok{# Invert lnCVR}
\NormalTok{  invert_lncvr<-}\StringTok{ }\KeywordTok{orchard_plot}\NormalTok{(lnCVR_mods_pers[[}\DecValTok{3}\NormalTok{]], }\DataTypeTok{mod =} \StringTok{"personality_trait"}\NormalTok{, }\DataTypeTok{xlab =} \StringTok{"log Coefficient of Variance (lnCVR)"}\NormalTok{, }\DataTypeTok{angle =} \DecValTok{45}\NormalTok{, }\DataTypeTok{alpha =} \FloatTok{0.5}\NormalTok{, }\DataTypeTok{transfm =} \StringTok{"none"}\NormalTok{)}
  \CommentTok{# Mammal lnCVR}
\NormalTok{  mammal_lncvr <-}\StringTok{ }\KeywordTok{orchard_plot}\NormalTok{(lnCVR_mods_pers[[}\DecValTok{4}\NormalTok{]], }\DataTypeTok{mod =} \StringTok{"personality_trait"}\NormalTok{, }\DataTypeTok{xlab =} \StringTok{"log Coefficient of Variance (lnCVR)"}\NormalTok{, }\DataTypeTok{angle =} \DecValTok{45}\NormalTok{, }\DataTypeTok{alpha =} \FloatTok{0.5}\NormalTok{, }\DataTypeTok{transfm =} \StringTok{"none"}\NormalTok{)}
  \CommentTok{# Reptile lnCVR}
\NormalTok{  reptile_lncvr <-}\StringTok{ }\KeywordTok{orchard_plot}\NormalTok{(lnCVR_mods_pers[[}\DecValTok{5}\NormalTok{]], }\DataTypeTok{mod =} \StringTok{"personality_trait"}\NormalTok{, }\DataTypeTok{xlab =} \StringTok{"log Coefficient of Variance (lnCVR)"}\NormalTok{, }\DataTypeTok{angle =} \DecValTok{45}\NormalTok{, }\DataTypeTok{alpha =} \FloatTok{0.5}\NormalTok{, }\DataTypeTok{transfm =} \StringTok{"none"}\NormalTok{)}
\end{Highlighting}
\end{Shaded}

\subsubsection{SMD}\label{smd}

\begin{Shaded}
\begin{Highlighting}[]
  \CommentTok{# Bird SMD}
\NormalTok{  bird_SMD <-}\StringTok{ }\KeywordTok{orchard_plot}\NormalTok{(smd_mods_pers[[}\DecValTok{1}\NormalTok{]], }\DataTypeTok{mod =} \StringTok{"personality_trait"}\NormalTok{, }\DataTypeTok{xlab =} \StringTok{"Standardised mean difference"}\NormalTok{, }\DataTypeTok{angle =} \DecValTok{45}\NormalTok{, }\DataTypeTok{alpha =} \FloatTok{0.5}\NormalTok{, }\DataTypeTok{transfm =} \StringTok{"none"}\NormalTok{) }
  \CommentTok{# Fish SMD}
\NormalTok{  fish_SMD <-}\StringTok{ }\KeywordTok{orchard_plot}\NormalTok{(smd_mods_pers[[}\DecValTok{2}\NormalTok{]], }\DataTypeTok{mod =} \StringTok{"personality_trait"}\NormalTok{, }\DataTypeTok{xlab =} \StringTok{"Standardised mean difference"}\NormalTok{, }\DataTypeTok{angle =} \DecValTok{45}\NormalTok{, }\DataTypeTok{alpha =} \FloatTok{0.5}\NormalTok{, }\DataTypeTok{transfm =} \StringTok{"none"}\NormalTok{)}
  \CommentTok{# Invert SMD}
\NormalTok{  invert_SMD<-}\StringTok{ }\KeywordTok{orchard_plot}\NormalTok{(smd_mods_pers[[}\DecValTok{3}\NormalTok{]], }\DataTypeTok{mod =} \StringTok{"personality_trait"}\NormalTok{, }\DataTypeTok{xlab =} \StringTok{"Standardised mean difference"}\NormalTok{, }\DataTypeTok{angle =} \DecValTok{45}\NormalTok{, }\DataTypeTok{alpha =} \FloatTok{0.5}\NormalTok{, }\DataTypeTok{transfm =} \StringTok{"none"}\NormalTok{)}
  \CommentTok{# Mammal SMD}
\NormalTok{  mammal_SMD <-}\StringTok{ }\KeywordTok{orchard_plot}\NormalTok{(smd_mods_pers[[}\DecValTok{4}\NormalTok{]], }\DataTypeTok{mod =} \StringTok{"personality_trait"}\NormalTok{, }\DataTypeTok{xlab =} \StringTok{"Standardised mean difference"}\NormalTok{, }\DataTypeTok{angle =} \DecValTok{45}\NormalTok{, }\DataTypeTok{alpha =} \FloatTok{0.5}\NormalTok{, }\DataTypeTok{transfm =} \StringTok{"none"}\NormalTok{)}
  \CommentTok{# Reptile SMD}
\NormalTok{  reptile_SMD <-}\StringTok{ }\KeywordTok{orchard_plot}\NormalTok{(smd_mods_pers[[}\DecValTok{5}\NormalTok{]], }\DataTypeTok{mod =} \StringTok{"personality_trait"}\NormalTok{, }\DataTypeTok{xlab =} \StringTok{"Standardised mean difference"}\NormalTok{, }\DataTypeTok{angle =} \DecValTok{45}\NormalTok{, }\DataTypeTok{alpha =} \FloatTok{0.5}\NormalTok{, }\DataTypeTok{transfm =} \StringTok{"none"}\NormalTok{)}
\end{Highlighting}
\end{Shaded}

Putting the SMD and lnCVR plots together

Endotherms:

\begin{Shaded}
\begin{Highlighting}[]
  \CommentTok{# window size for orchard plots}
  \CommentTok{# the precision guides on the plots are a bit ugly, collect them to the side and crop them out   }
  
\NormalTok{## Mammals}
 
  \CommentTok{# dev.new(width=8,height=7,noRStudioGD = TRUE)}

\CommentTok{# place plots together}

\NormalTok{mammal_SMD }\OperatorTok{/}\StringTok{ }\NormalTok{mammal_lncvr }
\end{Highlighting}
\end{Shaded}

\includegraphics{meta-analysis-and-models_clean_files/figure-latex/orchard plots figs-1.pdf}

\begin{Shaded}
\begin{Highlighting}[]
\CommentTok{# ggsave("./figs/finished figs/mammal_effects.tiff", width = 8, height = 7, units = "in") #save image}

\NormalTok{## Birds}

\NormalTok{bird_SMD }\OperatorTok{/}\StringTok{ }\NormalTok{bird_lncvr}
\end{Highlighting}
\end{Shaded}

\includegraphics{meta-analysis-and-models_clean_files/figure-latex/orchard plots figs-2.pdf}

\begin{Shaded}
\begin{Highlighting}[]
\CommentTok{# ggsave("./figs/finished figs/bird_effects.tiff", width = 8, height = 7, units = "in") #save image}
\end{Highlighting}
\end{Shaded}

Ectotherms:

\begin{Shaded}
\begin{Highlighting}[]
\CommentTok{# window size a bit smaller for these guys}
    \CommentTok{# dev.new(width=8,height=5,noRStudioGD = TRUE)}

\NormalTok{## Reptiles and amphibians}

\NormalTok{reptile_SMD }\OperatorTok{/}\StringTok{ }\NormalTok{reptile_lncvr }\OperatorTok{/}\StringTok{ }\KeywordTok{plot_layout}\NormalTok{(}\DataTypeTok{guides =} \StringTok{'collect'}\NormalTok{)}
\end{Highlighting}
\end{Shaded}

\includegraphics{meta-analysis-and-models_clean_files/figure-latex/orchard plots figs again-1.pdf}

\begin{Shaded}
\begin{Highlighting}[]
\CommentTok{# ggsave("~/Documents/GitHub/sex_meta/figs/finished figs/rep_effects.tiff", width = 8, height = 5, units = "in")}

\NormalTok{## Fish}

\NormalTok{fish_SMD }\OperatorTok{/}\StringTok{ }\NormalTok{fish_lncvr }\OperatorTok{/}\StringTok{ }\KeywordTok{plot_layout}\NormalTok{(}\DataTypeTok{guides =} \StringTok{'collect'}\NormalTok{)}
\end{Highlighting}
\end{Shaded}

\includegraphics{meta-analysis-and-models_clean_files/figure-latex/orchard plots figs again-2.pdf}

\begin{Shaded}
\begin{Highlighting}[]
\CommentTok{# ggsave("~/Documents/GitHub/sex_meta/figs/finished figs/fish_effects.tiff", width = 8, height = 5, units = "in")}

\NormalTok{## Invertebrates}

\NormalTok{invert_SMD }\OperatorTok{/}\StringTok{ }\NormalTok{invert_lncvr }\OperatorTok{/}\StringTok{ }\KeywordTok{plot_layout}\NormalTok{(}\DataTypeTok{guides =} \StringTok{'collect'}\NormalTok{)}
\end{Highlighting}
\end{Shaded}

\includegraphics{meta-analysis-and-models_clean_files/figure-latex/orchard plots figs again-3.pdf}

\begin{Shaded}
\begin{Highlighting}[]
\CommentTok{# ggsave("~/Documents/GitHub/sex_meta/figs/finished figs/invert_effects.tiff", width = 8, height = 5, units = "in")}
\end{Highlighting}
\end{Shaded}

The precision guides will get cropped out when joining the orchard plots
and phylogenies together.

\begin{center}\rule{0.5\linewidth}{0.5pt}\end{center}

\subsection{Phylogenetic trees with
heatmaps}\label{phylogenetic-trees-with-heatmaps}

Using ggtree to plot lots of complex data onto phylogenetic trees see:
\url{https://guangchuangyu.github.io/ggtree-book/chapter-ggtree.html}
for more information about using ggtree

\begin{Shaded}
\begin{Highlighting}[]
\CommentTok{# install ggtree using this method:}
  \KeywordTok{source}\NormalTok{(}\StringTok{"https://bioconductor.org/biocLite.R"}\NormalTok{)}
\NormalTok{  BiocManager}\OperatorTok{::}\KeywordTok{install}\NormalTok{(}\StringTok{"ggtree"}\NormalTok{)}
\end{Highlighting}
\end{Shaded}

\begin{verbatim}
## 
## The downloaded binary packages are in
##  /var/folders/0b/pxghylq157gfhs1vrzdpx2gc0000gq/T//RtmpyX6Pi3/downloaded_packages
\end{verbatim}

\begin{Shaded}
\begin{Highlighting}[]
  \KeywordTok{library}\NormalTok{(ggtree)}

\CommentTok{# load organised SSD data using figs_data.csv }
\NormalTok{  figs_data <-}\StringTok{ }\KeywordTok{read.csv}\NormalTok{(}\StringTok{"./data/figs_data.csv"}\NormalTok{, }\DataTypeTok{stringsAsFactors =} \OtherTok{FALSE}\NormalTok{)}
\end{Highlighting}
\end{Shaded}

\subsection{bird tree}\label{bird-tree}

\begin{Shaded}
\begin{Highlighting}[]
\CommentTok{# subset dataset to include only birds}
\NormalTok{ bird_data <-}\StringTok{ }\KeywordTok{as.data.frame}\NormalTok{(figs_data }\OperatorTok
\StringTok{    }\KeywordTok{filter}\NormalTok{(taxo_group }\OperatorTok{==}\StringTok{ "bird"}\NormalTok{))}

\CommentTok{# setting up the basic tree structure}
  \CommentTok{# load tree}
\NormalTok{    birdtree <-}\StringTok{ }\KeywordTok{read.tree}\NormalTok{(}\StringTok{"./trees/bird_species.nwk"}\NormalTok{)}

  \CommentTok{# prune tree to get rid of species we no longer have data for}
\NormalTok{    pruned.birdtree <-}\StringTok{ }\KeywordTok{drop.tip}\NormalTok{(birdtree, }\KeywordTok{setdiff}\NormalTok{(birdtree}\OperatorTok{$}\NormalTok{tip.label, bird_data}\OperatorTok{$}\NormalTok{spp_name_phylo)) }

  \CommentTok{# remove underscores from tip labels}
\NormalTok{    pruned.birdtree}\OperatorTok{$}\NormalTok{tip.label =}\StringTok{ }\KeywordTok{gsub}\NormalTok{(}\StringTok{"_"}\NormalTok{, }\StringTok{" "}\NormalTok{, pruned.birdtree}\OperatorTok{$}\NormalTok{tip.label)}
   
  \CommentTok{# remove underscores from species name in our species data list}
\NormalTok{    bird_data}\OperatorTok{$}\NormalTok{spp_name_phylo =}\StringTok{ }\KeywordTok{gsub}\NormalTok{(}\StringTok{"_"}\NormalTok{, }\StringTok{" "}\NormalTok{, bird_data}\OperatorTok{$}\NormalTok{spp_name_phylo) }

  \CommentTok{# set row names}
    \KeywordTok{row.names}\NormalTok{(bird_data) <-}\StringTok{ }\NormalTok{bird_data}\OperatorTok{$}\NormalTok{spp_name_phylo }

  \CommentTok{# define objects for the plot}
\NormalTok{    species <-}\StringTok{ }\NormalTok{pruned.birdtree}\OperatorTok{$}\NormalTok{tip.label}

    \KeywordTok{rownames}\NormalTok{(bird_data) <-}\StringTok{ }\NormalTok{pruned.birdtree}\OperatorTok{$}\NormalTok{tip.label }

\CommentTok{# set window size}
\CommentTok{# dev.new(width=8, height=8,noRStudioGD = TRUE) #opens quartz window of set size}

\CommentTok{# now need to make a matrix of effect sizes (n) for each species for each personality trait to add to our plot!}
    \CommentTok{# subset dataset}
\NormalTok{     pers_bird <-}\StringTok{ }\KeywordTok{as.data.frame}\NormalTok{(pers_new }\OperatorTok
\StringTok{     }\KeywordTok{filter}\NormalTok{(taxo_group }\OperatorTok{==}\StringTok{ "bird"}\NormalTok{)) }
  
    \CommentTok{# make this a matrix-style dataframe}
\NormalTok{      pers_bird <-}\StringTok{ }\KeywordTok{data.frame}\NormalTok{(pers_bird }\OperatorTok
\StringTok{      }\KeywordTok{group_by}\NormalTok{(spp_name_phylo, personality_trait) }\OperatorTok
\StringTok{      }\KeywordTok{summarise}\NormalTok{(}\DataTypeTok{n =} \KeywordTok{n}\NormalTok{()))}
\end{Highlighting}
\end{Shaded}

\begin{verbatim}
## `summarise()` has grouped output by 'spp_name_phylo'. You can override using the `.groups` argument.
\end{verbatim}

\begin{Shaded}
\begin{Highlighting}[]
      \CommentTok{# remove underscores species names}
\NormalTok{      pers_bird}\OperatorTok{$}\NormalTok{spp_name_phylo =}\StringTok{ }\KeywordTok{gsub}\NormalTok{(}\StringTok{"_"}\NormalTok{, }\StringTok{" "}\NormalTok{, pers_bird}\OperatorTok{$}\NormalTok{spp_name_phylo)}
      
      \CommentTok{# create matrix}
\NormalTok{      pers_bird <-}\StringTok{ }\KeywordTok{data.frame}\NormalTok{(pers_bird }\OperatorTok\StringTok{ }
\StringTok{                                }\KeywordTok{spread}\NormalTok{(personality_trait, n, }\DataTypeTok{fill =} \DecValTok{0}\NormalTok{))}
      
      \CommentTok{# set species name as row name for matrix}
      \KeywordTok{row.names}\NormalTok{(pers_bird) <-}\StringTok{ }\NormalTok{pers_bird}\OperatorTok{$}\NormalTok{spp_name_phylo }
    
      
\NormalTok{      pers_bird <-}\StringTok{ }\NormalTok{pers_bird[,}\DecValTok{2}\OperatorTok{:}\DecValTok{6}\NormalTok{]}

  \CommentTok{# matrix    }
\NormalTok{    birds_matrix <-}\StringTok{ }\KeywordTok{data.matrix}\NormalTok{(pers_bird) }
    
\CommentTok{# FINAL TREE}
  \CommentTok{# making the tree}
\NormalTok{p_b1 <-}\StringTok{ }\KeywordTok{ggtree}\NormalTok{(pruned.birdtree, }\DataTypeTok{size =} \FloatTok{0.3}\NormalTok{, }\DataTypeTok{layout =} \StringTok{'circular'}\NormalTok{, }\DataTypeTok{branch.length =} \StringTok{'none'}\NormalTok{) }\OperatorTok\StringTok{ }\NormalTok{bird_data }\OperatorTok{+}\StringTok{ }
\StringTok{  }\KeywordTok{xlim}\NormalTok{(}\OperatorTok{-}\DecValTok{40}\NormalTok{, }\OtherTok{NA}\NormalTok{) }\OperatorTok{+}\StringTok{ }
\StringTok{  }\KeywordTok{geom_tippoint}\NormalTok{(}\KeywordTok{aes}\NormalTok{(}\DataTypeTok{color=}\NormalTok{SSD_index)) }\OperatorTok{+}\StringTok{ }
\StringTok{  }\KeywordTok{scale_color_gradient2}\NormalTok{(}\DataTypeTok{midpoint =} \DecValTok{0}\NormalTok{, }\DataTypeTok{low =} \StringTok{"red3"}\NormalTok{, }\DataTypeTok{mid =} \StringTok{"seashell2"}\NormalTok{, }\DataTypeTok{high =} \StringTok{"deepskyblue2"}\NormalTok{) }\OperatorTok{+}\StringTok{ }
\StringTok{  }\KeywordTok{geom_tiplab2}\NormalTok{(}\DataTypeTok{size =} \FloatTok{2.2}\NormalTok{, }\DataTypeTok{offset =} \DecValTok{4}\NormalTok{, }\DataTypeTok{colour =} \StringTok{"black"}\NormalTok{, }\DataTypeTok{fontface =} \StringTok{"italic"}\NormalTok{) }\OperatorTok{+}
\StringTok{  }\KeywordTok{theme}\NormalTok{(}\DataTypeTok{legend.position =} \StringTok{'right'}\NormalTok{)}
\end{Highlighting}
\end{Shaded}

\begin{verbatim}
## Warning: `data_frame()` was deprecated in tibble 1.1.0.
## Please use `tibble()` instead.
## This warning is displayed once every 8 hours.
## Call `lifecycle::last_lifecycle_warnings()` to see where this warning was generated.
\end{verbatim}

\begin{verbatim}
## Warning: `mutate_()` was deprecated in dplyr 0.7.0.
## Please use `mutate()` instead.
## See vignette('programming') for more help
## This warning is displayed once every 8 hours.
## Call `lifecycle::last_lifecycle_warnings()` to see where this warning was generated.
\end{verbatim}

\begin{Shaded}
\begin{Highlighting}[]
  \CommentTok{# adding heatmap of traits}
\NormalTok{p_b2 <-}\StringTok{ }\KeywordTok{gheatmap}\NormalTok{(p_b1, birds_matrix, }\DataTypeTok{offset=}\DecValTok{68}\NormalTok{, }\DataTypeTok{width=}\DecValTok{2}\NormalTok{, }\DataTypeTok{low =} \StringTok{"white"}\NormalTok{, }\DataTypeTok{high =} \StringTok{"mediumseagreen"}\NormalTok{, }\DataTypeTok{color=}\OtherTok{NULL}\NormalTok{,}
                 \DataTypeTok{colnames=}\NormalTok{T, }\DataTypeTok{colnames_angle =} \DecValTok{60}\NormalTok{, }\DataTypeTok{colnames_offset_y =}\NormalTok{ .}\DecValTok{1}\NormalTok{, }\DataTypeTok{colnames_offset_x =}\NormalTok{ .}\DecValTok{2}\NormalTok{) }\OperatorTok{+}
\StringTok{  }\KeywordTok{theme}\NormalTok{(}\DataTypeTok{plot.tag =} \KeywordTok{element_text}\NormalTok{(}\DataTypeTok{size =} \DecValTok{2}\NormalTok{, }\DataTypeTok{face =} \StringTok{"bold"}\NormalTok{),}
        \DataTypeTok{legend.text =} \KeywordTok{element_text}\NormalTok{(}\DataTypeTok{size =} \DecValTok{8}\NormalTok{))}

\NormalTok{p_b2}
\end{Highlighting}
\end{Shaded}

\includegraphics{meta-analysis-and-models_clean_files/figure-latex/unnamed-chunk-4-1.pdf}

\begin{Shaded}
\begin{Highlighting}[]
\CommentTok{# ggsave("./figs/finished figs/birdphylo.tiff", p_b2, width=8, height = 8, units = "in")}
\end{Highlighting}
\end{Shaded}

\subsection{mammals}\label{mammals-1}

\begin{Shaded}
\begin{Highlighting}[]
\CommentTok{# subset dataset to include only mammals}
\NormalTok{ mammal_data <-}\StringTok{ }\KeywordTok{as.data.frame}\NormalTok{(figs_data }\OperatorTok
\StringTok{    }\KeywordTok{filter}\NormalTok{(taxo_group }\OperatorTok{==}\StringTok{ "mammal"}\NormalTok{))}

\CommentTok{# setting up the basic tree structure}
 
  \CommentTok{# load tree, set node colours}
\NormalTok{    mammaltree <-}\StringTok{ }\KeywordTok{read.tree}\NormalTok{(}\StringTok{"./trees/mammal_species.nwk"}\NormalTok{)}

  \CommentTok{# prune tree to get rid of species we no longer have data for}
\NormalTok{    pruned.mammaltree <-}\StringTok{ }\KeywordTok{drop.tip}\NormalTok{(mammaltree, }\KeywordTok{setdiff}\NormalTok{(mammaltree}\OperatorTok{$}\NormalTok{tip.label, mammal_data}\OperatorTok{$}\NormalTok{spp_name_phylo)) }

  \CommentTok{# remove underscores from tip labels}
\NormalTok{    pruned.mammaltree}\OperatorTok{$}\NormalTok{tip.label =}\StringTok{ }\KeywordTok{gsub}\NormalTok{(}\StringTok{"_"}\NormalTok{, }\StringTok{" "}\NormalTok{, pruned.mammaltree}\OperatorTok{$}\NormalTok{tip.label)}

  \CommentTok{# set rownames for labelling tips}
    \KeywordTok{rownames}\NormalTok{(mammal_data) <-}\StringTok{ }\NormalTok{pruned.mammaltree}\OperatorTok{$}\NormalTok{tip.label}

  \CommentTok{# remove underscores from species name from mammal dataset}
\NormalTok{    mammal_data}\OperatorTok{$}\NormalTok{spp_name_phylo =}\StringTok{ }\KeywordTok{gsub}\NormalTok{(}\StringTok{"_"}\NormalTok{, }\StringTok{" "}\NormalTok{, mammal_data}\OperatorTok{$}\NormalTok{spp_name_phylo)}

  \CommentTok{# set row names}
    \KeywordTok{row.names}\NormalTok{(mammal_data) <-}\StringTok{ }\NormalTok{mammal_data}\OperatorTok{$}\NormalTok{spp_name_phylo }

\CommentTok{# set window}
\CommentTok{# dev.new(width=8,height=7,noRStudioGD = TRUE)}

\CommentTok{# make a matrix of effect sizes (n) for each species for each personality trait to add to our plot!}
    \CommentTok{# subset dataset}
\NormalTok{    pers_mammal <-}\StringTok{ }\KeywordTok{as.data.frame}\NormalTok{(pers_new }\OperatorTok
\StringTok{    }\KeywordTok{filter}\NormalTok{(taxo_group }\OperatorTok{==}\StringTok{ "mammal"}\NormalTok{)) }
  
  \CommentTok{# make this a matrix-style dataframe}
\NormalTok{      pers_mammal <-}\StringTok{ }\KeywordTok{data.frame}\NormalTok{(pers_mammal }\OperatorTok
\StringTok{      }\KeywordTok{group_by}\NormalTok{(spp_name_phylo, personality_trait) }\OperatorTok
\StringTok{      }\KeywordTok{summarise}\NormalTok{(}\DataTypeTok{n =} \KeywordTok{n}\NormalTok{()))}
\end{Highlighting}
\end{Shaded}

\begin{verbatim}
## `summarise()` has grouped output by 'spp_name_phylo'. You can override using the `.groups` argument.
\end{verbatim}

\begin{Shaded}
\begin{Highlighting}[]
      \CommentTok{# remove underscores species names}
\NormalTok{    pers_mammal}\OperatorTok{$}\NormalTok{spp_name_phylo =}\StringTok{ }\KeywordTok{gsub}\NormalTok{(}\StringTok{"_"}\NormalTok{, }\StringTok{" "}\NormalTok{, pers_mammal}\OperatorTok{$}\NormalTok{spp_name_phylo)  }
      
\NormalTok{    pers_mammal <-}\StringTok{ }\KeywordTok{data.frame}\NormalTok{(pers_mammal }\OperatorTok\StringTok{ }
\StringTok{                                }\KeywordTok{spread}\NormalTok{(personality_trait, n, }\DataTypeTok{fill =} \DecValTok{0}\NormalTok{))}
    
    \KeywordTok{row.names}\NormalTok{(pers_mammal) <-}\StringTok{ }\NormalTok{pers_mammal}\OperatorTok{$}\NormalTok{spp_name_phylo }
    
\NormalTok{    pers_mammal <-}\StringTok{ }\NormalTok{pers_mammal[,}\DecValTok{2}\OperatorTok{:}\DecValTok{6}\NormalTok{]}

  \CommentTok{# matrix    }
\NormalTok{    mammal_matrix <-}\StringTok{ }\KeywordTok{data.matrix}\NormalTok{(pers_mammal) }
    
  \CommentTok{# making the tree}
\NormalTok{p_m1 <-}\StringTok{ }\KeywordTok{ggtree}\NormalTok{(pruned.mammaltree, }\DataTypeTok{size =} \FloatTok{0.3}\NormalTok{, }\DataTypeTok{layout =} \StringTok{'circular'}\NormalTok{, }\DataTypeTok{branch.length =} \StringTok{'none'}\NormalTok{) }\OperatorTok\StringTok{ }\NormalTok{mammal_data }\OperatorTok{+}\StringTok{ }
\StringTok{  }\KeywordTok{geom_tippoint}\NormalTok{(}\KeywordTok{aes}\NormalTok{(}\DataTypeTok{color=}\NormalTok{SSD_index)) }\OperatorTok{+}\StringTok{ }
\StringTok{  }\KeywordTok{scale_color_gradient2}\NormalTok{(}\DataTypeTok{midpoint =} \DecValTok{0}\NormalTok{, }\DataTypeTok{low =} \StringTok{"red3"}\NormalTok{, }\DataTypeTok{mid =} \StringTok{"seashell2"}\NormalTok{, }\DataTypeTok{high =} \StringTok{"deepskyblue2"}\NormalTok{) }\OperatorTok{+}\StringTok{ }
\StringTok{  }\KeywordTok{geom_tiplab2}\NormalTok{(}\DataTypeTok{size =} \FloatTok{2.5}\NormalTok{, }\DataTypeTok{offset =} \DecValTok{2}\NormalTok{, }\DataTypeTok{colour =} \StringTok{"black"}\NormalTok{, }\DataTypeTok{fontface =} \StringTok{"italic"}\NormalTok{) }\OperatorTok{+}
\StringTok{  }\KeywordTok{theme}\NormalTok{(}\DataTypeTok{legend.position =} \StringTok{'right'}\NormalTok{)}

  \CommentTok{# adding heatmap of traits}
\NormalTok{p_m2 <-}\StringTok{ }\KeywordTok{gheatmap}\NormalTok{(p_m1, mammal_matrix, }\DataTypeTok{offset=}\DecValTok{32}\NormalTok{, }\DataTypeTok{width=}\FloatTok{1.3}\NormalTok{, }\DataTypeTok{low =} \StringTok{"white"}\NormalTok{, }\DataTypeTok{high =} \StringTok{"mediumseagreen"}\NormalTok{, }\DataTypeTok{color=}\OtherTok{NULL}\NormalTok{,}
                 \DataTypeTok{colnames=}\NormalTok{T, }\DataTypeTok{colnames_angle =} \DecValTok{60}\NormalTok{, }\DataTypeTok{colnames_offset_y =}\NormalTok{ .}\DecValTok{1}\NormalTok{, }\DataTypeTok{colnames_offset_x =}\NormalTok{ .}\DecValTok{2}\NormalTok{) }\OperatorTok{+}
\StringTok{  }\KeywordTok{theme}\NormalTok{(}\DataTypeTok{plot.tag =} \KeywordTok{element_text}\NormalTok{(}\DataTypeTok{size =} \DecValTok{9}\NormalTok{, }\DataTypeTok{face =} \StringTok{"bold"}\NormalTok{),}
        \DataTypeTok{legend.text =} \KeywordTok{element_text}\NormalTok{(}\DataTypeTok{size =} \DecValTok{8}\NormalTok{))}

\NormalTok{p_m2}
\end{Highlighting}
\end{Shaded}

\includegraphics{meta-analysis-and-models_clean_files/figure-latex/unnamed-chunk-5-1.pdf}

\begin{Shaded}
\begin{Highlighting}[]
\CommentTok{# ggsave("./figs/finished figs/mammalphylo.tiff", p_m2, width=8, height = 7, units = "in")}
\end{Highlighting}
\end{Shaded}

\subsection{reptiles}\label{reptiles}

\begin{Shaded}
\begin{Highlighting}[]
\CommentTok{# subset dataset to include only reptiles}
\NormalTok{ rep_data <-}\StringTok{ }\KeywordTok{as.data.frame}\NormalTok{(figs_data }\OperatorTok
\StringTok{    }\KeywordTok{filter}\NormalTok{(taxo_group }\OperatorTok{==}\StringTok{ "reptilia"}\NormalTok{))}
 
 \KeywordTok{row.names}\NormalTok{(rep_data) <-}\StringTok{ }\NormalTok{rep_data}\OperatorTok{$}\NormalTok{spp_name_phylo }
 
\CommentTok{# setting up the basic tree structure}
 
  \CommentTok{# load tree, set node colours}
\NormalTok{    reptree <-}\StringTok{ }\KeywordTok{read.tree}\NormalTok{(}\StringTok{"./trees/reptile_species.nwk"}\NormalTok{)}

  \CommentTok{# prune tree to get rid of species we no longer have data for}
\NormalTok{    pruned.reptree <-}\StringTok{ }\KeywordTok{drop.tip}\NormalTok{(reptree, }\KeywordTok{setdiff}\NormalTok{(reptree}\OperatorTok{$}\NormalTok{tip.label, rep_data}\OperatorTok{$}\NormalTok{spp_name_phylo)) }

  \CommentTok{# remove underscores from tip labels}
\NormalTok{    pruned.reptree}\OperatorTok{$}\NormalTok{tip.label =}\StringTok{ }\KeywordTok{gsub}\NormalTok{(}\StringTok{"_"}\NormalTok{, }\StringTok{" "}\NormalTok{, pruned.reptree}\OperatorTok{$}\NormalTok{tip.label)}

  \CommentTok{# set rownames for labelling tips}
    \KeywordTok{rownames}\NormalTok{(rep_data) <-}\StringTok{ }\NormalTok{pruned.reptree}\OperatorTok{$}\NormalTok{tip.label}

  \CommentTok{# remove underscores from species name from mammal dataset}
\NormalTok{    rep_data}\OperatorTok{$}\NormalTok{spp_name_phylo =}\StringTok{ }\KeywordTok{gsub}\NormalTok{(}\StringTok{"_"}\NormalTok{, }\StringTok{" "}\NormalTok{, rep_data}\OperatorTok{$}\NormalTok{spp_name_phylo)}

\CommentTok{# set window size}
\CommentTok{#dev.new(width=7,height=5,noRStudioGD = TRUE)}

 \CommentTok{# tree structure }
\NormalTok{p3 <-}\StringTok{ }\KeywordTok{ggtree}\NormalTok{(pruned.reptree, }\DataTypeTok{branch.length=}\StringTok{'none'}\NormalTok{, }\DataTypeTok{size =} \FloatTok{0.3}\NormalTok{, }\DataTypeTok{layout=}\StringTok{'circular'}\NormalTok{) }\OperatorTok\StringTok{ }\NormalTok{rep_data }\OperatorTok{+}\StringTok{ }
\StringTok{  }\KeywordTok{geom_tippoint}\NormalTok{(}\KeywordTok{aes}\NormalTok{(}\DataTypeTok{color=}\NormalTok{SSD_index)) }\OperatorTok{+}\StringTok{ }
\StringTok{  }\KeywordTok{scale_color_gradient2}\NormalTok{(}\DataTypeTok{midpoint =} \DecValTok{0}\NormalTok{, }\DataTypeTok{low =} \StringTok{"red3"}\NormalTok{, }\DataTypeTok{mid =} \StringTok{"seashell2"}\NormalTok{, }\DataTypeTok{high =} \StringTok{"deepskyblue2"}\NormalTok{) }\OperatorTok{+}\StringTok{ }
\StringTok{  }\KeywordTok{geom_tiplab2}\NormalTok{(}\DataTypeTok{align=}\NormalTok{T, }\DataTypeTok{linetype=}\OtherTok{NA}\NormalTok{, }\DataTypeTok{size=}\FloatTok{2.5}\NormalTok{, }\DataTypeTok{offset=}\DecValTok{4}\NormalTok{, }\DataTypeTok{hjust=}\DecValTok{0}\NormalTok{, }\DataTypeTok{colour =} \StringTok{"black"}\NormalTok{, }\DataTypeTok{fontface =} \StringTok{"italic"}\NormalTok{) }

\CommentTok{# make a matrix of effect sizes (n) for each species for each personality trait to add to our plot!}
  \CommentTok{# subset dataset}
\NormalTok{      pers_rep <-}\StringTok{ }\KeywordTok{as.data.frame}\NormalTok{(pers_new }\OperatorTok
\StringTok{      }\KeywordTok{filter}\NormalTok{(taxo_group }\OperatorTok{==}\StringTok{ "reptilia"}\NormalTok{)) }
  
  \CommentTok{# make this a matrix-style dataframe}
\NormalTok{      pers_rep <-}\StringTok{ }\KeywordTok{data.frame}\NormalTok{(pers_rep }\OperatorTok
\StringTok{      }\KeywordTok{group_by}\NormalTok{(spp_name_phylo, personality_trait) }\OperatorTok
\StringTok{      }\KeywordTok{summarise}\NormalTok{(}\DataTypeTok{n =} \KeywordTok{n}\NormalTok{()))}
\end{Highlighting}
\end{Shaded}

\begin{verbatim}
## `summarise()` has grouped output by 'spp_name_phylo'. You can override using the `.groups` argument.
\end{verbatim}

\begin{Shaded}
\begin{Highlighting}[]
  \CommentTok{# remove underscores from species name from mammal dataset}
\NormalTok{      pers_rep}\OperatorTok{$}\NormalTok{spp_name_phylo =}\StringTok{ }\KeywordTok{gsub}\NormalTok{(}\StringTok{"_"}\NormalTok{, }\StringTok{" "}\NormalTok{, pers_rep}\OperatorTok{$}\NormalTok{spp_name_phylo)}
          
\NormalTok{      pers_rep <-}\StringTok{ }\KeywordTok{data.frame}\NormalTok{(pers_rep }\OperatorTok\StringTok{ }
\StringTok{                                }\KeywordTok{spread}\NormalTok{(personality_trait, n, }\DataTypeTok{fill =} \DecValTok{0}\NormalTok{))}
      
    \KeywordTok{row.names}\NormalTok{(pers_rep) <-}\StringTok{ }\NormalTok{pers_rep}\OperatorTok{$}\NormalTok{spp_name_phylo }
    
\NormalTok{    pers_rep <-}\StringTok{ }\NormalTok{pers_rep[,}\DecValTok{2}\OperatorTok{:}\DecValTok{6}\NormalTok{]}

  \CommentTok{# matrix    }
\NormalTok{    rep_matrix <-}\StringTok{ }\KeywordTok{data.matrix}\NormalTok{(pers_rep) }
    
  \CommentTok{# add the heatmap data to our plot}
\NormalTok{rep_plot <-}\StringTok{ }\KeywordTok{gheatmap}\NormalTok{(p3, rep_matrix, }\DataTypeTok{offset =} \DecValTok{40}\NormalTok{, }\DataTypeTok{width =} \FloatTok{3.5}\NormalTok{,}
            \DataTypeTok{low =} \StringTok{"white"}\NormalTok{, }\DataTypeTok{high =} \StringTok{"mediumseagreen"}\NormalTok{, }\DataTypeTok{color=}\OtherTok{NULL}\NormalTok{, }
            \DataTypeTok{colnames_position=}\StringTok{"top"}\NormalTok{, }
            \DataTypeTok{colnames_angle=}\DecValTok{60}\NormalTok{, }\DataTypeTok{colnames_offset_y =} \DecValTok{0}\NormalTok{, }
             \DataTypeTok{hjust=}\DecValTok{0}\NormalTok{, }\DataTypeTok{font.size=}\DecValTok{3}\NormalTok{) }\CommentTok{#just not aligning properly }

\NormalTok{rep_plot}
\end{Highlighting}
\end{Shaded}

\includegraphics{meta-analysis-and-models_clean_files/figure-latex/unnamed-chunk-6-1.pdf}

\begin{Shaded}
\begin{Highlighting}[]
\CommentTok{# ggsave("./figs/finished figs/repphylo.tiff", rep_plot, width=7, height = 5, units = "in")}
\end{Highlighting}
\end{Shaded}

\subsection{fish}\label{fish-1}

\begin{Shaded}
\begin{Highlighting}[]
\CommentTok{# subset dataset to include only fish}
\NormalTok{ fish_data <-}\StringTok{ }\KeywordTok{as.data.frame}\NormalTok{(figs_data }\OperatorTok
\StringTok{    }\KeywordTok{filter}\NormalTok{(taxo_group }\OperatorTok{==}\StringTok{ "fish"}\NormalTok{))}
 
 \CommentTok{# window size}
 \CommentTok{#dev.new(width=8,height=6,noRStudioGD = TRUE)}
 
\CommentTok{# setting up the basic tree structure}
 
  \CommentTok{# load tree}
\NormalTok{fishtree <-}\StringTok{ }\KeywordTok{read.tree}\NormalTok{(}\StringTok{"./trees/fish_species.nwk"}\NormalTok{)}

\CommentTok{# prune tree to get rid of species we no longer have data for}
\NormalTok{  pruned.fishtree <-}\StringTok{ }\KeywordTok{drop.tip}\NormalTok{(fishtree, }\KeywordTok{setdiff}\NormalTok{(fishtree}\OperatorTok{$}\NormalTok{tip.label, fish_data}\OperatorTok{$}\NormalTok{spp_name_phylo)) }

\CommentTok{# remove underscores from tip labels}
\NormalTok{  pruned.fishtree}\OperatorTok{$}\NormalTok{tip.label =}\StringTok{ }\KeywordTok{gsub}\NormalTok{(}\StringTok{"_"}\NormalTok{, }\StringTok{" "}\NormalTok{, pruned.fishtree}\OperatorTok{$}\NormalTok{tip.label)}

\CommentTok{# set rownames for labelling tips}
  \KeywordTok{rownames}\NormalTok{(fish_data) <-}\StringTok{ }\NormalTok{pruned.fishtree}\OperatorTok{$}\NormalTok{tip.label}

\CommentTok{# remove underscores from species name from fish dataset}
\NormalTok{  fish_data}\OperatorTok{$}\NormalTok{spp_name_phylo =}\StringTok{ }\KeywordTok{gsub}\NormalTok{(}\StringTok{"_"}\NormalTok{, }\StringTok{" "}\NormalTok{, fish_data}\OperatorTok{$}\NormalTok{spp_name_phylo)}

  \KeywordTok{row.names}\NormalTok{(fish_data) <-}\StringTok{ }\NormalTok{fish_data}\OperatorTok{$}\NormalTok{spp_name_phylo }

\CommentTok{# make a matrix of effect sizes (n) for each species for each personality trait to add to our plot!}
    \CommentTok{# subset dataset}
\NormalTok{  pers_fish <-}\StringTok{ }\KeywordTok{as.data.frame}\NormalTok{(pers_new }\OperatorTok
\StringTok{  }\KeywordTok{filter}\NormalTok{(taxo_group }\OperatorTok{==}\StringTok{ "fish"}\NormalTok{)) }
  
  \CommentTok{# make this a matrix-style dataframe}
\NormalTok{      pers_fish <-}\StringTok{ }\KeywordTok{data.frame}\NormalTok{(pers_fish }\OperatorTok
\StringTok{      }\KeywordTok{group_by}\NormalTok{(spp_name_phylo, personality_trait) }\OperatorTok
\StringTok{      }\KeywordTok{summarise}\NormalTok{(}\DataTypeTok{n =} \KeywordTok{n}\NormalTok{()))}
\end{Highlighting}
\end{Shaded}

\begin{verbatim}
## `summarise()` has grouped output by 'spp_name_phylo'. You can override using the `.groups` argument.
\end{verbatim}

\begin{Shaded}
\begin{Highlighting}[]
      \CommentTok{# remove underscores from tip labels}
\NormalTok{    pers_fish}\OperatorTok{$}\NormalTok{spp_name_phylo =}\StringTok{ }\KeywordTok{gsub}\NormalTok{(}\StringTok{"_"}\NormalTok{, }\StringTok{" "}\NormalTok{, pers_fish}\OperatorTok{$}\NormalTok{spp_name_phylo)}
      
\NormalTok{      pers_fish <-}\StringTok{ }\KeywordTok{data.frame}\NormalTok{(pers_fish }\OperatorTok\StringTok{ }
\StringTok{                                }\KeywordTok{spread}\NormalTok{(personality_trait, n, }\DataTypeTok{fill =} \DecValTok{0}\NormalTok{))}
    
      \KeywordTok{row.names}\NormalTok{(pers_fish) <-}\StringTok{ }\NormalTok{pers_fish}\OperatorTok{$}\NormalTok{spp_name_phylo }
    
\NormalTok{    pers_fish <-}\StringTok{ }\NormalTok{pers_fish[,}\DecValTok{2}\OperatorTok{:}\DecValTok{6}\NormalTok{]}

  \CommentTok{# matrix    }
\NormalTok{    fish_matrix <-}\StringTok{ }\KeywordTok{data.matrix}\NormalTok{(pers_fish) }
    
    \CommentTok{# FINAL TREE   }
\NormalTok{  p_f1 <-}\StringTok{ }\KeywordTok{ggtree}\NormalTok{(pruned.fishtree, }\DataTypeTok{size =} \FloatTok{0.3}\NormalTok{, }\DataTypeTok{layout =} \StringTok{'circular'}\NormalTok{, }\DataTypeTok{branch.length =} \StringTok{'none'}\NormalTok{) }\OperatorTok\StringTok{ }\NormalTok{fish_data }\OperatorTok{+}\StringTok{ }
\StringTok{  }\KeywordTok{xlim}\NormalTok{(}\OperatorTok{-}\DecValTok{30}\NormalTok{, }\OtherTok{NA}\NormalTok{) }\OperatorTok{+}\StringTok{ }
\StringTok{  }\KeywordTok{geom_tippoint}\NormalTok{(}\KeywordTok{aes}\NormalTok{(}\DataTypeTok{color=}\NormalTok{SSD_index)) }\OperatorTok{+}\StringTok{ }
\StringTok{  }\KeywordTok{scale_color_gradient2}\NormalTok{(}\DataTypeTok{midpoint =} \DecValTok{0}\NormalTok{, }\DataTypeTok{low =} \StringTok{"red3"}\NormalTok{, }\DataTypeTok{mid =} \StringTok{"seashell2"}\NormalTok{, }\DataTypeTok{high =} \StringTok{"deepskyblue2"}\NormalTok{) }\OperatorTok{+}\StringTok{ }
\StringTok{  }\KeywordTok{geom_tiplab2}\NormalTok{(}\DataTypeTok{size =} \FloatTok{2.5}\NormalTok{, }\DataTypeTok{offset =} \DecValTok{6}\NormalTok{, }\DataTypeTok{colour =} \StringTok{"black"}\NormalTok{, }\DataTypeTok{fontface =} \StringTok{"italic"}\NormalTok{) }\OperatorTok{+}
\StringTok{  }\KeywordTok{theme}\NormalTok{(}\DataTypeTok{legend.position =} \StringTok{'right'}\NormalTok{)}
  
  \CommentTok{# add the heatmap data to our plot}
\NormalTok{     fish_plot2 <-}\StringTok{ }\KeywordTok{gheatmap}\NormalTok{(p_f1, fish_matrix, }\DataTypeTok{offset =} \DecValTok{170}\NormalTok{, }\DataTypeTok{width =} \FloatTok{5.5}\NormalTok{,}
         \DataTypeTok{low =} \StringTok{"white"}\NormalTok{, }\DataTypeTok{high =} \StringTok{"mediumseagreen"}\NormalTok{, }\DataTypeTok{color=}\OtherTok{NULL}\NormalTok{, }
         \DataTypeTok{colnames_position=}\StringTok{"bottom"}\NormalTok{, }
         \DataTypeTok{colnames_angle=}\DecValTok{60}\NormalTok{, }
         \DataTypeTok{hjust=}\DecValTok{0}\NormalTok{, }\DataTypeTok{font.size=}\DecValTok{3}\NormalTok{) }

\NormalTok{fish_plot2     }
\end{Highlighting}
\end{Shaded}

\includegraphics{meta-analysis-and-models_clean_files/figure-latex/unnamed-chunk-7-1.pdf}

\begin{Shaded}
\begin{Highlighting}[]
\CommentTok{# ggsave("./figs/finished figs/fishphylo.tiff", fish_plot2, width=8, height = 6, units = "in")}
\end{Highlighting}
\end{Shaded}

\subsection{inverts}\label{inverts-1}

\begin{Shaded}
\begin{Highlighting}[]
\CommentTok{# subset dataset to include only inverts}
\NormalTok{  invert_data <-}\StringTok{ }\KeywordTok{as.data.frame}\NormalTok{(figs_data }\OperatorTok
\StringTok{    }\KeywordTok{filter}\NormalTok{(taxo_group }\OperatorTok{==}\StringTok{ "invertebrate"}\NormalTok{))}
 
 \CommentTok{# setting up the basic tree structure}
 
  \CommentTok{# load tree, set node colours}
\NormalTok{  inverttree <-}\StringTok{ }\KeywordTok{read.tree}\NormalTok{(}\StringTok{"./trees/invert_species.nwk"}\NormalTok{)}

  \CommentTok{# prune tree to get rid of species we no longer have data for}
\NormalTok{  pruned.inverttree <-}\StringTok{ }\KeywordTok{drop.tip}\NormalTok{(inverttree, }\KeywordTok{setdiff}\NormalTok{(inverttree}\OperatorTok{$}\NormalTok{tip.label, invert_data}\OperatorTok{$}\NormalTok{spp_name_phylo)) }

  \CommentTok{# remove underscores from tip labels}
\NormalTok{  pruned.inverttree}\OperatorTok{$}\NormalTok{tip.label =}\StringTok{ }\KeywordTok{gsub}\NormalTok{(}\StringTok{"_"}\NormalTok{, }\StringTok{" "}\NormalTok{, pruned.inverttree}\OperatorTok{$}\NormalTok{tip.label)}

  \CommentTok{# remove underscores from dataset and fix row names}
\NormalTok{  invert_data}\OperatorTok{$}\NormalTok{spp_name_phylo =}\StringTok{ }\KeywordTok{gsub}\NormalTok{(}\StringTok{"_"}\NormalTok{, }\StringTok{" "}\NormalTok{, invert_data}\OperatorTok{$}\NormalTok{spp_name_phylo)}

  \KeywordTok{row.names}\NormalTok{(invert_data) <-}\StringTok{ }\NormalTok{invert_data}\OperatorTok{$}\NormalTok{spp_name_phylo }

\CommentTok{# set rownames for labelling tips}
  \KeywordTok{rownames}\NormalTok{(invert_data) <-}\StringTok{ }\NormalTok{pruned.inverttree}\OperatorTok{$}\NormalTok{tip.label}
  
\CommentTok{# dev.new(width=8,height=6,noRStudioGD = TRUE) }

 \CommentTok{# tree structure (cladogram, circular)}
\NormalTok{  p5 <-}\StringTok{ }\KeywordTok{ggtree}\NormalTok{(pruned.inverttree, }\DataTypeTok{branch.length=}\StringTok{'none'}\NormalTok{, }\DataTypeTok{layout=}\StringTok{'circular'}\NormalTok{) }\OperatorTok\StringTok{ }\NormalTok{invert_data }\OperatorTok{+}\StringTok{ }
\StringTok{  }\KeywordTok{geom_tippoint}\NormalTok{(}\KeywordTok{aes}\NormalTok{(}\DataTypeTok{color=}\NormalTok{SSD_index)) }\OperatorTok{+}\StringTok{ }
\StringTok{  }\KeywordTok{scale_color_gradient2}\NormalTok{(}\DataTypeTok{midpoint =} \DecValTok{0}\NormalTok{, }\DataTypeTok{low =} \StringTok{"red3"}\NormalTok{, }\DataTypeTok{mid =} \StringTok{"seashell2"}\NormalTok{, }\DataTypeTok{high =} \StringTok{"deepskyblue2"}\NormalTok{) }\OperatorTok{+}\StringTok{ }
\StringTok{  }\KeywordTok{geom_tiplab2}\NormalTok{(}\DataTypeTok{align=}\NormalTok{T, }\DataTypeTok{linetype=}\OtherTok{NA}\NormalTok{, }\DataTypeTok{size=}\FloatTok{2.2}\NormalTok{, }\DataTypeTok{offset=}\DecValTok{2}\NormalTok{, }\DataTypeTok{fontface =} \StringTok{"italic"}\NormalTok{) }\OperatorTok{+}
\StringTok{  }\KeywordTok{theme}\NormalTok{(}\DataTypeTok{legend.position =} \StringTok{"right"}\NormalTok{)}

\CommentTok{# make a matrix of effect sizes (n) for each species for each personality trait to add to our plot!}
    \CommentTok{# subset dataset}
\NormalTok{  pers_invert <-}\StringTok{ }\KeywordTok{as.data.frame}\NormalTok{(pers_new }\OperatorTok
\StringTok{  }\KeywordTok{filter}\NormalTok{(taxo_group }\OperatorTok{==}\StringTok{ "invertebrate"}\NormalTok{)) }
  
  \CommentTok{# make this a matrix-style dataframe}
\NormalTok{      pers_invert <-}\StringTok{ }\KeywordTok{data.frame}\NormalTok{(pers_invert }\OperatorTok
\StringTok{      }\KeywordTok{group_by}\NormalTok{(spp_name_phylo, personality_trait) }\OperatorTok
\StringTok{      }\KeywordTok{summarise}\NormalTok{(}\DataTypeTok{n =} \KeywordTok{n}\NormalTok{()))}
\end{Highlighting}
\end{Shaded}

\begin{verbatim}
## `summarise()` has grouped output by 'spp_name_phylo'. You can override using the `.groups` argument.
\end{verbatim}

\begin{Shaded}
\begin{Highlighting}[]
  \CommentTok{# remove underscores from dataset and fix row names}
\NormalTok{      pers_invert}\OperatorTok{$}\NormalTok{spp_name_phylo =}\StringTok{ }\KeywordTok{gsub}\NormalTok{(}\StringTok{"_"}\NormalTok{, }\StringTok{" "}\NormalTok{, pers_invert}\OperatorTok{$}\NormalTok{spp_name_phylo)}
       
\NormalTok{      pers_invert <-}\StringTok{ }\KeywordTok{data.frame}\NormalTok{(pers_invert }\OperatorTok\StringTok{ }
\StringTok{                                }\KeywordTok{spread}\NormalTok{(personality_trait, n, }\DataTypeTok{fill =} \DecValTok{0}\NormalTok{))}
      
      \KeywordTok{row.names}\NormalTok{(pers_invert) <-}\StringTok{ }\NormalTok{pers_invert}\OperatorTok{$}\NormalTok{spp_name_phylo }
    
\NormalTok{      pers_invert <-}\StringTok{ }\NormalTok{pers_invert[,}\DecValTok{2}\OperatorTok{:}\DecValTok{6}\NormalTok{]}

  \CommentTok{# matrix    }
\NormalTok{    invert_matrix <-}\StringTok{ }\KeywordTok{data.matrix}\NormalTok{(pers_invert) }
    
  \CommentTok{# add the heatmap data to our plot}
\NormalTok{     invertplot <-}\StringTok{ }\KeywordTok{gheatmap}\NormalTok{(p5, invert_matrix, }\DataTypeTok{offset =} \DecValTok{40}\NormalTok{, }\DataTypeTok{width =} \FloatTok{1.5}\NormalTok{, }
         \DataTypeTok{low =} \StringTok{"white"}\NormalTok{, }\DataTypeTok{high =} \StringTok{"mediumseagreen"}\NormalTok{, }\DataTypeTok{color=}\OtherTok{NULL}\NormalTok{, }
         \DataTypeTok{colnames_position=}\StringTok{"bottom"}\NormalTok{, }
         \DataTypeTok{colnames_angle=}\DecValTok{45}\NormalTok{, }\DataTypeTok{colnames_offset_y =} \DecValTok{0}\NormalTok{, }
         \DataTypeTok{hjust=}\DecValTok{0}\NormalTok{, }\DataTypeTok{font.size=}\FloatTok{2.5}\NormalTok{)}

\NormalTok{invertplot     }
\end{Highlighting}
\end{Shaded}

\includegraphics{meta-analysis-and-models_clean_files/figure-latex/unnamed-chunk-8-1.pdf}

\begin{Shaded}
\begin{Highlighting}[]
\CommentTok{# save plot}
\CommentTok{# ggsave("./figs/finished figs/invertphylo.tiff", invertplot, width=8, height = 6, units = "in")}
\end{Highlighting}
\end{Shaded}

These plots were edited together outside of R with the addition of
creative commons animal silhouettes from PhyloPic to create Figures 2-6.
Figure 1, the PRISMA diagram, was created using sankeymatic.com

\begin{center}\rule{0.5\linewidth}{0.5pt}\end{center}

\end{document}
